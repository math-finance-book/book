If the market value of the asset exceeds the exercise price, then we say the call option is in the money.  Buying a call option is a way to bet on the upside of the underlying asset.  

A put option \index{put option} is the right to sell an asset at a pre-specified (exercise, strike) price.  Buying a put is a way to bet on an asset price becoming low (similar to shorting).  A put option is in the money if the exercise price exceeds the value of the asset.  Both puts and calls are potentially valuable and hence the buyer of a put or call must pay the seller.  

A long put option provides insurance to someone who is long the underlying asset, because it guarantees that the asset can always be sold at the strike price of the put (of course, it can be sold at the market price, if that is higher than the strike of the put).  Symmetrically, a long call option provides insurance to someone who is short the underlying asset. The terminology in option markets reflects the parallels between options and insurance contracts.  In particular, the seller of an option is said to write the option and the compensation (price) he receives from the buyer is called the option premium, \index{option premium} just as an insurance company writes insurance contracts in exchange for premium income.  Calculating the price at which one should be willing to trade an option is the main topic of this book.

It is important to recognize the different situations of someone who is short a call option and someone who is long a put.  Both positions are bets on the downside of the asset.  Both the investor who is short a call and the investor who is long a put may eventually sell the underlying asset  and receive the exercise price in exchange.  However, the investor who is long a put has an option to sell the asset at the exercise price and the investor who is short a call has an \emph{obligation} to sell the asset at the exercise price, should the counterparty choose to exercise the call.  Thus, the investor who is long a put will be selling at the exercise price when it is profitable to do so, whereas the investor who is short a call will be selling at the exercise price when it is unprofitable.  The buyer of a put must pay the premium to the seller; he then profits if the asset price is low, with his maximum possible profit being quite large (the maximum value is attained when the market value of the underlying asset reaches zero).  In contrast, the seller of a call receives premium income, and the premium is his maximum possible profit, whereas his potential losses are unbounded.  Thus, these are very different positions.

Individuals who sell calls usually sell out-of-the-money covered calls.  Covered \index{covered call} means that they own the underlying asset and can therefore deliver the underlying if the call is exercised without incurring any further expense---they experience only an opportunity cost in delivering it for less than the market price.^[In contrast, one who sells a call without owning the underlying is said to sell a naked call.  \index{naked call] The seller of a naked call, or the seller of a put, must post margin, just like a short seller of stocks, in order to ensure that he can meet his obligation.  However, this does not apply to sellers of covered calls.}  A  call being out of the money implies that the price of the underlying must rise before the call would be exercised against the seller; thus, the seller of an out-of-the money covered call still has some potential for profit from the underlying.  In addition, of course, the seller receives the premium income from the call.  Institutions often follow this strategy also, using the premium income to enhance their return from the underlying.     One can hedge a short call without owning a full share of the underlying asset, if one is able to rebalance the hedge over time.  Calculating such hedges is another of the principal topics of this book.

In a certain sense, option markets are zero-sum games.  The profit earned by one counterparty to an option transaction is a loss suffered by the other.  However, options can allow for an increase in the welfare of all investors by improving the allocation of risk.  A producer who must purchase a certain input may buy a call option, giving him the right to buy the input at a fixed price.  This caps his expense.  The seller of the call now bears the risk that the input price will be high---in this case, the option will be exercised and he will be forced to sell at a price below the market price.  It may be that the seller is in a better position to bear the risk (for example, he may have less of the risk in his portfolio) and the option transaction may thereby improve the allocation of risks across investors.  The similarity to insurance should be apparent.

Quite complex bets or hedges can be created by combining options.  For example, a long call and put with the same strike price is called a straddle.  \index{straddle} Such a portfolio is (almost) always in the money.  It is in fact a bet on volatility---a big move in the underlying asset value away from the exercise price will lead to either the call or put having a high value.  Another important example of an option portfolio is a collar.  \index{collar} A collar consists of a long put and a short call, or a short call and a long put, with the options having the same maturity.  As mentioned before, a long put provides insurance to someone who is long the underlying asset.  Selling a call provides premium income that can be used to offset the cost of the put (the most popular type of collar is a zero-cost collar: \index{zero-cost collar} a collar in which the premium of the call is equal to the premium of the put).  The cost of selling a call for an owner of the underlying is that it sells off the upside of the underlying asset---if the value of the asset exceeds the strike price of the call, then the call will be exercised and the underlying asset must be delivered for the strike price (rather than the higher market price).  Thus, one can purchase the downside insurance provided by a long put by selling part of the upside potential of the asset, rather than paying the cost of the insurance out of pocket.  There are many other examples of option portfolios that could be given.  

Some puts and calls are traded on exchanges.  In this case, the exchange clearinghouse steps between the buyer and seller and becomes the counterparty to both the buyer and seller.  This eliminates the risk that the seller might default on his obligation when the buyer chooses to exercise his option.  If the owner of an option chooses to exercise, the clearinghouse randomly chooses someone who is short the option to fulfill the obligation.  Most exchange traded options are never exercised, because any gain on a long contract can be captured by selling the contract at the market price, thus cancelling the position.  Obviously, however, the right to exercise is essential, because it determines the market price.  Puts and calls are also transacted over the counter, \index{over the counter} which means that they are private contracts of the counterparties.  Moreover, puts and calls are embedded in many other financial instruments.  A prosaic but important example is that most homeowners have the right to pay off their mortgages early.  This means they have call options on their mortgages, with exercise price equal to the remaining mortgage principal.  Similarly, callable bonds can be redeemed early by the company issuing them, convertible bonds have  embedded call options on the company's stock (which are exercised by converting the bonds) and there are many, many other examples.  Puts and calls also exist outside financial markets.  For example, a company may begin manufacturing a new product at a small scale; if the product is successful, the scale can be expanded.  In this case, the company buys a call option on large-scale production with the premium being the cost of launching small-scale production.  Adapting the methods developed for financial options to value such real options is an important and growing field. \index{real option}

### Exercise Policies for Calls and Puts

It may be rational to exercise a call if the asset value exceeds the exercise price.  Thus, denoting the price of the asset by $S$ and the exercise price by $K$, the owner of a call option can profit by $S-K$ dollars by exercising the option when $S>K$.  If $S<K$, exercise would be irrational.  Thus, the payoff to the owner of the call option is^[We use the standard notation:  $\max(a,b)$ denotes the larger of $a$ and $b$ and $\min(a,b)$ denotes the smaller.] $\max(0,S-K)$.  It has been said that timing is everything, and the timing here should be made clearer.  The simplest type of option is called a European option.  \index{European option} A European option has a finite lifetime and can only be exercised at its maturity date.  For a European call option, the exercise strategy just described is the optimal one, with $S$ representing the asset price at the maturity date of the option.  Equally, if not more, important are American options, \index{American option} which can be exercised at any time before maturity.   

For an American call option, the exercise strategy just described is the optimal one at the maturity date, but it may also be optimal to exercise prior to maturity.  \index{early exercise}  Let $K$ denote the exercise price, $T$ the date the option matures, and $S_T$ the price of the underlying asset at date $t \leq T$.  The intrinsic value \index{intrinsic value} of the call option at date $t$ is defined to be $\max(0,S_T-K)$.  One would of course never exercise unless the intrinsic value is positive---i.e., unless the option is in the money.  Moreover, if the asset does not pay a dividend (or other type of cash flow) prior to the option maturity then one should not exercise in any circumstances prior to maturity.  This is captured in the saying:  calls are better alive than dead.  Exercise being suboptimal is equivalent  to the value of the option exceeding the intrinsic value.

The principle that calls on non-dividend-paying assets are better alive than dead follows from two facts: (i) it is generally a good thing (in financial markets as well as in life) to keep one's options open, and (ii) early exercise implies early payment of the exercise price and hence foregone interest.  The usual protest that is heard when this statement is made is that one should surely exercise if he expects the stock price to plummet, because by exercising (and then selling the stock acquired) one can lock in the current stock price rather than waiting for it to fall, in which case the option will surely be worth less.  This intuition is a reasonable one, but it ignores the fact that the investor could short sell the stock if he expects it to plummet---he doesn't need to exercise the option to lock in the current stock price.  In fact, shorting the stock and retaining the option is always better than exercising, assuming the underlying asset does not pay a dividend.  

Specifically, suppose an investor considers exercising at date $t$.  As an alternative to exercising early, consider shorting the stock at date $t$ and retaining the option.  This is always better than exercising at date $t$, because the short position can  be covered (the stock can be purchased and returned to the lender to cancel the short position) at cost $K$ at date $T$ by exercising the option, and paying $K$ at date $T$ is better than paying it at date $t$, given that interest rates must be nonnegative.  To be more precise, note that exercise at date $t$ produces  $S_T-K$ dollars at date $t$.  On the other hand, retaining the option, shorting the stock at date $t$, and covering the short either by exercising the option or buying the stock in the market (whichever is cheaper) produces $S_T$ dollars at date $t$ and 
$$\max(0,S_T-K) - S_T = \max(-S_T, -K) = -\min(S_T,K) \geq -K$$
dollars at date $T$.  If $S_T>K$, one has $-K$ dollars at date $T$, in which case retaining the option has been superior due to the time value of money.  Furthermore,  if $S_T<K$, the strategy of retaining the option and shorting the stock produces $-S_T > -K$ dollars at date $T$, so retaining the option is superior due both to flexibility (waiting until $T$ to decide whether to exercise turns out to be better than committing at date $t$) and because of the time value of money.^[Recall that we are assuming investors earn interest on the proceeds of short sales; otherwise, the $S_T$ dollars earned from exercising the option and selling the stock will be worth more than the $S_T$ dollars earned from shorting the stock.  In this case, early exercise could be optimal.  However, assuming institutional investors can earn interest on the proceeds of shorts,  such investors should prefer owning the option and shorting the stock to exercising.  This means  they should bid up the price of the option to the point where it exceeds the value $S_T-K$ of exercise.  If this is the case, then an investor who cannot earn interest on the proceeds of shorts should simply sell the option in the market rather than exercise it.  Thus, a sufficient condition for calls to be better alive than dead is that there be some investors who can earn interest on the proceeds of shorts.  This type of reasoning is possible for each situation in this book where the assumption of earning interest on margin deposits is important, and we will not deal with it in this much detail again.]

Early exercise of a call option can be optimal when the underlying asset pays a dividend.  The above analysis does not apply in this case, because paying the dividend to the lender of the stock is an additional cost for the strategy of retaining the option and shorting the stock.  If the dividend is so small that it cannot offset the time value of money on the exercise price, then early exercise will not be optimal.  In other cases, deriving the optimal exercise strategy is a complicated problem that we will first begin to study in @sec-c_introcomputation.

A European put option will be exercised at its maturity $T$ if the price $S_T$ of the underlying asset is below the exercise price $K$.  In general, the value at maturity can be expressed as $\max(0,K-S_T)$.  Early exercise of an American put can be optimal, regardless of whether the underlying pays a dividend.  While it is valuable to keep one's options open (for puts as well as calls) the time value of money works in the opposite direction for puts.  Early exercise of a put option implies early receipt of the exercise price, and it is better to receive cash earlier rather than later. In general, whether early exercise is optimal depends on how deeply the option is in the money---if the underlying asset price is sufficiently low, then it will be fairly certain that exercise will be optimal, whether earlier or late; in this case, one should exercise earlier to earn interest on the exercise price.  How low it should be to justify early exercise depends on the interest rate (a higher rate makes the time-value-of-money issue more important, leading to earlier exercise) and the volatility of the underlying asset price (a lower volatility reduces the value of keeping one's options open, leading also to earlier exercise).  We will begin to study the valuation of American puts in @sec-c_introcomputation also.

### Compounding Interest

During most of the first two parts of the book (the only exception being @sec-c_forwardexchange) we will assume there is a risk-free asset earning a constant rate of return.  \index{risk-free asset} For simplicity, we will specify the rate of return as a continuously compounded rate.  \index{continuously compounded interest} For example, if the annual rate with annual compounding is $r_a$, then the corresponding continuously compounded rate is $r$ defined as $r = \log (1+r_a)$, where $\log$ denotes the natural logarithm function.  This means that the gross return over a year (one plus the rate of return) is $\mathrm{e}^r = 1+r_a$.  More generally, an investment of $x$ dollars for a time period of length $T$ (we measure time in years, so, e.g., a six-month investment would mean $T=0.5$) will result in the ownership of $x\mathrm{e}^{rT}$ dollars at the end of the time period.  

Expressing the interest rate as a continuously compounded rate enables us to avoid having to specify in each instance whether the rate is for annual compounding, semi-annual compounding, monthly compounding, etc.  For example, the meaning of an annualized rate $r_s$ for semi-annual compounding is that an investment of $x$ dollars will grow over a year to $x(1+r_s/2)^2$.  The equivalent continuously compounded rate is defined as $r = \log (1+r_s/2)^2$, and in terms of this rate we can say that the investment will grow in six months to $x\mathrm{e}^{0.5 r}$ and that it will grow in one year to $x\mathrm{e}^r$.  We can interpret this rate as being continuously compounded because compounding $n$ times per year at an annualized rate of $r$ results in \$1 growing in a year to $(1+r/n)^n$ and
$$\lim_{n \rightarrow \infty} \left(1+\frac{r}{n}\right)^n = \mathrm{e}^r\;\;.$$
To develop pricing and hedging formulas for derivative securities, it is a great convenience to assume that investors can trade continuously in time.  This requires us to assume also that returns are computed continuously.  In the case of a risk-free investment of $x(t)$ dollars at any date $t$ at a continuously compounded rate of $r$, we will say that the interest earned in an instant $\mathrm{d} t$ is $x(t)r\,\mathrm{d} t$ dollars.  This is only meaningful when we accumulate the interest over a non-infinitesimal period of time.  So consider investing $x(0)$ dollars at time 0 and reinvesting interest in the risk-free asset over a time period of length $T$.  Let $x(t)$ denote the account balance at date $t$, for $0\leq t \leq T$.  The change in the account balance in each instant is the interest earned, so we have
$\mathrm{d} x(t) = x(t)r\,\mathrm{d} t$.  The real meaning of this equation is that $x(t)$ satisfies the differential equation
$$\frac{\mathrm{d} x(t)}{\mathrm{d} t} = x(t)r\; ,$$
and it is well known (and easy to verify) that the solution is
$$x(t) = x(0)\mathrm{e}^{rt}\; ,$$
leading to an account balance at the end of the time period of $x(T) = x(0)\mathrm{e}^{rT}$.  Thus, the statement that the interest earned in an instant $\mathrm{d} t$ is $x(t)r\,\mathrm{d} t$ is equivalent to the statement that interest is continuously compounded at the rate $r$.

In the last part of the book, we will drop the assumption that the risk-free asset earns a constant rate of return.  In this case, we will still generally assume that there is a risk-free asset for very short-term investments (i.e., for investments with infinitesimal durations!).  We will let $r(t)$ denote the risk-free rate for an instantaneous investment at date $t$. This means that an investment of $x(t)$ dollars at date $t$ in the risk-free asset earns interest in an instant $\mathrm{d} t$ equal to $x(t)r(t)\,\mathrm{d} t$.  Consider again an investment of $x(0)$ dollars at date 0 in this instantaneously risk-free asset with interest reinvested and let $x(t)$ denote the account balance at date $t$.  Then $x(t)$ must satisfy the differential equation
$$\frac{\mathrm{d} x(t)}{\mathrm{d} t} = x(t)r(t)\;\;.$$
The solution of this differential equation is
$$x(t) = x(0)\exp\left(\int_0^t r(s)\,ds\right)\;\;.$$
The expression $\int_0^t r(s)\,ds$ can be interpreted as a continuous sum over time of the rates of interest $r(s)$ earned at times $s$ between 0 and $t$.  If these rates are all the same, say equal to $r$, then $\int_0^t r(s)\,ds = rt$ and our compounding factor $\exp\left(\int_0^t r(s)\,ds\right)$ is $\mathrm{e}^{rt}$ as before.  


## Fundamental Concepts {#sec-s_fundamentalconcepts}

### Longs, Shorts, and Margin

In financial markets, the owner of an asset is said to be long the asset. \index{long}  If person A owes something to person B, the debt is an asset to person B but a liability to person A.  One also says that person A is short the asset.  \index{short} For example, if someone borrows money and invests the money in stocks, then the individual is short cash and long stocks.  

One must invest some of one's own money when borrowing money to buy stocks.  For example, an individual could invest \$600, borrow \$400, and buy \$1000 of stock.  The \$600 is called the margin \index{margin} posted by the investor, and buying stocks in this way is called buying on margin.  The investor, or the portfolio, is also said to be levered, because buying \$1000 of stock with only a \$600 investment amplifies the risk and return per dollar of investment.  \index{leverage} On a percentage basis, we would say the account has 60\% margin, the 60\% being the ratio of the equity (assets minus liabilities = \$1000 of stock minus \$400 debt) to the assets (\$1000 of stock).   If the value of the stock drops sufficiently far, then it may become doubtful whether the investor can repay the \$400.  In this case, the investor must either sell the stock or invest more of his own funds (i.e., he receives a margin call).  \index{margin call} In other words, in actual markets there are margin requirements, \index{margin requirement} that specify a minimum percent margin an investor must have initially (when borrowing money) and a minimum percent margin the investor must maintain.  

Rather than borrowing money to buy stocks, an investor can do the opposite---he can borrow stocks to buy money.  In this case, buying money means selling the borrowed stocks for cash.  Such an investor will be short stocks and long cash.  This is called short selling (or, more briefly, shorting) stocks.  \index{short selling} For example, suppose individual A borrows 100 shares of stock from individual B and then sells them to individual C.  Both B and C are long the 100 shares and A is short, so the net long position is $2 \times 100 - 100$, which is the original 100 shares that B was long.  A short seller of stocks must pay to the lender of the stocks any dividends that are paid on the stock.  In our example, both B and C own the 100 shares so both expect to receive dividends.  The company will pay dividends only to C, and A must pay the dividends to~B.

Of course, investors always wish to buy low and sell high.  The usual method is to buy stocks and hope they rise.  An investor who short sells also wishes to buy low and sell high, but he reverses the order---he sells first and then hopes the stocks fall.  The risk is that the stocks will instead rise, which will increase the value of his liability (short stock position) without increasing the value of his assets (long cash position), thus putting him under water.  To shield the lender of the stocks from this risk, a short seller must also invest some of his own funds, and this amount is again called the investor's margin.  For example, an investor might invest \$600, and borrow and sell \$1000 of stock.  In this case, the investor will be long \$1600 cash and short \$1000 worth of stock.  His equity is \$600 and his percent margin is calculated as \$600/\$1000 = 60\%.  Again, there are typically both initial and maintenance margin requirements.  An additional feature of short selling for small individual investors is that they typically will not earn interest on the proceeds of the short sale (the \$1000 cash obtained from selling stocks in the above example).  

In this book, we will assume there is a single risk-free rate at which one can both borrow and lend.  Moreover, we will assume that investors earn this rate on margin deposits, including the proceeds of short sales (and including any margin that may be required when buying and selling forward and futures contracts).  Thus, investors gain from buying on margin if the asset return is sure to exceed the risk-free rate, and they gain from short selling if the return on an asset is sure to be below the risk-free rate.   These assumptions are not reasonable for small individual investors, but they are fairly reasonable for institutional investors.  We will assume that no asset has a return that is certain to be above the risk-free rate nor certain to be below the risk-free rate, because institutional investors could arbitrage \index{arbitrage} such guaranteed high-return or guaranteed low-return assets.

### Calls and Puts

Call and put options are the basic derivative securities and the building blocks of many others.  A derivative security is a security the value of which depends upon another security. \index{derivative security}
A call option \index{call option} is the right to buy an asset at a pre-specified price.  The pre-specified price is called the exercise price, the strike price, or simply the strike.  \index{exercise price} \index{strike price} We will often call the asset a stock, but there are options on many other types of assets also, and everything we say will be applicable to those as well.^[One caveat is that by asset we mean something that can be stored; thus, for example, electricity is, practically speaking, not an asset.]  The asset to which the call option pertains is called the underlying asset, \index{underlying asset} or, more briefly, the underlying.  If the market value of the asset exceeds the exercise price, then we say the call option is in the money.  Buying a call option is a way to bet on the upside of the underlying asset.  

A put option \index{put option} is the right to sell an asset at a pre-specified (exercise, strike) price.  Buying a put is a way to bet on an asset price becoming low (similar to shorting).  A put option is in the money if the exercise price exceeds the value of the asset.  Both puts and calls are potentially valuable and hence the buyer of a put or call must pay the seller.  

A long put option provides insurance to someone who is long the underlying asset, because it guarantees that the asset can always be sold at the strike price of the put (of course, it can be sold at the market price, if that is higher than the strike of the put).  Symmetrically, a long call option provides insurance to someone who is short the underlying asset. The terminology in option markets reflects the parallels between options and insurance contracts.  In particular, the seller of an option is said to write the option and the compensation (price) he receives from the buyer is called the option premium, \index{option premium} just as an insurance company writes insurance contracts in exchange for premium income.  Calculating the price at which one should be willing to trade an option is the main topic of this book.

It is important to recognize the different situations of someone who is short a call option and someone who is long a put.  Both positions are bets on the downside of the asset.  Both the investor who is short a call and the investor who is long a put may eventually sell the underlying asset  and receive the exercise price in exchange.  However, the investor who is long a put has an option to sell the asset at the exercise price and the investor who is short a call has an \emph{obligation} to sell the asset at the exercise price, should the counterparty choose to exercise the call.  Thus, the investor who is long a put will be selling at the exercise price when it is profitable to do so, whereas the investor who is short a call will be selling at the exercise price when it is unprofitable.  The buyer of a put must pay the premium to the seller; he then profits if the asset price is low, with his maximum possible profit being quite large (the maximum value is attained when the market value of the underlying asset reaches zero).  In contrast, the seller of a call receives premium income, and the premium is his maximum possible profit, whereas his potential losses are unbounded.  Thus, these are very different positions.

Individuals who sell calls usually sell out-of-the-money covered calls.  Covered \index{covered call} means that they own the underlying asset and can therefore deliver the underlying if the call is exercised without incurring any further expense---they experience only an opportunity cost in delivering it for less than the market price.^[In contrast, one who sells a call without owning the underlying is said to sell a naked call.  \index{naked call] The seller of a naked call, or the seller of a put, must post margin, just like a short seller of stocks, in order to ensure that he can meet his obligation.  However, this does not apply to sellers of covered calls.}  A  call being out of the money implies that the price of the underlying must rise before the call would be exercised against the seller; thus, the seller of an out-of-the money covered call still has some potential for profit from the underlying.  In addition, of course, the seller receives the premium income from the call.  Institutions often follow this strategy also, using the premium income to enhance their return from the underlying.     One can hedge a short call without owning a full share of the underlying asset, if one is able to rebalance the hedge over time.  Calculating such hedges is another of the principal topics of this book.

In a certain sense, option markets are zero-sum games.  The profit earned by one counterparty to an option transaction is a loss suffered by the other.  However, options can allow for an increase in the welfare of all investors by improving the allocation of risk.  A producer who must purchase a certain input may buy a call option, giving him the right to buy the input at a fixed price.  This caps his expense.  The seller of the call now bears the risk that the input price will be high---in this case, the option will be exercised and he will be forced to sell at a price below the market price.  It may be that the seller is in a better position to bear the risk (for example, he may have less of the risk in his portfolio) and the option transaction may thereby improve the allocation of risks across investors.  The similarity to insurance should be apparent.

Quite complex bets or hedges can be created by combining options.  For example, a long call and put with the same strike price is called a straddle.  \index{straddle} Such a portfolio is (almost) always in the money.  It is in fact a bet on volatility---a big move in the underlying asset value away from the exercise price will lead to either the call or put having a high value.  Another important example of an option portfolio is a collar.  \index{collar} A collar consists of a long put and a short call, or a short call and a long put, with the options having the same maturity.  As mentioned before, a long put provides insurance to someone who is long the underlying asset.  Selling a call provides premium income that can be used to offset the cost of the put (the most popular type of collar is a zero-cost collar: \index{zero-cost collar} a collar in which the premium of the call is equal to the premium of the put).  The cost of selling a call for an owner of the underlying is that it sells off the upside of the underlying asset---if the value of the asset exceeds the strike price of the call, then the call will be exercised and the underlying asset must be delivered for the strike price (rather than the higher market price).  Thus, one can purchase the downside insurance provided by a long put by selling part of the upside potential of the asset, rather than paying the cost of the insurance out of pocket.  There are many other examples of option portfolios that could be given.  

Some puts and calls are traded on exchanges.  In this case, the exchange clearinghouse steps between the buyer and seller and becomes the counterparty to both the buyer and seller.  This eliminates the risk that the seller might default on his obligation when the buyer chooses to exercise his option.  If the owner of an option chooses to exercise, the clearinghouse randomly chooses someone who is short the option to fulfill the obligation.  Most exchange traded options are never exercised, because any gain on a long contract can be captured by selling the contract at the market price, thus cancelling the position.  Obviously, however, the right to exercise is essential, because it determines the market price.  Puts and calls are also transacted over the counter, \index{over the counter} which means that they are private contracts of the counterparties.  Moreover, puts and calls are embedded in many other financial instruments.  A prosaic but important example is that most homeowners have the right to pay off their mortgages early.  This means they have call options on their mortgages, with exercise price equal to the remaining mortgage principal.  Similarly, callable bonds can be redeemed early by the company issuing them, convertible bonds have  embedded call options on the company's stock (which are exercised by converting the bonds) and there are many, many other examples.  Puts and calls also exist outside financial markets.  For example, a company may begin manufacturing a new product at a small scale; if the product is successful, the scale can be expanded.  In this case, the company buys a call option on large-scale production with the premium being the cost of launching small-scale production.  Adapting the methods developed for financial options to value such real options is an important and growing field. \index{real option}

### Exercise Policies for Calls and Puts

It may be rational to exercise a call if the asset value exceeds the exercise price.  Thus, denoting the price of the asset by $S$ and the exercise price by $K$, the owner of a call option can profit by $S-K$ dollars by exercising the option when $S>K$.  If $S<K$, exercise would be irrational.  Thus, the payoff to the owner of the call option is^[We use the standard notation:  $\max(a,b)$ denotes the larger of $a$ and $b$ and $\min(a,b)$ denotes the smaller.] $\max(0,S-K)$.  It has been said that timing is everything, and the timing here should be made clearer.  The simplest type of option is called a European option.  \index{European option} A European option has a finite lifetime and can only be exercised at its maturity date.  For a European call option, the exercise strategy just described is the optimal one, with $S$ representing the asset price at the maturity date of the option.  Equally, if not more, important are American options, \index{American option} which can be exercised at any time before maturity.   

For an American call option, the exercise strategy just described is the optimal one at the maturity date, but it may also be optimal to exercise prior to maturity.  \index{early exercise}  Let $K$ denote the exercise price, $T$ the date the option matures, and $S(t)$ the price of the underlying asset at date $t \leq T$.  The intrinsic value \index{intrinsic value} of the call option at date $t$ is defined to be $\max(0,S(t)-K)$.  One would of course never exercise unless the intrinsic value is positive---i.e., unless the option is in the money.  Moreover, if the asset does not pay a dividend (or other type of cash flow) prior to the option maturity then one should not exercise in any circumstances prior to maturity.  This is captured in the saying:  calls are better alive than dead.  Exercise being suboptimal is equivalent  to the value of the option exceeding the intrinsic value.

The principle that calls on non-dividend-paying assets are better alive than dead follows from two facts: (i) it is generally a good thing (in financial markets as well as in life) to keep one's options open, and (ii) early exercise implies early payment of the exercise price and hence foregone interest.  The usual protest that is heard when this statement is made is that one should surely exercise if he expects the stock price to plummet, because by exercising (and then selling the stock acquired) one can lock in the current stock price rather than waiting for it to fall, in which case the option will surely be worth less.  This intuition is a reasonable one, but it ignores the fact that the investor could short sell the stock if he expects it to plummet---he doesn't need to exercise the option to lock in the current stock price.  In fact, shorting the stock and retaining the option is always better than exercising, assuming the underlying asset does not pay a dividend.  

Specifically, suppose an investor considers exercising at date $t$.  As an alternative to exercising early, consider shorting the stock at date $t$ and retaining the option.  This is always better than exercising at date $t$, because the short position can  be covered (the stock can be purchased and returned to the lender to cancel the short position) at cost $K$ at date $T$ by exercising the option, and paying $K$ at date $T$ is better than paying it at date $t$, given that interest rates must be nonnegative.  To be more precise, note that exercise at date $t$ produces  $S(t)-K$ dollars at date $t$.  On the other hand, retaining the option, shorting the stock at date $t$, and covering the short either by exercising the option or buying the stock in the market (whichever is cheaper) produces $S(t)$ dollars at date $t$ and 
$$\max(0,S(T)-K) - S(T) = \max(-S(T), -K) = -\min(S(T),K) \geq -K$$
dollars at date $T$.  If $S(T)>K$, one has $-K$ dollars at date $T$, in which case retaining the option has been superior due to the time value of money.  Furthermore,  if $S(T)<K$, the strategy of retaining the option and shorting the stock produces $-S(T) > -K$ dollars at date $T$, so retaining the option is superior due both to flexibility (waiting until $T$ to decide whether to exercise turns out to be better than committing at date $t$) and because of the time value of money.^[Recall that we are assuming investors earn interest on the proceeds of short sales; otherwise, the $S(t)$ dollars earned from exercising the option and selling the stock will be worth more than the $S(t)$ dollars earned from shorting the stock.  In this case, early exercise could be optimal.  However, assuming institutional investors can earn interest on the proceeds of shorts,  such investors should prefer owning the option and shorting the stock to exercising.  This means  they should bid up the price of the option to the point where it exceeds the value $S(t)-K$ of exercise.  If this is the case, then an investor who cannot earn interest on the proceeds of shorts should simply sell the option in the market rather than exercise it.  Thus, a sufficient condition for calls to be better alive than dead is that there be some investors who can earn interest on the proceeds of shorts.  This type of reasoning is possible for each situation in this book where the assumption of earning interest on margin deposits is important, and we will not deal with it in this much detail again.]

Early exercise of a call option can be optimal when the underlying asset pays a dividend.  The above analysis does not apply in this case, because paying the dividend to the lender of the stock is an additional cost for the strategy of retaining the option and shorting the stock.  If the dividend is so small that it cannot offset the time value of money on the exercise price, then early exercise will not be optimal.  In other cases, deriving the optimal exercise strategy is a complicated problem that we will first begin to study in @sec-c_introcomputation.

A European put option will be exercised at its maturity $T$ if the price $S(T)$ of the underlying asset is below the exercise price $K$.  In general, the value at maturity can be expressed as $\max(0,K-S(T))$.  Early exercise of an American put can be optimal, regardless of whether the underlying pays a dividend.  While it is valuable to keep one's options open (for puts as well as calls) the time value of money works in the opposite direction for puts.  Early exercise of a put option implies early receipt of the exercise price, and it is better to receive cash earlier rather than later. In general, whether early exercise is optimal depends on how deeply the option is in the money---if the underlying asset price is sufficiently low, then it will be fairly certain that exercise will be optimal, whether earlier or late; in this case, one should exercise earlier to earn interest on the exercise price.  How low it should be to justify early exercise depends on the interest rate (a higher rate makes the time-value-of-money issue more important, leading to earlier exercise) and the volatility of the underlying asset price (a lower volatility reduces the value of keeping one's options open, leading also to earlier exercise).  We will begin to study the valuation of American puts in @sec-c_introcomputation also.

### Compounding Interest

During most of the first two parts of the book (the only exception being @sec-c_forwardexchange) we will assume there is a risk-free asset earning a constant rate of return.  \index{risk-free asset} For simplicity, we will specify the rate of return as a continuously compounded rate.  \index{continuously compounded interest} For example, if the annual rate with annual compounding is $r_a$, then the corresponding continuously compounded rate is $r$ defined as $r = \log (1+r_a)$, where $\log$ denotes the natural logarithm function.  This means that the gross return over a year (one plus the rate of return) is $\mathrm{e}^r = 1+r_a$.  More generally, an investment of $x$ dollars for a time period of length $T$ (we measure time in years, so, e.g., a six-month investment would mean $T=0.5$) will result in the ownership of $x\mathrm{e}^{rT}$ dollars at the end of the time period.  

Expressing the interest rate as a continuously compounded rate enables us to avoid having to specify in each instance whether the rate is for annual compounding, semi-annual compounding, monthly compounding, etc.  For example, the meaning of an annualized rate $r_s$ for semi-annual compounding is that an investment of $x$ dollars will grow over a year to $x(1+r_s/2)^2$.  The equivalent continuously compounded rate is defined as $r = \log (1+r_s/2)^2$, and in terms of this rate we can say that the investment will grow in six months to $x\mathrm{e}^{0.5 r}$ and that it will grow in one year to $x\mathrm{e}^r$.  We can interpret this rate as being continuously compounded because compounding $n$ times per year at an annualized rate of $r$ results in \$1 growing in a year to $(1+r/n)^n$ and
$$\lim_{n \rightarrow \infty} \left(1+\frac{r}{n}\right)^n = \mathrm{e}^r\;\;.$$
To develop pricing and hedging formulas for derivative securities, it is a great convenience to assume that investors can trade continuously in time.  This requires us to assume also that returns are computed continuously.  In the case of a risk-free investment of $x(t)$ dollars at any date $t$ at a continuously compounded rate of $r$, we will say that the interest earned in an instant $\mathrm{d} t$ is $x(t)r\,\mathrm{d} t$ dollars.  This is only meaningful when we accumulate the interest over a non-infinitesimal period of time.  So consider investing $x(0)$ dollars at time 0 and reinvesting interest in the risk-free asset over a time period of length $T$.  Let $x(t)$ denote the account balance at date $t$, for $0\leq t \leq T$.  The change in the account balance in each instant is the interest earned, so we have
$\mathrm{d} x(t) = x(t)r\,\mathrm{d} t$.  The real meaning of this equation is that $x(t)$ satisfies the differential equation
$$\frac{\mathrm{d} x(t)}{\mathrm{d} t} = x(t)r\; ,$$
and it is well known (and easy to verify) that the solution is
$$x(t) = x(0)\mathrm{e}^{rt}\; ,$$
leading to an account balance at the end of the time period of $x(T) = x(0)\mathrm{e}^{rT}$.  Thus, the statement that the interest earned in an instant $\mathrm{d} t$ is $x(t)r\,\mathrm{d} t$ is equivalent to the statement that interest is continuously compounded at the rate $r$.

In the last part of the book, we will drop the assumption that the risk-free asset earns a constant rate of return.  In this case, we will still generally assume that there is a risk-free asset for very short-term investments (i.e., for investments with infinitesimal durations!).  We will let $r(t)$ denote the risk-free rate for an instantaneous investment at date $t$. This means that an investment of $x(t)$ dollars at date $t$ in the risk-free asset earns interest in an instant $\mathrm{d} t$ equal to $x(t)r(t)\,\mathrm{d} t$.  Consider again an investment of $x(0)$ dollars at date 0 in this instantaneously risk-free asset with interest reinvested and let $x(t)$ denote the account balance at date $t$.  Then $x(t)$ must satisfy the differential equation
$$\frac{\mathrm{d} x(t)}{\mathrm{d} t} = x(t)r(t)\;\;.$$
The solution of this differential equation is
$$x(t) = x(0)\exp\left(\int_0^t r(s)\,ds\right)\;\;.$$
The expression $\int_0^t r(s)\,ds$ can be interpreted as a continuous sum over time of the rates of interest $r(s)$ earned at times $s$ between 0 and $t$.  If these rates are all the same, say equal to $r$, then $\int_0^t r(s)\,ds = rt$ and our compounding factor $\exp\left(\int_0^t r(s)\,ds\right)$ is $\mathrm{e}^{rt}$ as before.  
