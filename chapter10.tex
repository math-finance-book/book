% Options for packages loaded elsewhere
\PassOptionsToPackage{unicode}{hyperref}
\PassOptionsToPackage{hyphens}{url}
\PassOptionsToPackage{dvipsnames,svgnames,x11names}{xcolor}
%
\documentclass[
  letterpaper,
  DIV=11,
  numbers=noendperiod]{scrartcl}

\usepackage{amsmath,amssymb}
\usepackage{iftex}
\ifPDFTeX
  \usepackage[T1]{fontenc}
  \usepackage[utf8]{inputenc}
  \usepackage{textcomp} % provide euro and other symbols
\else % if luatex or xetex
  \usepackage{unicode-math}
  \defaultfontfeatures{Scale=MatchLowercase}
  \defaultfontfeatures[\rmfamily]{Ligatures=TeX,Scale=1}
\fi
\usepackage{lmodern}
\ifPDFTeX\else  
    % xetex/luatex font selection
\fi
% Use upquote if available, for straight quotes in verbatim environments
\IfFileExists{upquote.sty}{\usepackage{upquote}}{}
\IfFileExists{microtype.sty}{% use microtype if available
  \usepackage[]{microtype}
  \UseMicrotypeSet[protrusion]{basicmath} % disable protrusion for tt fonts
}{}
\makeatletter
\@ifundefined{KOMAClassName}{% if non-KOMA class
  \IfFileExists{parskip.sty}{%
    \usepackage{parskip}
  }{% else
    \setlength{\parindent}{0pt}
    \setlength{\parskip}{6pt plus 2pt minus 1pt}}
}{% if KOMA class
  \KOMAoptions{parskip=half}}
\makeatother
\usepackage{xcolor}
\setlength{\emergencystretch}{3em} % prevent overfull lines
\setcounter{secnumdepth}{-\maxdimen} % remove section numbering
% Make \paragraph and \subparagraph free-standing
\makeatletter
\ifx\paragraph\undefined\else
  \let\oldparagraph\paragraph
  \renewcommand{\paragraph}{
    \@ifstar
      \xxxParagraphStar
      \xxxParagraphNoStar
  }
  \newcommand{\xxxParagraphStar}[1]{\oldparagraph*{#1}\mbox{}}
  \newcommand{\xxxParagraphNoStar}[1]{\oldparagraph{#1}\mbox{}}
\fi
\ifx\subparagraph\undefined\else
  \let\oldsubparagraph\subparagraph
  \renewcommand{\subparagraph}{
    \@ifstar
      \xxxSubParagraphStar
      \xxxSubParagraphNoStar
  }
  \newcommand{\xxxSubParagraphStar}[1]{\oldsubparagraph*{#1}\mbox{}}
  \newcommand{\xxxSubParagraphNoStar}[1]{\oldsubparagraph{#1}\mbox{}}
\fi
\makeatother

\usepackage{color}
\usepackage{fancyvrb}
\newcommand{\VerbBar}{|}
\newcommand{\VERB}{\Verb[commandchars=\\\{\}]}
\DefineVerbatimEnvironment{Highlighting}{Verbatim}{commandchars=\\\{\}}
% Add ',fontsize=\small' for more characters per line
\usepackage{framed}
\definecolor{shadecolor}{RGB}{241,243,245}
\newenvironment{Shaded}{\begin{snugshade}}{\end{snugshade}}
\newcommand{\AlertTok}[1]{\textcolor[rgb]{0.68,0.00,0.00}{#1}}
\newcommand{\AnnotationTok}[1]{\textcolor[rgb]{0.37,0.37,0.37}{#1}}
\newcommand{\AttributeTok}[1]{\textcolor[rgb]{0.40,0.45,0.13}{#1}}
\newcommand{\BaseNTok}[1]{\textcolor[rgb]{0.68,0.00,0.00}{#1}}
\newcommand{\BuiltInTok}[1]{\textcolor[rgb]{0.00,0.23,0.31}{#1}}
\newcommand{\CharTok}[1]{\textcolor[rgb]{0.13,0.47,0.30}{#1}}
\newcommand{\CommentTok}[1]{\textcolor[rgb]{0.37,0.37,0.37}{#1}}
\newcommand{\CommentVarTok}[1]{\textcolor[rgb]{0.37,0.37,0.37}{\textit{#1}}}
\newcommand{\ConstantTok}[1]{\textcolor[rgb]{0.56,0.35,0.01}{#1}}
\newcommand{\ControlFlowTok}[1]{\textcolor[rgb]{0.00,0.23,0.31}{\textbf{#1}}}
\newcommand{\DataTypeTok}[1]{\textcolor[rgb]{0.68,0.00,0.00}{#1}}
\newcommand{\DecValTok}[1]{\textcolor[rgb]{0.68,0.00,0.00}{#1}}
\newcommand{\DocumentationTok}[1]{\textcolor[rgb]{0.37,0.37,0.37}{\textit{#1}}}
\newcommand{\ErrorTok}[1]{\textcolor[rgb]{0.68,0.00,0.00}{#1}}
\newcommand{\ExtensionTok}[1]{\textcolor[rgb]{0.00,0.23,0.31}{#1}}
\newcommand{\FloatTok}[1]{\textcolor[rgb]{0.68,0.00,0.00}{#1}}
\newcommand{\FunctionTok}[1]{\textcolor[rgb]{0.28,0.35,0.67}{#1}}
\newcommand{\ImportTok}[1]{\textcolor[rgb]{0.00,0.46,0.62}{#1}}
\newcommand{\InformationTok}[1]{\textcolor[rgb]{0.37,0.37,0.37}{#1}}
\newcommand{\KeywordTok}[1]{\textcolor[rgb]{0.00,0.23,0.31}{\textbf{#1}}}
\newcommand{\NormalTok}[1]{\textcolor[rgb]{0.00,0.23,0.31}{#1}}
\newcommand{\OperatorTok}[1]{\textcolor[rgb]{0.37,0.37,0.37}{#1}}
\newcommand{\OtherTok}[1]{\textcolor[rgb]{0.00,0.23,0.31}{#1}}
\newcommand{\PreprocessorTok}[1]{\textcolor[rgb]{0.68,0.00,0.00}{#1}}
\newcommand{\RegionMarkerTok}[1]{\textcolor[rgb]{0.00,0.23,0.31}{#1}}
\newcommand{\SpecialCharTok}[1]{\textcolor[rgb]{0.37,0.37,0.37}{#1}}
\newcommand{\SpecialStringTok}[1]{\textcolor[rgb]{0.13,0.47,0.30}{#1}}
\newcommand{\StringTok}[1]{\textcolor[rgb]{0.13,0.47,0.30}{#1}}
\newcommand{\VariableTok}[1]{\textcolor[rgb]{0.07,0.07,0.07}{#1}}
\newcommand{\VerbatimStringTok}[1]{\textcolor[rgb]{0.13,0.47,0.30}{#1}}
\newcommand{\WarningTok}[1]{\textcolor[rgb]{0.37,0.37,0.37}{\textit{#1}}}

\providecommand{\tightlist}{%
  \setlength{\itemsep}{0pt}\setlength{\parskip}{0pt}}\usepackage{longtable,booktabs,array}
\usepackage{calc} % for calculating minipage widths
% Correct order of tables after \paragraph or \subparagraph
\usepackage{etoolbox}
\makeatletter
\patchcmd\longtable{\par}{\if@noskipsec\mbox{}\fi\par}{}{}
\makeatother
% Allow footnotes in longtable head/foot
\IfFileExists{footnotehyper.sty}{\usepackage{footnotehyper}}{\usepackage{footnote}}
\makesavenoteenv{longtable}
\usepackage{graphicx}
\makeatletter
\def\maxwidth{\ifdim\Gin@nat@width>\linewidth\linewidth\else\Gin@nat@width\fi}
\def\maxheight{\ifdim\Gin@nat@height>\textheight\textheight\else\Gin@nat@height\fi}
\makeatother
% Scale images if necessary, so that they will not overflow the page
% margins by default, and it is still possible to overwrite the defaults
% using explicit options in \includegraphics[width, height, ...]{}
\setkeys{Gin}{width=\maxwidth,height=\maxheight,keepaspectratio}
% Set default figure placement to htbp
\makeatletter
\def\fps@figure{htbp}
\makeatother

\KOMAoption{captions}{tableheading}
\makeatletter
\@ifpackageloaded{caption}{}{\usepackage{caption}}
\AtBeginDocument{%
\ifdefined\contentsname
  \renewcommand*\contentsname{Table of contents}
\else
  \newcommand\contentsname{Table of contents}
\fi
\ifdefined\listfigurename
  \renewcommand*\listfigurename{List of Figures}
\else
  \newcommand\listfigurename{List of Figures}
\fi
\ifdefined\listtablename
  \renewcommand*\listtablename{List of Tables}
\else
  \newcommand\listtablename{List of Tables}
\fi
\ifdefined\figurename
  \renewcommand*\figurename{Figure}
\else
  \newcommand\figurename{Figure}
\fi
\ifdefined\tablename
  \renewcommand*\tablename{Table}
\else
  \newcommand\tablename{Table}
\fi
}
\@ifpackageloaded{float}{}{\usepackage{float}}
\floatstyle{ruled}
\@ifundefined{c@chapter}{\newfloat{codelisting}{h}{lop}}{\newfloat{codelisting}{h}{lop}[chapter]}
\floatname{codelisting}{Listing}
\newcommand*\listoflistings{\listof{codelisting}{List of Listings}}
\usepackage{amsthm}
\theoremstyle{definition}
\newtheorem{exercise}{Exercise}[section]
\theoremstyle{remark}
\AtBeginDocument{\renewcommand*{\proofname}{Proof}}
\newtheorem*{remark}{Remark}
\newtheorem*{solution}{Solution}
\newtheorem{refremark}{Remark}[section]
\newtheorem{refsolution}{Solution}[section]
\makeatother
\makeatletter
\makeatother
\makeatletter
\@ifpackageloaded{caption}{}{\usepackage{caption}}
\@ifpackageloaded{subcaption}{}{\usepackage{subcaption}}
\makeatother

\ifLuaTeX
  \usepackage{selnolig}  % disable illegal ligatures
\fi
\usepackage{bookmark}

\IfFileExists{xurl.sty}{\usepackage{xurl}}{} % add URL line breaks if available
\urlstyle{same} % disable monospaced font for URLs
\hypersetup{
  pdftitle={Finite Difference Methods},
  colorlinks=true,
  linkcolor={blue},
  filecolor={Maroon},
  citecolor={Blue},
  urlcolor={Blue},
  pdfcreator={LaTeX via pandoc}}


\title{Finite Difference Methods}
\author{}
\date{}

\begin{document}
\maketitle


In this chapter we will see how to estimate derivative values by
numerically solving the partial differential equation (pde) that the
derivative value satisfies, using finite difference methods. More
advanced discussions of this topic can be found in Wilmott, DeWynne and
Howison {[}@WDH{]}, Wilmott {[}@Wilmott{]}, and Tavella {[}@Tavella{]},
among other places. We will only consider derivatives written on a
single underlying asset, but the ideas generalize to derivatives written
on multiple underlying assets (e.g., basket and spread options) in much
the same way that binomial models can be applied to derivatives on
multiple underlying assets. The curse of dimensionality is the same for
finite difference methods as for binomial models---the computation time
increases exponentially with the number of underlying assets.

\subsection{Fundamental PDE}\label{sec-fundamentalpde}

\index{fundamental pde}Consider an asset with price \(S\) and constant
dividend yield \(q\).\\
Set \(X=\log S\). Then we have
\[\mathrm{d} X = \nu\,\mathrm{d} t+\sigma\,\mathrm{d} B\; ,\] where
\(\nu =r-q-\sigma^2/2\) and \(B\) is a Brownian motion under the
risk-neutral measure.

Let \(T\) denote the maturity date of a derivative security. At time
\(t\) (when the remaining time to maturity is \(T-t\)), assume the price
of the derivative can be represented as \(C(t,X(t))\).\footnote{If the
  price of the derivative is a function of the asset price \(S\) and
  time, then we can always write it in this form as a function of the
  natural logarithm of \(S\) and time.}\\
Since \(C\) is a function of \(t\) and \(X\), Ito's formula implies

\begin{equation}\phantomsection\label{eq-2}{
\mathrm{d} C  = \frac{\partial C}{\partial t}\,\mathrm{d} t + \frac{\partial C}{\partial X}\,\mathrm{d} X +\frac{1}{2}\frac{\partial^2 C}{\partial X^2}(\mathrm{d} X)^2
}\end{equation}

\[
=\frac{\partial C}{\partial t}\,\mathrm{d} t+ \frac{\partial C}{\partial X}\big(\nu\,\mathrm{d} t+\sigma\,\mathrm{d} B\big) + \frac{1}{2}\frac{\partial^2 C}{\partial X^2}\sigma^2\,\mathrm{d} t\;.
\]

On the other hand, under the risk-neutral measure, the instantaneous
expected rate of return on the derivative is the risk-free rate, so
\[\frac{\mathrm{d} C}{C} =r\,\mathrm{d} t + \text{something}\,\,\mathrm{d} B\; .\]
where the something is the volatility of the derivative value. We can of
course rearrange this as \begin{equation}\phantomsection\label{eq-1}{
\mathrm{d} C = rC\,\mathrm{d} t+\text{something}\,\,C\,\mathrm{d} B\;.
}\end{equation}

In order for both Equation~\ref{eq-2} and Equation~\ref{eq-1} to hold,
the drifts on both right-hand sides must be equal.\footnote{Suppose a
  process \(X\) satisfies
  \(\mathrm{d} X=\alpha_1\,\mathrm{d} t+\sigma_1\,\mathrm{d} B = \alpha_2\,\mathrm{d} t+\sigma_2\,\mathrm{d} B\)
  for coefficients \(\alpha_i\) and \(\sigma_i\). This implies
  \((\alpha_1-\alpha_2)\,\mathrm{d} t=(\sigma_2-\sigma_2)\,\mathrm{d} B\).
  The right-hand side defines a (local) martingale and the left-hand
  side defines a continuous finite-variation process. As discussed in
  \textbf{?@sec-s\_quadraticvariation}, the only continuous
  finite-variation martingales are constants, so the changes must be
  zero; i.e., \(\alpha_1=\alpha_2\) and \(\sigma_1=\sigma_2\).} This
implies \begin{equation}\phantomsection\label{eq-3}{
rC = \frac{\partial C}{\partial t}+ \nu\frac{\partial C}{\partial X}+ \frac{1}{2}\sigma^2\frac{\partial^2 C}{\partial X^2}\;.
}\end{equation}

This equation is the fundamental pde. It is an equation that we want to
solve for the function \(C\). Every derivative written on \(S\)
satisfies this same equation. Different derivatives have different
values because of boundary conditions. The boundary conditions are the
intrinsic value at maturity, optimality conditions for early exercise,
barriers and the like.

To translate the terms in Equation~\ref{eq-3} into more familiar ones,
notice that, because \(S=\mathrm{e}^X\), we have \[
\frac{\partial S}{\partial X}=\mathrm{e}^X=S\;.
\] Therefore, by the chain rule of calculus, \[
\frac{\partial C}{\partial X} = \frac{\partial C}{\partial S}\frac{\partial S}{\partial X} = S\frac{\partial C}{\partial S}\;.\]
Thus the term \(\partial C/\partial X\) is the delta of the derivative
multiplied by the price of the underlying. Similarly, by ordinary
calculus, the term \(\partial^2 C/\partial X^2\) can be written in terms
of the delta and the gamma of the derivative.

Sometimes one writes the derivative value as a function of time to
maturity (\(\tau = T-t\)) instead of \(t\). The partial derivative of
\(C\) with respect to \(\tau\) is the negative of the partial derivative
with respect to \(t\), so the fundamental pde is the same except for a
different sign on the first term of the right-hand side of
Equation~\ref{eq-3}. Rearranging a little, we have
\begin{equation}\phantomsection\label{eq-4}{
\frac{\partial C}{\partial \tau} = -rC + \nu\frac{\partial C}{\partial X}+ \frac{1}{2}\sigma^2\frac{\partial^2 C}{\partial X^2}\;.
}\end{equation}

In this form, the pde is similar to important equations in physics, in
particular the equation for how heat propagates through a rod over time.
In fact, it can be transformed exactly into the heat equation, which is
how Black and Scholes originally solved the option valuation problem.
The terminal condition for a call option, \(C = \max(S-K,0)\), can be
viewed as defining \(C\) over the \(X\) dimension at \(\tau=0\), just as
the temperature along the length of the rod might be specified at an
initial date, and as \(\tau\) increases \(C\) changes at each point
\(X\) according to Equation~\ref{eq-4}, which is similar, as noted, to
the equation for the change in temperature at a point on the rod as time
passes.

\subsection{Discretizing the PDE}\label{discretizing-the-pde}

To numerically solve the fundamental pde, we consider a discrete grid on
the \((t,x)\) space. We label the time points as
\(t_0, t_1, t_2, \ldots, t_N\), and the \(x\) points as
\(x_{-M}, x_{-M+1}, \ldots, x_0, x_1, \ldots, x_M\), with \(t_0=0\),
\(t_N=T\), and \(x_0=\log S(0)\). The equation should hold for
\(-\infty< x < \infty\), but obviously we will have to bound this space,
and we have denoted the upper and lower bounds by \(x_M\) and \(x_{-M}\)
here. We take the points to be evenly spaced and set
\(\Delta t= t_i-t_{i-1}\) and \(\Delta x = x_j -x_{j-1}\) for any \(i\)
and \(j\).

For specificity, we will consider a call option, though the discussion
in this section applies to any derivative. We will compute a value for
the call at each of the points on the grid. Then we return the value of
the call at the point \((t_0,x_0)\).

Consider a point \((t_i, x_j)\). We could denote the estimated value of
the call at this point by \(C_{ij}\) but for now we will just use the
symbol \(C\). Think of \(t\) being on the horizontal axis and \(x\) on
the vertical axis. There are four points that can be reached from
\((t_i,x_j)\) by one step (an increase or decrease) in either \(t\) or
\(x\). Let's denote the estimated call value at \((t_i, x_j+\Delta x)\)
as \(C_{\text{up}}\), the value at \((t_i,x_j-\Delta x)\) as
\(C_{\text{down}}\), the value at \((t_i+\Delta t, x_j)\) as
\(C_{\text{right}}\) and the value at \((t_i-\Delta t,x_j)\) as
\(C_{\text{left}}\).

We want to force Equation~\ref{eq-3} to hold on the grid. To estimate
\(\partial C/\partial X\) and \(\partial^2 C/\partial X^2\), we make
exactly the same calculations we made to estimate deltas and gammas in a
binomial model. At the point \((t_i,x_j)\), we estimate

\begin{equation}\phantomsection\label{eq-pdedelta}{
\frac{\partial C}{\partial X} \approx \frac{C_{\text{up}}-C_{\text{down}}}{2\Delta x}\; .
}\end{equation}

There are two other obvious estimates of this derivative: \[
\frac{C_{\text{up}}-C}{\Delta x} \qquad \text{and} \qquad \frac{C-C_{\text{down}}}{\Delta x}\;.
\] The first of these should be understood as an estimate at the
midpoint of \(x_j\) and \(x_j+\Delta x\) and the second as an estimate
at the midpoint of \(x_j\) and \(x_j-\Delta x\). The distance between
these two midpoints is \(\Delta x\), so the difference in these two
estimates of \(\partial C/\partial X\) divided by \(\Delta x\) is an
estimate of the second derivative:
\begin{equation}\phantomsection\label{eq-pdegamma}{
\frac{\partial^2 C}{\partial X^2} \approx \frac{C_{\text{up}}-2C+C_{\text{down}}}{(\Delta x)^2}\;.
}\end{equation}

The obvious estimate of \(\partial C/\partial t\), which is analogous to
the estimate of \(\partial C/\partial X\), is \[
\frac{C_{\text{right}}-C_{\text{left}}}{2\Delta t}\;.
\] This is \emph{not} the estimate we are going to use. The reason is
that we want to solve for the call values on the grid in much the same
way that we solved the binomial model---starting at the end and working
backwards. If we use the above estimate of the time derivative, then at
each point \((t_i,x_j)\), Equation~\ref{eq-3} will link the call values
at times \(t_{i-1}\), \(t_i\) and \(t_{i+1}\). This would substantially
complicate the backing up process. However, in a sense, it is the right
estimate, and the Crank-Nicolson method to be discussed below uses a
similar idea.

The other two choices for estimating \(\partial C/\partial t\) are
analogous to the other two choices for estimating
\(\partial C/\partial X\). We can use either

\begin{equation}\phantomsection\label{eq-explicit_dt}{
\frac{\partial C}{\partial t} \approx \frac{C-C_{\text{left}}}{\Delta t}\;,
}\end{equation}

\[
\] or \begin{equation}\phantomsection\label{eq-implicit_dt}{
\frac{\partial C}{\partial t} \approx \frac{C_{\text{right}}-C}{\Delta t}\;. 
}\end{equation}

Using the first is called the explicit method of solving the pde, and
using the second is called the implicit method. The reason for these
names should become clear below.

\subsection{Explicit and Implicit
Methods}\label{explicit-and-implicit-methods}

We first consider the explicit method. \index{explicit method}We set the
value of the call at the final date \(t_N\) and each point \(x_j\) to be
its intrinsic value, \(\max\left(\mathrm{e}^{x_j}-K,0\right)\). Now
consider calculating the value at date \(t_{N-1}\) and any point
\(x_j\). We do this by forcing the approximation to Equation~\ref{eq-3}
based on Equation~\ref{eq-pdedelta}--Equation~\ref{eq-explicit_dt} to
hold at the point \((t_N,x_j)\). Using the same notation as before, for
\((t_i,x_j)=(t_N,x_j)\), implies
\begin{equation}\phantomsection\label{eq-explicit}{
rC = \frac{C-C_{\text{left}}}{\Delta t}+ \nu\left(\frac{C_{\text{up}}-C_{\text{down}}}{2\Delta x}\right)+ \frac{1}{2}\sigma^2\left(\frac{C_{\text{up}}+C_{\text{down}}-2C}{(\Delta x)^2}\right)\;.
}\end{equation}

Given that \(t_i\) is the final date \(t_N\), the values \(C\),
\(C_{\text{up}}\) and \(C_{\text{down}}\) have already been calculated
as the intrinsic value of the call at maturity. The only unknown is
\(C_{\text{left}}\), which is the value of the call at
\((t_{N-1},x_j)\). We can solve this \emph{explicitly} for
\(C_{\text{left}}\), whence the name of the algorithm. We do this at
each point \(x_j\) at date \(t_{N-1}\) (except for the top and bottom
points, which we will discuss below) and then we follow the same
procedure to back up sequentially to the initial date, as in the
binomial model.

Equation Equation~\ref{eq-explicit} cannot be used to find
\(C_{\text{left}}\) at the bottom point \(x_{-M}\), because at this
point there is no \(C_{\text{down}}\) at date \(t_N\). Similarly, we
cannot use it to find \(C_{\text{left}}\) at the top point \(x_M\),
because at that point there is no \(C_{\text{up}}\). We have to define
the values along the top and bottom of the grid in some other fashion.
We do this using conditions the derivative is known to satisfy as the
stock price approaches \(+\infty\) or 0. For example, for a European
call option, we use the conditions that
\(\partial C/\partial S \rightarrow 1\) as \(S \rightarrow \infty\) and
\(\partial C/\partial S \rightarrow 0\) as \(S \rightarrow 0\). We will
explain this in more detail in the following section.

The solution of Equation~\ref{eq-explicit} for \(C_{\text{left}}\) can
be written as \begin{equation}\phantomsection\label{eq-fdtrinomial}{
C_{\text{left}} = \big(1-r\Delta t\big)\big(p_uC_{\text{up}}+pC + p_dC_{\text{down}}\big)\;,
}\end{equation}

where \begin{align*}
p_u &= \frac{\sigma^2\Delta t+\nu\Delta t\Delta x}{2(1-r\Delta t)(\Delta x)^2}\; ,\\
p_d &= \frac{\sigma^2\Delta t-\nu\Delta t\Delta x}{2(1-r\Delta t)(\Delta x)^2}\; ,\\
p &= 1- p_u-p_d\;.
\end{align*} This can be interpreted as discounting the
probability-weighted values of the call at the next date, where we
consider that starting at the grid point \((t_{i},x_j)\), the logarithm
of the stock price takes three possible values (\(x_j-\Delta x\),
\(x_j\), and \(x_j+\Delta x\)) at the next date \(t_{i+1}\), and where
we use \(1-r\Delta t\) as the discount factor. Thus, it is essentially a
trinomial model. \index{trinomial model}This relationship was first
noted by Brennan and Schwartz {[}@BrennanSchwartz{]}.

Actually, for this to be a sensible trinomial model, the probabilities
\(p_u\), \(p\) and \(p_d\) should be nonnegative. Assuming
\(1-r\Delta t>0\), this will be the case if and only if \[
\Delta x \leq  \frac{\sigma^2}{|\nu|} \qquad \text{and} \qquad \Delta t \leq \frac{(\Delta x)^2}{\sigma^2 + r(\Delta x)^2}\;.
\] The first of these conditions characterizes \(p_u\) and \(p_d\) being
nonnegative. The second is derived from \(p_u+p_d \leq 1\). It is
interesting to examine these conditions in terms of the number \(N\) of
time periods and the number of steps in the \(x\) dimension, which is
\(2M\). To simplify the notation in the following somewhat, denote the
distance of the upper \(x\) boundary from \(x_0\) by \(D\) (i.e.,
\(D=x_M-x_0\)). Then \(\Delta t=T/N\) and \(\Delta x = D/M\). The
probabilities are nonnegative if and only if \[
M \geq \frac{|\nu| D}{\sigma^2} \qquad \text{and} \qquad N \geq rT + \left(\frac{\sigma^2T}{D^2}\right)M^2\;.
\] Consider fixing \(D\) and increasing the number of time periods and
space steps (i.e., steps along the \(x\) dimension). To maintain
positive probabilities, the above shows that the number of time periods
must increase as the square of the number of space steps: increasing
\(M\) by a factor of 10 requires increasing \(N\) by a factor of 100.
The upshot is it can be computationally expensive to use a large number
of space steps, if we want to maintain nonnegative probabilities.

One can reasonably ask whether this is important, because we can
certainly solve Equation~\ref{eq-explicit} to estimate the call values
even when the probabilities are negative. The answer is that it is
important, but for a reason we have not yet discussed. In a numerical
algorithm for solving a partial differential equation (or for solving
many other types of problems) there are two types of errors:
discretization error and roundoff error. If we increase \(N\) and \(M\)
sufficiently, we should reduce the discretization error. However, each
calculation on the computer introduces roundoff error. An algorithm is
said to be stable if the \index{stable algorithm} roundoff errors stay
small and bounded as the discretization error is reduced. An unfortunate
fact about the explicit method is that it is stable only if the number
of time steps increases with the square of the number of space steps. In
the absence of this condition, the roundoff errors can accumulate and
prevent one from reaching a solution of the desired accuracy.

The implicit method is known to be fully stable, so it is to be
preferred to the explicit method. We will discuss briefly how to
implement this method, before moving in the next section to the
Crank-Nicolson method, which is also fully stable and known to be more
efficient than the implicit method.

The implicit method \index{implicit method} uses the approximation
Equation~\ref{eq-implicit_dt} for \(\partial C/\partial t\). As before,
the call values are defined at the final date as the intrinsic value.
Backing up a period, consider a grid point \((t_{N-1},x_j)\). We will
try to estimate the call value at this date by forcing
Equation~\ref{eq-3} to hold at this point. This means
\begin{equation}\phantomsection\label{eq-implicit}{
rC = \frac{C_{\text{right}}-C}{\Delta t}+ \nu\left(\frac{C_{\text{up}}-C_{\text{down}}}{2\Delta x}\right)+ \frac{1}{2}\sigma^2\left(\frac{C_{\text{up}}+C_{\text{down}}-2C}{(\Delta x)^2}\right)\;.
}\end{equation}

We know \(C_{\text{right}}\), because it is the intrinsic value at
\((t_N,x_j)\). This equation links three unknowns (\(C\),
\(C_{\text{up}}\), and \(C_{\text{down}}\)) to the known value
\(C_{\text{right}}\). We cannot solve it explicitly for these three
unknowns. Instead, we need to solve a system of linear equations to
simultaneously solve for all the call values at date \(t_{N-1}\). There
are \(2M-1\) equations of the form Equation~\ref{eq-implicit} plus
conditions that we will impose at the upper and lower boundaries, and we
need to solve these for the \(2M+1\) call values. This system of
equations has the same form, and is solved in the same way, as the
system of equations in the Crank-Nicolson method.

\subsection{Crank-Nicolson}\label{crank-nicolson}

\index{Crank-Nicolson method} The estimate Equation~\ref{eq-implicit_dt}
of \(\partial C/\partial t\) used in the implicit method is best
understood as an estimate of \(\partial C/\partial t\) at the midpoint
of \((t_i,x_j)\) and \((t_{i+1},x_j)\), i.e., at
\((t_i+\Delta t/2,x_j)\). This is the basic idea of the Crank-Nicolson
method. With this method, we continue to estimate the call values at the
grid points, but we do so by forcing Equation~\ref{eq-3} to hold at
midpoints of this type. To do this, we also need estimates of \(C\),
\(\partial C/\partial X\) and \(\partial^2 C/\partial X^2\) at the
midpoints, but these are easy to obtain.

Let's modify the previous notation somewhat, writing \(C'\) for
\(C_{\text{right}}\) and \(C'_{\text{up}}\) and \(C'_{\text{down}}\) for
the values to the right and one step up and down, i.e., at the grid
points \((t_i+\Delta t,x_i+\Delta x)\) and
\((t_i+\Delta t,x_i-\Delta x)\) respectively. The obvious estimate of
the call value at the midpoint \((t_i+\Delta t/2,x_j)\) is the average
of \(C\) and \(C'\), so set \[
C^{\text{mid}} = \frac{C+C'}{2}\;.
\] Analogously, define \begin{equation}\phantomsection\label{eq-cn1000}{
C^{\text{mid}}_{\text{up}} = \frac{C_{\text{up}}+C'_{\text{up}}}{2}\;, \qquad \text{and} \qquad
C^{\text{mid}}_{\text{down}} = \frac{C_{\text{down}}+C'_{\text{down}}}{2}\;.
}\end{equation}

The formulas Equation~\ref{eq-cn1000} give us estimates of the call
value at the midpoints one space step up and one space step down from
\(x_j\)---i.e., at \((t_i+\Delta t/2,x_{j+1})\) and
\((t_i+\Delta t/2,x_{j-1})\). We can now estimate
\(\partial C/\partial X\) and \(\partial^2 C/\partial X^2\) at the
midpoint \((t_i+\Delta t/2,x_j)\) exactly as before: \[
\frac{\partial C}{\partial X} \approx \frac{C^{\text{mid}}_{\text{up}}-C^{\text{mid}}_{\text{down}}}{2\Delta x}\;,
\] and \[
\frac{\partial^2 C}{\partial X^2} \approx \frac{C^{\text{mid}}_{\text{up}}+C^{\text{mid}}_{\text{down}}-2C^{\text{mid}}}{(\Delta x)^2}\;.
\] Now, Equation~\ref{eq-3} becomes
\begin{equation}\phantomsection\label{eq-crank}{
rC^{\text{mid}} = \frac{C'-C}{\Delta t}+ \nu\left(\frac{C^{\text{mid}}_{\text{up}}-C^{\text{mid}}_{\text{down}}}{2\Delta x}\right)+ \frac{1}{2}\sigma^2\left(\frac{C^{\text{mid}}_{\text{up}}+C^{\text{mid}}_{\text{down}}-2C^{\text{mid}}}{(\Delta x)^2}\right)\;.
}\end{equation}

Substituting from the formulas for \(C^{\text{mid}}\),
\(C^{\text{mid}}_{\text{up}}\), and \(C^{\text{mid}}_{\text{down}}\), we
can re-write Equation~\ref{eq-crank} as

\begin{multline}
\left(\frac{r}{2}+\frac{1}{\Delta t}+\frac{\sigma^2}{2(\Delta x)^2}\right)C - \left(\frac{\sigma^2}{4(\Delta x)^2}+\frac{\nu}{4\Delta x}\right)C_{\text{up}}\\ - \left(\frac{\sigma^2}{4(\Delta x)^2}-\frac{\nu}{4\Delta x}\right)C_{\text{down}} 
\quad = \quad \left(\frac{1}{\Delta t}- \frac{r}{2}-\frac{\sigma^2}{2(\Delta x)^2}\right)C' \\+ \left(\frac{\sigma^2}{4(\Delta x)^2}+\frac{\nu}{4\Delta x}\right)C'_{\text{up}} + \left(\frac{\sigma^2}{4(\Delta x)^2}-\frac{\nu}{4\Delta x}\right)C'_{\text{down}}
\end{multline} \{\#eq-crank2\}

We can also write this as
\begin{equation}\phantomsection\label{eq-crank3}{
a_1C - a_2C_{\text{up}} - a_3C_{\text{down}} 
= a_4C' + a_2C'_{\text{up}} + a_3C'_{\text{down}}\;,}\end{equation}

where the constants \(a_i\) are the factors in parentheses in
\textbf{?@eq-crank2}.

As before, we start at the final date \(t_N\) and define the call value
at that date by its intrinsic value. Consider a grid point
\((t_{N-1},x_j)\). Forcing Equation~\ref{eq-3} to hold at the midpoint
\((t_{N-1}+\Delta t/2,x_j)\) leads us to Equation~\ref{eq-crank3}. In
this equation, \(C'\), \(C'_{\text{up}}\) and \(C'_{\text{down}}\) are
known from the intrinsic value at maturity, and we need to solve for
\(C\), \(C_{\text{up}}\) and \(C_{\text{down}}\). There are \(2M-1\)
linear equations of this type and we will add linear equations at the
upper and lower boundaries of the grid and solve the resulting system of
\(2M+1\) linear equations for the \(2M+1\) call values. After finding
the call values at date \(t_{N-1}\), we then repeat the calculation at
\(t_{N-2}\) and continue backing up in this way until we reach the
initial date.

Notice that the Crank-Nicolson equations \textbf{?@eq-crank2} are
similar to the equations Equation~\ref{eq-implicit} in the implicit
method, but more information is used in each step of the Crank-Nicolson
method than is used in each step of the implicit method. Equation
\textbf{?@eq-crank2} links the call values \(C\), \(C_{\text{up}}\) and
\(C_{\text{down}}\) to the previously calculated \(C'\),
\(C'_{\text{up}}\) and \(C'_{\text{down}}\), whereas in the implicit
method they were linked only to \(C'\) (which we called
\(C_{\text{right}}\)).

\subsection{European Options}\label{european-options}

To value a European option, one simply defines the values at the final
date as the intrinsic value and then backs up to the initial date, using
any of the methods described (explicit, implicit, or Crank-Nicolson).
The value that should be returned is the value at the middle node at the
initial date, which corresponds to the initial price of the underlying.

The boundary conditions normally used at the bottom and top of the grid
are conditions that the first derivative \(\partial C/\partial S\) of
the option value are known to satisfy as \(S \rightarrow 0\) and
\(S \rightarrow \infty\). These are conditions of the form
\begin{equation}\phantomsection\label{eq-cn_boundary}{
\lim_{S \rightarrow \infty} \frac{\partial C}{\partial S} =  \lambda_0\;,\qquad \text{and} \qquad
\lim_{S \rightarrow 0} \frac{\partial C}{\partial S} = \lambda_{\infty}\;,
}\end{equation}

for constants \(\lambda_0\) and \(\lambda_{\infty}\). In the case of a
call option, we have \(\lambda_0=0\) and \(\lambda_{\infty}=1\). For a
put option, we have \(\lambda_0=-1\) and \(\lambda_\infty = 0.\)

These conditions are implemented on the grid by forcing each value \(C\)
at a point \((t_i,x_{-M})\) on the bottom of the grid to satisfy

\begin{equation}\phantomsection\label{eq-crank4a}{
C - C_{\text{up}} = \lambda_0 (S - S_{\text{up}})
}\end{equation}

and by forcing each value \(C\) at a point \((t_i,x_{M})\) on the top of
the grid to satisfy

\begin{equation}\phantomsection\label{eq-crank4b}{
C-C_{\text{down}} = \lambda_\infty(S-S_{\text{down}})\;.
}\end{equation}

These two linear equations in the values at time \(t_i\) augment the
\(2M-1\) equations already described to form a system of \(2M+1\) linear
equations to be solved for the derivative values at the \(2M+1\) grid
points at time \(t_i\).

We will create a program that solves a system of equations of the form
Equation~\ref{eq-crank3}, Equation~\ref{eq-crank4a} and
Equation~\ref{eq-crank4b}. We input the vector \(a\) of coefficients, a
vector \(y\) of dimension \(2M+1\) containing the estimated values of
the derivative at any date \(t_{i+1}\), an integer \(L\) from which
\(M\) is defined as \(M=(L-1)/2\) (i.e., \(L=2M+1\)). The function will
return the vector of values at date \(t_i\).

We will write the boundary conditions Equation~\ref{eq-crank4a} and
Equation~\ref{eq-crank4b}, respectively, in the more general forms

\begin{equation}\phantomsection\label{eq-crank4aa}{
C = z_1 + b_1C_{\text{up}}\;,
}\end{equation}

and \begin{equation}\phantomsection\label{eq-crank4bb}{
C = z_L + b_LC_{\text{down}}\;,
}\end{equation}

where \(z_1\), \(b_1\), \(z_L\) and \(b_L\) are numbers to be calculated
or input by the user. The equations Equation~\ref{eq-crank4a} and
Equation~\ref{eq-crank4b} are the special cases in which
\(z_1 = \lambda_0(S-S_{\text{up}})\), \(b_1 = 1\),
\(z_L = \lambda_\infty(S-S_{\text{down}})\), and \(b_L=1\). The
additional generality in allowing \(b_1\) and \(b_L\) to be different
from one is important for many purposes, and we will see an example of
it in the valuation of barrier options.

The system of equations that we want to solve is therefore
\setcounter{MaxMatrixCols}{11}

\begin{pmatrix} 
1 & -b_1 & 0 & 0 & 0  & \cdots  & 0 & 0 & 0 & 0 & 0\\
-a_3 & a_1 & - a_2 & 0 & 0 & \cdots & 0 & 0 & 0 & 0 & 0\\
0 & - a_3 & a_1 & -a_2 & 0 & \cdots & 0 & 0 & 0 & 0 & 0\\
\vdots & \vdots & \vdots & \vdots & \vdots & \vdots &\vdots & \vdots & \vdots & \vdots & \vdots \\
0 & 0 & 0 & 0 & 0 & \cdots & 0 & -a_3 & a_1 & - a_2 & 0\\
0 & 0 & 0 & 0 & 0 & \cdots  & 0 & 0 & -a_3 & a_1 & - a_2 \\
0 & 0 & 0 & 0 & 0 & \cdots  & 0 & 0 & 0 & -b_L & 1 
\end{pmatrix}

\begin{pmatrix}
C_1 \\
C_2\\
C_3\\
\vdots\\
C_{L-2}\\
C_{L-1}\\
C_L
\end{pmatrix}
\begin{matrix}
\phantom\\
\phantom\\
\phantom\\
= 
\phantom\\
\phantom\\
\phantom\\
\end{matrix}
\begin{pmatrix}
z_1 \\
z_2\\
z_3\\
\vdots\\
z_{L-2}\\
z_{L-1}\\
z_L
\end{pmatrix}

where we are denoting the derivative values to be determined at date
\(t_i\) across the \(L\) (\(=2M+1\)) space nodes as
\(C_1, \ldots, C_L\). The coefficients \(a_i\) are defined in
Equation~\ref{eq-crank3}. The numbers \(z_2, \ldots, z_{L-1}\) are the
right-hand sides of Equation~\ref{eq-crank3} and are determined by the
coefficients \(a_i\) and the derivative values \(y_1, \ldots, y_{L}\) at
date \(t_{i+1}\). The system of equations that must be solved to
implement the implicit method is of this same form.

The first equation in this array (Equation~\ref{eq-crank4aa}) can be
written as

\[
C_1 = u_1 + b_1C_2\;,
\] where \(u_1 = z_1\).\\
By induction, we will see that we can write, for each
\(j = 2, \ldots, L\),
\begin{equation}\phantomsection\label{eq-crank101}{
C_{j-1} = u_{j-1} + b_{j-1}C_{j}
}\end{equation}

for some coefficients \(u_{j-1}\) and \(b_{j-1}\) to be determined.\\
The \(j\)--th equation (\(j = 2, \ldots, L-1\)) in the array
(Equation~\ref{eq-crank3}) is
\[-a_3C_{j-1}+a_1C_j-a_2C_{j+1} = z_j\; .\] Supposing
Equation~\ref{eq-crank101} holds and using it to substitute for
\(C_{j-1}\), we have
\[-a_3\left(u_{j-1} + b_{j-1}C_{j}\right) +a_1C_j-a_2C_{j+1} = z_j \quad\Longleftrightarrow\quad C_{j} = u_{j} + b_{j}C_{j+1}\; ,\]
where \begin{align*}
u_j &= \frac{z_j + a_3u_{j-1}}{a_1-a_3b_{j-1}} \;,\\
b_j &= \frac{a_2}{a_1-a_3b_{j-1}}\;.
\end{align*} This establishes that Equation~\ref{eq-crank101} holds for
each \(j = 2, \ldots, L\).

The last equation in the array (Equation~\ref{eq-crank4bb}) is
\begin{equation}\phantomsection\label{eq-crankC_L}{
C_L = z_L + b_LC_{L-1}\; .}\end{equation} Our induction argument gives
us \[
C_{L-1} = u_{L-1} + b_{L-1}C_{L}\; ,
\] and when we combine these we have two equations in two unknowns and
can solve for \(C_L\) as \[
C_L = \frac{z_L+ b_Lu_{L-1}}{1-b_Lb_{L-1}}\;.
\] We then successively obtain \(C_{L-1}, C_{L-2}, \ldots, C_1\) from
Equation~\ref{eq-crank101}.

We will demonstrate the Crank-Nicolson method by valuing a European
call. Any other path-independent European derivative is valued in the
same way, by appropriately redefining the value of the derivative at the
final date and redefining the constants \(z_1\) and \(z_L\) in the
boundary conditions Equation~\ref{eq-crank4aa} -
Equation~\ref{eq-crank4bb} at the bottom and top of the grid.

As elsewhere in this chapter, \(N\) denotes the number of time periods,
and \(2M+1\) will be the number of \(x\) values on the grid. We use the
symbol \(D\) to denote the distance of the top (or bottom) of the grid
from \(\log S(0)\). In other words, \(D = x_M-x_0\). A reasonable value
for \(D\) would be three standard deviations for \(\log S\), which would
mean \(D = |\nu|T +3 \sigma\sqrt{T}\). For example, for a one-year
option on a stock with a volatility of 30\%, it should suffice to input
\(D=1\).

As should be clear, the program is conceptually very similar to a
binomial model. The difference is that the backing up procedure, which
involves node-by-node discounting in a binomial model, here is
accomplished via the Crank-Nicolson algorithm.\footnote{We use a
  different variable (\(y\)) for the call values at the final date---and
  consequently need to separate the first step of backing up (to the
  penultimate date) and the other steps of backing up (to date zero).
  See Appendix\textasciitilde A for more discussion.}

The following code illustrates how to use the Crank\_Nicolson method to
vale a European call option.

\phantomsection\label{crank_nicolson}
\begin{Shaded}
\begin{Highlighting}[]
\ImportTok{import}\NormalTok{ numpy }\ImportTok{as}\NormalTok{ np}

\KeywordTok{def}\NormalTok{ crank\_nicolson(a, y, L, z1, b1, zL, bL):}
\NormalTok{    u }\OperatorTok{=}\NormalTok{ np.zeros(L)}
\NormalTok{    b }\OperatorTok{=}\NormalTok{ np.zeros(L)}
\NormalTok{    c }\OperatorTok{=}\NormalTok{ np.zeros(L)}
\NormalTok{    z }\OperatorTok{=}\NormalTok{ np.zeros(L)}

\NormalTok{    u[}\DecValTok{0}\NormalTok{] }\OperatorTok{=}\NormalTok{ z1}
\NormalTok{    b[}\DecValTok{0}\NormalTok{] }\OperatorTok{=}\NormalTok{ b1}
    \ControlFlowTok{for}\NormalTok{ j }\KeywordTok{in} \BuiltInTok{range}\NormalTok{(}\DecValTok{1}\NormalTok{, L }\OperatorTok{{-}} \DecValTok{1}\NormalTok{):}
\NormalTok{        z[j] }\OperatorTok{=}\NormalTok{ a[}\DecValTok{3}\NormalTok{] }\OperatorTok{*}\NormalTok{ y[j] }\OperatorTok{+}\NormalTok{ a[}\DecValTok{1}\NormalTok{] }\OperatorTok{*}\NormalTok{ y[j }\OperatorTok{+} \DecValTok{1}\NormalTok{] }\OperatorTok{+}\NormalTok{ a[}\DecValTok{2}\NormalTok{] }\OperatorTok{*}\NormalTok{ y[j }\OperatorTok{{-}} \DecValTok{1}\NormalTok{]}
\NormalTok{        u[j] }\OperatorTok{=}\NormalTok{ (a[}\DecValTok{2}\NormalTok{] }\OperatorTok{*}\NormalTok{ u[j }\OperatorTok{{-}} \DecValTok{1}\NormalTok{] }\OperatorTok{+}\NormalTok{ z[j]) }\OperatorTok{/}\NormalTok{ (a[}\DecValTok{0}\NormalTok{] }\OperatorTok{{-}}\NormalTok{ a[}\DecValTok{2}\NormalTok{] }\OperatorTok{*}\NormalTok{ b[j }\OperatorTok{{-}} \DecValTok{1}\NormalTok{])}
\NormalTok{        b[j] }\OperatorTok{=}\NormalTok{ a[}\DecValTok{1}\NormalTok{] }\OperatorTok{/}\NormalTok{ (a[}\DecValTok{0}\NormalTok{] }\OperatorTok{{-}}\NormalTok{ a[}\DecValTok{2}\NormalTok{] }\OperatorTok{*}\NormalTok{ b[j }\OperatorTok{{-}} \DecValTok{1}\NormalTok{])}
\NormalTok{    c[}\OperatorTok{{-}}\DecValTok{1}\NormalTok{] }\OperatorTok{=}\NormalTok{ (zL }\OperatorTok{+}\NormalTok{ bL }\OperatorTok{*}\NormalTok{ u[}\OperatorTok{{-}}\DecValTok{2}\NormalTok{]) }\OperatorTok{/}\NormalTok{ (}\DecValTok{1} \OperatorTok{{-}}\NormalTok{ bL }\OperatorTok{*}\NormalTok{ b[}\OperatorTok{{-}}\DecValTok{2}\NormalTok{])}
    \ControlFlowTok{for}\NormalTok{ j }\KeywordTok{in} \BuiltInTok{range}\NormalTok{(L }\OperatorTok{{-}} \DecValTok{2}\NormalTok{, }\OperatorTok{{-}}\DecValTok{1}\NormalTok{, }\OperatorTok{{-}}\DecValTok{1}\NormalTok{):}
\NormalTok{        c[j] }\OperatorTok{=}\NormalTok{ u[j] }\OperatorTok{+}\NormalTok{ b[j] }\OperatorTok{*}\NormalTok{ c[j }\OperatorTok{+} \DecValTok{1}\NormalTok{]}
    \ControlFlowTok{return}\NormalTok{ c}

\KeywordTok{def}\NormalTok{ european\_call\_crank\_nicolson(S0, K, r, sigma, q, T, N, M, Dist):}
\NormalTok{    dt }\OperatorTok{=}\NormalTok{ T }\OperatorTok{/}\NormalTok{ N}
\NormalTok{    dx }\OperatorTok{=}\NormalTok{ Dist }\OperatorTok{/}\NormalTok{ M}
\NormalTok{    dx2 }\OperatorTok{=}\NormalTok{ dx }\OperatorTok{**} \DecValTok{2}
\NormalTok{    u }\OperatorTok{=}\NormalTok{ np.exp(dx)}
\NormalTok{    sig2 }\OperatorTok{=}\NormalTok{ sigma }\OperatorTok{**} \DecValTok{2}
\NormalTok{    nu }\OperatorTok{=}\NormalTok{ r }\OperatorTok{{-}}\NormalTok{ q }\OperatorTok{{-}}\NormalTok{ sig2 }\OperatorTok{/} \DecValTok{2}
\NormalTok{    St }\OperatorTok{=}\NormalTok{ S0 }\OperatorTok{*}\NormalTok{ np.exp(Dist)}
\NormalTok{    Sb }\OperatorTok{=}\NormalTok{ S0 }\OperatorTok{*}\NormalTok{ np.exp(}\OperatorTok{{-}}\NormalTok{Dist)}
\NormalTok{    a }\OperatorTok{=}\NormalTok{ np.zeros(}\DecValTok{4}\NormalTok{)}
\NormalTok{    a[}\DecValTok{0}\NormalTok{] }\OperatorTok{=}\NormalTok{ r }\OperatorTok{/} \DecValTok{2} \OperatorTok{+} \DecValTok{1} \OperatorTok{/}\NormalTok{ dt }\OperatorTok{+}\NormalTok{ sig2 }\OperatorTok{/}\NormalTok{ (}\DecValTok{2} \OperatorTok{*}\NormalTok{ dx2)}
\NormalTok{    a[}\DecValTok{1}\NormalTok{] }\OperatorTok{=}\NormalTok{ sig2 }\OperatorTok{/}\NormalTok{ (}\DecValTok{4} \OperatorTok{*}\NormalTok{ dx2) }\OperatorTok{+}\NormalTok{ nu }\OperatorTok{/}\NormalTok{ (}\DecValTok{4} \OperatorTok{*}\NormalTok{ dx)}
\NormalTok{    a[}\DecValTok{2}\NormalTok{] }\OperatorTok{=}\NormalTok{ a[}\DecValTok{1}\NormalTok{] }\OperatorTok{{-}}\NormalTok{ nu }\OperatorTok{/}\NormalTok{ (}\DecValTok{2} \OperatorTok{*}\NormalTok{ dx)}
\NormalTok{    a[}\DecValTok{3}\NormalTok{] }\OperatorTok{=} \OperatorTok{{-}}\NormalTok{a[}\DecValTok{0}\NormalTok{] }\OperatorTok{+} \DecValTok{2} \OperatorTok{/}\NormalTok{ dt}

\NormalTok{    L }\OperatorTok{=} \DecValTok{2} \OperatorTok{*}\NormalTok{ M }\OperatorTok{+} \DecValTok{1}
\NormalTok{    y }\OperatorTok{=}\NormalTok{ np.zeros(L)}
\NormalTok{    S }\OperatorTok{=}\NormalTok{ Sb}
\NormalTok{    y[}\DecValTok{0}\NormalTok{] }\OperatorTok{=} \BuiltInTok{max}\NormalTok{(S }\OperatorTok{{-}}\NormalTok{ K, }\DecValTok{0}\NormalTok{)}
    \ControlFlowTok{for}\NormalTok{ j }\KeywordTok{in} \BuiltInTok{range}\NormalTok{(}\DecValTok{1}\NormalTok{, L):}
\NormalTok{        S }\OperatorTok{*=}\NormalTok{ u}
\NormalTok{        y[j] }\OperatorTok{=} \BuiltInTok{max}\NormalTok{(S }\OperatorTok{{-}}\NormalTok{ K, }\DecValTok{0}\NormalTok{)}

\NormalTok{    z1 }\OperatorTok{=} \DecValTok{0}
\NormalTok{    b1 }\OperatorTok{=} \DecValTok{1}
\NormalTok{    zL }\OperatorTok{=}\NormalTok{ St }\OperatorTok{{-}}\NormalTok{ St }\OperatorTok{/}\NormalTok{ u}
\NormalTok{    bL }\OperatorTok{=} \DecValTok{1}
\NormalTok{    CallV }\OperatorTok{=}\NormalTok{ crank\_nicolson(a, y, L, z1, b1, zL, bL)}

    \ControlFlowTok{for}\NormalTok{ \_ }\KeywordTok{in} \BuiltInTok{range}\NormalTok{(N }\OperatorTok{{-}} \DecValTok{2}\NormalTok{, }\OperatorTok{{-}}\DecValTok{1}\NormalTok{, }\OperatorTok{{-}}\DecValTok{1}\NormalTok{):}
\NormalTok{        CallV }\OperatorTok{=}\NormalTok{ crank\_nicolson(a, CallV, L, z1, b1, zL, bL)}
    \ControlFlowTok{return}\NormalTok{ CallV[M]}

\CommentTok{\# Example usage}
\NormalTok{S0 }\OperatorTok{=} \DecValTok{100}
\NormalTok{K }\OperatorTok{=} \DecValTok{90}
\NormalTok{r }\OperatorTok{=} \FloatTok{0.05}
\NormalTok{sigma }\OperatorTok{=} \FloatTok{0.2}
\NormalTok{q }\OperatorTok{=} \FloatTok{0.02}
\NormalTok{T }\OperatorTok{=} \DecValTok{1}
\NormalTok{N }\OperatorTok{=} \DecValTok{100}
\NormalTok{M }\OperatorTok{=} \DecValTok{50}
\NormalTok{Dist }\OperatorTok{=} \DecValTok{3}
\NormalTok{Bar }\OperatorTok{=} \DecValTok{85}

\BuiltInTok{print}\NormalTok{(}\StringTok{"European Call Crank{-}Nicolson:"}\NormalTok{, european\_call\_crank\_nicolson(S0, K, r, sigma, q, T, N, M, Dist))}
\end{Highlighting}
\end{Shaded}

\begin{verbatim}
European Call Crank-Nicolson: 15.113207926084973
\end{verbatim}

\subsection{American Options}\label{american-options}

The explicit method is easily adapted to American options. As in a
binomial model, we compute the option value at each node as the larger
of its discounted expected value and its intrinsic value. To be somewhat
more precise, we replace the trinomial value
Equation~\ref{eq-fdtrinomial} with \[
C_{\text{left}} = \max\left(\big(1-r\Delta t\big)\big(p_uC_{\text{up}}+pC + p_dC_{\text{down}}\big), \text{intrinsic value}\right)\;.
\]

In the Crank-Nicolson method, one can in similar fashion compute the
value of the derivative at each space node at any date by solving the
system of equations Equation~\ref{eq-crank3} and then replace the
computed values by the intrinsic value when that is higher. However,
because the values at the different space nodes are linked (i.e., the
method is an implicit-type method), this one-at-a-time replacement of
values by intrinsic values is not the most efficient method. See Wilmott
{[}@Wilmott{]} for more details (and for VBA code implementing the
projected successive over-relaxation method).

\subsection{Barrier Options}\label{sec-s_finitedifferencebarriers}

\index{barrier option}\index{down-and-out option}\index{down-and-in option}\index{knock-out option}\index{knock-in option}Finite-difference
methods work well for valuing discretely-sampled barrier options. For a
down-and-out option, one should place the bottom of the grid at the
knock-out boundary. For an up-and-out option, one should place the top
of the grid at the knock-out boundary. As discussed in
\textbf{?@sec-c\_exotics}, knock-in options can be valued as standard
options minus knock-out options.

For barrier options, the boundary information
Equation~\ref{eq-cn_boundary} can be replaced by assigning a value of
zero at the knock-out boundary. For example, for a down-and-out option,
the condition Equation~\ref{eq-crank4a} can be replaced by \(C=0\). If
the contract specifies that a rebate is to be paid to the buyer when the
option is knocked out, then condition Equation~\ref{eq-crank4a} should
be replaced by \(C=\) Rebate.

To price a down-and-out (or up-and-out option), we put the bottom (or
top) of the grid at the boundary. The boundary condition that we use is
that the value at the boundary is zero. We will consider the example of
a down-and-out call option. In this case, the boundary condition at the
bottom of the grid is Equation~\ref{eq-crank4aa} with \(z_1=0\) and
\(b_1 = 0\). The boundary condition at the top is the same as for an
ordinary call. We can easily handle a rebate paid when the option is
knocked out by inputting the value of the rebate as \(z_1\).

The main new issue that we encounter in valuing barriers is locating the
boundary of the grid at the barrier. For the down-and-out, we will input
the value of the stock price at which the option is knocked out as
\texttt{Bar}. We want the bottom of the grid to lie at the natural
logarithm of this number. This will influence our choice of the space
step \(\Delta x\), because we want to have an integer number of steps
between the bottom of the grid and \(\log S(0)\). We assume that the
value \(M\) input by the user represents the desired number of space
steps above \(\log S(0)\). We start with \(\Delta x=D/M\) as an initial
estimate of the size of the space step. We then decrease it, if
necessary, to ensure that the distance between \texttt{Bar} and
\(\log S(0)\) is an integer multiple of \(\Delta x\). We then increase
\(M\), if necessary, to ensure that the top of the grid will still be at
or above \(D+\log S(0)\). Finally, we define the top of the grid to be
at \(\log S(0) +M \cdot \Delta x\).

\phantomsection\label{my_code_block}
\begin{Shaded}
\begin{Highlighting}[]
\KeywordTok{def}\NormalTok{ down\_and\_out\_call\_cn(S0, K, r, sigma, q, T, N, M, Dist, Bar):}
\NormalTok{    dx }\OperatorTok{=}\NormalTok{ Dist }\OperatorTok{/}\NormalTok{ M}
\NormalTok{    DistBot }\OperatorTok{=}\NormalTok{ np.log(S0) }\OperatorTok{{-}}\NormalTok{ np.log(Bar)}
\NormalTok{    NumBotSteps }\OperatorTok{=} \BuiltInTok{int}\NormalTok{(np.ceil(DistBot }\OperatorTok{/}\NormalTok{ dx))}
\NormalTok{    dx }\OperatorTok{=}\NormalTok{ DistBot }\OperatorTok{/}\NormalTok{ NumBotSteps}
\NormalTok{    NumTopSteps }\OperatorTok{=} \BuiltInTok{int}\NormalTok{(np.ceil(Dist }\OperatorTok{/}\NormalTok{ dx))}
\NormalTok{    DistTop }\OperatorTok{=}\NormalTok{ NumTopSteps }\OperatorTok{*}\NormalTok{ dx}
\NormalTok{    L }\OperatorTok{=}\NormalTok{ NumBotSteps }\OperatorTok{+}\NormalTok{ NumTopSteps }\OperatorTok{+} \DecValTok{1}
\NormalTok{    dt }\OperatorTok{=}\NormalTok{ T }\OperatorTok{/}\NormalTok{ N}
\NormalTok{    dx2 }\OperatorTok{=}\NormalTok{ dx }\OperatorTok{**} \DecValTok{2}
\NormalTok{    u }\OperatorTok{=}\NormalTok{ np.exp(dx)}
\NormalTok{    sig2 }\OperatorTok{=}\NormalTok{ sigma }\OperatorTok{**} \DecValTok{2}
\NormalTok{    nu }\OperatorTok{=}\NormalTok{ r }\OperatorTok{{-}}\NormalTok{ q }\OperatorTok{{-}}\NormalTok{ sig2 }\OperatorTok{/} \DecValTok{2}
\NormalTok{    St }\OperatorTok{=}\NormalTok{ S0 }\OperatorTok{*}\NormalTok{ np.exp(DistTop)}
\NormalTok{    a }\OperatorTok{=}\NormalTok{ np.zeros(}\DecValTok{4}\NormalTok{)}
\NormalTok{    a[}\DecValTok{0}\NormalTok{] }\OperatorTok{=}\NormalTok{ r }\OperatorTok{/} \DecValTok{2} \OperatorTok{+} \DecValTok{1} \OperatorTok{/}\NormalTok{ dt }\OperatorTok{+}\NormalTok{ sig2 }\OperatorTok{/}\NormalTok{ (}\DecValTok{2} \OperatorTok{*}\NormalTok{ dx2)}
\NormalTok{    a[}\DecValTok{1}\NormalTok{] }\OperatorTok{=}\NormalTok{ sig2 }\OperatorTok{/}\NormalTok{ (}\DecValTok{4} \OperatorTok{*}\NormalTok{ dx2) }\OperatorTok{+}\NormalTok{ nu }\OperatorTok{/}\NormalTok{ (}\DecValTok{4} \OperatorTok{*}\NormalTok{ dx)}
\NormalTok{    a[}\DecValTok{2}\NormalTok{] }\OperatorTok{=}\NormalTok{ a[}\DecValTok{1}\NormalTok{] }\OperatorTok{{-}}\NormalTok{ nu }\OperatorTok{/}\NormalTok{ (}\DecValTok{2} \OperatorTok{*}\NormalTok{ dx)}
\NormalTok{    a[}\DecValTok{3}\NormalTok{] }\OperatorTok{=} \OperatorTok{{-}}\NormalTok{a[}\DecValTok{0}\NormalTok{] }\OperatorTok{+} \DecValTok{2} \OperatorTok{/}\NormalTok{ dt}

\NormalTok{    y }\OperatorTok{=}\NormalTok{ np.zeros(L)}
\NormalTok{    S }\OperatorTok{=}\NormalTok{ Bar}
\NormalTok{    y[}\DecValTok{0}\NormalTok{] }\OperatorTok{=} \BuiltInTok{max}\NormalTok{(S }\OperatorTok{{-}}\NormalTok{ K, }\DecValTok{0}\NormalTok{)}
    \ControlFlowTok{for}\NormalTok{ j }\KeywordTok{in} \BuiltInTok{range}\NormalTok{(}\DecValTok{1}\NormalTok{, L):}
\NormalTok{        S }\OperatorTok{*=}\NormalTok{ u}
\NormalTok{        y[j] }\OperatorTok{=} \BuiltInTok{max}\NormalTok{(S }\OperatorTok{{-}}\NormalTok{ K, }\DecValTok{0}\NormalTok{)}

\NormalTok{    z1 }\OperatorTok{=} \DecValTok{0}
\NormalTok{    b1 }\OperatorTok{=} \DecValTok{0}
\NormalTok{    zL }\OperatorTok{=}\NormalTok{ St }\OperatorTok{{-}}\NormalTok{ St }\OperatorTok{/}\NormalTok{ u}
\NormalTok{    bL }\OperatorTok{=} \DecValTok{1}
\NormalTok{    CallV }\OperatorTok{=}\NormalTok{ crank\_nicolson(a, y, L, z1, b1, zL, bL)}

    \ControlFlowTok{for}\NormalTok{ \_ }\KeywordTok{in} \BuiltInTok{range}\NormalTok{(N }\OperatorTok{{-}} \DecValTok{2}\NormalTok{, }\OperatorTok{{-}}\DecValTok{1}\NormalTok{, }\OperatorTok{{-}}\DecValTok{1}\NormalTok{):}
\NormalTok{        CallV }\OperatorTok{=}\NormalTok{ crank\_nicolson(a, CallV, L, z1, b1, zL, bL)}
    \ControlFlowTok{return}\NormalTok{ CallV[NumBotSteps]}

\BuiltInTok{print}\NormalTok{(}\StringTok{"Down and Out Call CN:"}\NormalTok{, down\_and\_out\_call\_cn(S0, K, r, sigma, q, T, N, M, Dist, Bar))}
\end{Highlighting}
\end{Shaded}

\begin{verbatim}
Down and Out Call CN: 13.780367670052469
\end{verbatim}

\subsection{Exercises}\label{exercises}

\begin{exercise}[]\protect\hypertarget{exr-nolabel}{}\label{exr-nolabel}

Create a python program to compare the estimates of the value of a
discretely sampled barrier option given by the functions
Down\_And\_Out\_Call\_MC created in \textbf{?@exr-e\_knockoutmc} and the
function Down\_And\_Out\_Call\_CN. Allow the user to input \(S\), \(K\),
\(r\), \(\sigma\), \(q\), the knock-out barrier, the number of Monte
Carlo simulations, and the number of space steps above \(\log S(0)\) in
the Crank-Nicolson algorithm.

\end{exercise}

\begin{exercise}[]\protect\hypertarget{exr-nolabel}{}\label{exr-nolabel}

Create a python function Up\_And\_Out\_Put\_CN to value an up-and-out
put option by the Crank-Nicolson method.

\end{exercise}

\begin{exercise}[]\protect\hypertarget{exr-nolabel}{}\label{exr-nolabel}

Create a python function European\_Call\_Explicit that uses the explicit
method Equation~\ref{eq-fdtrinomial} to value a European call option.

\end{exercise}

\begin{exercise}[]\protect\hypertarget{exr-nolabel}{}\label{exr-nolabel}

Write the system of equations Equation~\ref{eq-implicit} for the
implicit method, together with boundary conditions of the form
Equation~\ref{eq-crank4aa} - Equation~\ref{eq-crank4bb} as a matrix
system and solve for \(u_j\) and \(b_j\) in Equation~\ref{eq-crank101} -
Equation~\ref{eq-crankC_L}, as in the subsection that defines the
function CrankNicolson.

\end{exercise}

\begin{exercise}[]\protect\hypertarget{exr-nolabel}{}\label{exr-nolabel}

Create a python function Implicit that solves the system of equations in
the preceding exercise.

\end{exercise}

\begin{exercise}[]\protect\hypertarget{exr-nolabel}{}\label{exr-nolabel}

Create a python function European\_Call\_Implicit that uses the implicit
method to value a European call option.

\end{exercise}




\end{document}
