\preface

This book is an outgrowth of notes compiled by the author while teaching courses for undergraduate and masters/MBA finance students  at Washington University in St.\ Louis and the Institut f\"ur H\"ohere Studien in Vienna.  At one time, a course in Options and Futures was considered an advanced finance elective, but now such a course is nearly mandatory for any finance major and is an elective chosen by many non-finance majors as well.  Moreover, students are exposed to derivative securities in courses on Investments, International Finance, Risk Management, Investment Banking, Fixed Income, etc.  This expansion of education in derivative securities mirrors the increased importance of derivative securities in corporate finance and investment management.  

MBA and undergraduate courses typically (and appropriately) focus on the use of derivatives for hedging and speculating.  This is sufficient for many students.  However, the seller of derivatives, in addition to needing to understand buy-side demands,  is confronted with the need to price and hedge.  Moreover, the buyer of derivatives, depending on the degree of competition between sellers, may very likely benefit from some knowledge of pricing as well.  It is ``pricing and hedging'' that is the primary focus of this book.  Through learning the fundamentals of pricing and hedging, students also acquire a deeper understanding of the contracts themselves.  
Hopefully, this book will also be of use to practitioners and for students in  Masters of Financial Engineering programs and, to some extent, Ph.D. students in finance.

The book is concerned with pricing and hedging derivatives in frictionless markets.  By ``frictionless,'' I mean  that the book ignores transaction costs (commissions, bid-ask spreads and the price impacts of trades), margin (collateral)  requirements and any restrictions on short selling.  The theory of pricing and hedging in frictionless markets stems of course from the work of Black and Scholes \cite{BS} and Merton \cite{Merton} and is a very well developed theory.  It is based on the assumption that there are no arbitrage opportunities in the market.  The theory is the foundation for pricing and hedging in markets with frictions (i.e., in real markets!) but practice can differ from theory in important ways if the frictions are significant.  For example, an arbitrage opportunity in a frictionless market often will not be an arbitrage opportunity for a trader who moves the market when he trades, faces collateral requirements, etc.    This book has nothing to say about how one should deviate from the benchmark frictionless theory when frictions are important.  Another important omission from the book is jump processes---the book deals exclusively with binomial and Brownian motion models.

The book is intended primarily to be used for advanced courses in derivative securities.  It is self-contained, and the first chapter presents the basic financial concepts.  However,  much material (functioning of security exchanges, payoff diagrams, spread strategies, etc.) that is standard in an introductory book has not been included here.  On the other hand, though it is not an introductory book, it is not truly an advanced book on derivatives either.   On any of the topics covered in the book, there are more advanced treatments available in book form already.  However, the books that I have seen (and there are indeed many)  are either too narrow in focus for the courses I taught or not easily accessible to the students I taught or (most commonly) both.  If this book is successful, it will be as a bridge between an introductory course in Options and Futures and the more advanced literature.  Towards that end, I have included cites to more advanced books in appropriate places throughout.

The book includes an introduction to computational methods, and the term ``introduction'' is meant quite seriously here.  The book was developed for students with no prior experience in programming or numerical analysis, and it only covers the most basic ideas.  Nevertheless, I believe that this is an extremely important feature of the book.  It is my experience that the theory becomes much more accessible to students when they  learn to code a formula or to simulate a process. 
The book builds up to binomial, Monte Carlo, and finite-difference methods by first developing simple programs for simple computations.  These serve two roles: they introduce the student to programming, and they result in tools that enable students to  solve real problems, allowing the inclusion of exercises of a practical rather than purely theoretical nature.   I have used the book for  semester-length courses emphasizing calculation (most of the exercises are of that form) and for short courses covering only the theory.  

Nearly all of the formulas and procedures described in the book are both derived from first principles  and implemented in Excel VBA.  The VBA programs are in the text and in an Excel workbook that can be downloaded free of charge at \verb!www.kerryback.net!.  I use a few special features of Excel, in particular the cumulative normal distribution function and the random number generator.  Otherwise, the programs can easily be translated into any other language.  In particular, it is easy to translate them into MATLAB, which also includes a random number generator and the cumulative normal distribution function (or, rather, the closely related ``error function'') as part of its basic implementation.   I chose VBA because students (finance students, at least) can be expected to already have it on their computers and because Excel is a good environment for many exercises, such as analyzing hedges, that do not require programming.  An appendix provides the necessary introduction to VBA programming.  

Viewed as a math book, this is a book in applied math, not math proper.  My goal is to get students as quickly as possible to the point where they can compute things.  Many mathematical issues  (filtrations, completion of filtrations, formal definitions of expectations and conditional expectations, etc.) are entirely ignored.  It would not be unfair to call this a ``cookbook'' approach.  I try to explain intuitively why the recipes work but do not give proofs or even formal statements of the facts that underlie them.

I have naturally taken pains to present the theory in what I think is the simplest possible manner.   The book uses almost exclusively the probabilistic/martingale approach, both because it is my preference and because it seems easier than partial differential equations for students in business and the social sciences to grasp.    A sampling of some of the more or less distinctive characteristics of the book, in terms of exposition, is:
\begin{itemize}
\item Important theoretical results are highlighted in boxes for easy reference; the derivations that are less important and more technical are presented in smaller type and relegated to the ends of sections.
\item Changes of numeraire are introduced in the first chapter in a one-period binomial model, the probability measure corresponding to the underlying as numeraire being given as much emphasis as the risk-neutral measure. 
\item The fundamental result for pricing (asset prices are martingales under changes of numeraire) is presented in the first chapter, because it does not need the machinery of  stochastic calculus.
\item The basic ideas in pricing digital and share digitals, and hence in deriving the Black-Scholes formula, are also presented in the first chapter.  Digitals and share digitals are priced in Chap.~3 before calls and puts.
\item Brownian motion is introduced by simulating it in discrete time.  The quadratic variation property is emphasized, including exercises that contrast Brownian motion with continuously differentiable functions of time, in order to motivate It\^o's formula.
\item The distribution of the underlying under different numeraires is derived directly from the fundamental pricing result and It\^o's formula, bypassing Girsanov's theorem (which is of course also a consequence of It\^o's formula).
\item Substantial emphasis is placed on forwards, synthetic forwards, options on forwards and hedging with forwards because these have many applications in fixed income and elsewhere---a simple but characteristic example is valuing a European option on a stock paying a known cash dividend as a European option on the synthetic forward with the same maturity.
\item Following Margrabe \cite{Margrabe} (who attributes the idea to S. Ross) the formula for exchange options is derived by a change of numeraire from the Black-Scholes formula.  Very simple  arguments derive Black's formula for forward and futures options from Margrabe's formula and Merton's formula for stock options in the absence of a constant risk-free rate from Black's formula.  This demonstrates the  equivalence of these important option pricing formulas as follows:
\begin{align*}
\text{Black-Scholes} &\Longrightarrow \text{Margrabe}\\
& \Longrightarrow \text{Black}\\
& \Longrightarrow \text{Merton}\\
& \Longrightarrow \text{Black-Scholes}
\end{align*}
\item Quanto forwards and options are priced by first finding the portfolio that replicates the value of a foreign security translated at a fixed exchange rate and then viewing quanto forwards and options as standard forwards and options on the replicating portfolio.
\item The market model is presented as an introduction to the pricing of fixed-income derivatives.  Forward rates are shown to be martingales under the forward measure by virtue of their being forward prices of portfolios that pay spot rates.
\item In order to illustrate how term structure models are used to price fixed-income derivatives, the Vasicek/Hull-White model is worked out in great detail.  Other important term structure models are discussed much more briefly.
\end{itemize}
Of course, none of these items is original, but in conjunction with the computational tools, I believe they make the ``rocket science'' of derivative securities accessible to a broader group of students.

The book is divided into three parts, labeled ``Introduction to Option Pricing,'' ``Advanced Option Pricing,'' and ``Fixed Income.''  Naturally, many of the chapters build upon one another, but it is possible to read Chaps.~\ref{c_basics}--\ref{c_blackscholes}, Sects.~\ref{s_margrabe}--\ref{s_black}  (the Margrabe and Black formulas) and then  Part III on fixed income.  For a more complete coverage, but still omitting two of the more difficult chapters, one could read all of Parts I and II except Chaps.~8 and~10, pausing in  Chap.~8 to read the definitions of baskets, spreads, barriers, lookbacks and Asians  and in Chap.~10 to read the discussion of the fundamental partial differential equation.

I would like to thank Mark Broadie, the series editor, for helpful comments, and especially I want to thank my wife, Diana, without whose encouragement and support I could not have written this.  She mowed the lawn---and managed everything else---while I typed, and that is a great gift.
\vspace{1cm}
\begin{flushright}\noindent
College Station, Texas\hfill {\it Kerry Back}\\
January, 2005 \hfill \\
\end{flushright}
