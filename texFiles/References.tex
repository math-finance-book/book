\begin{thebibliography}{[KLR73]}
\bibitem{Arrow} Arrow, K.J.: The role of securities in the optimal allocation of risk bearing.  Review of Economic Studies, \textbf{31}, 91--96 (1964).  Translation of Le r\^ole des valeurs boursi\`eres pour la repartition la meillure des risques.  Econometrie (1952)
\bibitem{BR}Bielecki, T. , Rutkowski, M.: Credit Risk: Modeling, Valuation and Hedging. Springer, Berlin Heidelberg New York (2002)
\bibitem{Black} Black, F.: The pricing of commodity contracts.  Journal of Financial Economics, \textbf{3}, 167--179 (1976)
\bibitem{BDT} Black, F., Derman, E., Toy, W.: A one-factor model of interest rates and its application to Treasury bond options. Financial Analysts Journal, January/February,  33--39 (1990)
\bibitem{BK} Black, F., Karasinski, P.: Bond and option pricing when short rates are lognormal. Financial Analysts Journal, July-August,  52--59 (1991)
\bibitem{BS} Black, F., Scholes, M.: The pricing of options and corporate liabilities. Journal of Political Economy, \textbf{81},  637--654 (1973)
\bibitem{Bollerslev} Bollerslev, T.: Generalized autoregressive conditional heteroskedasticity. Journal of Econometrics, \textbf{31},  307--327 (1986)
\bibitem{boyle} Boyle, P.: Options: a Monte Carlo approach.  Journal of Financial Economics, \textbf{4}, 323--338 (1977)
\bibitem{BGM} Brace, A., Gatarek, D. , Musiela, M.: The market model of interest rate dynamics. Mathematical Finance, \textbf{7},  127--154 (1996)
\bibitem{Brandimarte}Brandimarte, P.: Numerical Methods in Finance:  A MATLAB-Based Introduction, Wiley, New York (2002)
\bibitem{BrennanSchwartz} Brennan, M., Schwartz, E.: Finite difference methods and jump processes arising in the pricing of contingent claims:  a synthesis. Journal of Financial and Quantitative Analysis 13,  461--474 (1978)
\bibitem{BM}Brigo, D., Mercurio, F.: Interest Rate Models,  Theory and Practice, Springer, Berlin Heidelberg New York (2001)
\bibitem{BD} Broadie, M., Detemple, J.: American option valuation:  new bounds, approximations, and a comparison of existing methods. Review of Financial Studies, \textbf{9},  1211--1250 (1997)
\bibitem{bg2} Broadie, M., Glasserman, P.: Estimating security price derivatives using simulation.  Management Science, \textbf{42}, 269--285 (1996)
\bibitem{BG} Broadie, M., Glasserman, P.: Pricing American-style securities using simulation. Journal of Economic Dynamics and Control, \textbf{21},  1323--1352 (1997)
\bibitem{broadiekaya} Broadie, M., Kaya, O.: Exact simulation of stochastic volatility and other affine jump diffusion processes.  Operations Research (forthcoming)
\bibitem{CS}Clewlow, L., Strickland, C.: Implementing Derivatives Models, Wiley, New York (1998)
\bibitem{CV} Conze, A., Viswanathan.: Path dependent options: the case of lookback options. Journal of Finance, \textbf{46},  1893--1907 (1991)
\bibitem{CIR} Cox, J., Ingersoll, J., Ross, S.: A theory of the term structure of interest rates. Econometrica, \textbf{53},  385--408 (1985)
\bibitem{CR} Cox, J., Ross, S.: The valuation of options for alternative stochastic processes. Journal of Financial Economics, \textbf{3},  145--166 (1976)
\bibitem{CRR} Cox, J., Ross, S., Rubinstein, M.: Option pricing:  a simplified approach. Journal of Financial Economics, \textbf{7},  229--263 (1979)
\bibitem{DSing} Dai, Q., Singleton, K.: Specification analysis of affine term structure models. Journal of Finance, \textbf{55},  1943--1978 (2000)
\bibitem{Drezner} Drezner, Z.: Computation of the bivariate normal integral. Mathematics of Computation, \textbf{32},   277-279 (1978)
\bibitem{DK} Duffie, D., Kan, R.: A yield-factor model of interest rates., Mathematical Finance, \textbf{6}, 379--406 (1996)
\bibitem{DS}Duffie, D., Singleton, K.: Credit Risk:  Pricing, Measurement and Management, Princeton University Press, Princeton, New Jersey (2003)
\bibitem{Epps}Epps, T. W.: Pricing Derivative Securities, World Scientific Publishing, Singapore (2000)
\bibitem{GKR} Geman, H., El Karoui, N., Rochet, J.-C.: Changes of numeraire, changes of probability measure and option pricing. Journal of Applied Probability, \textbf{32},  443--458 (1995)
\bibitem{Geske} Geske, R.: The valuation of compound options. Journal of Financial Economics, \textbf{7},  63--81 (1979)
\bibitem{Glasserman} Glasserman, P.: Monte Carlo Methods in Financial Engineering, Springer, New York Berlin Heidelberg (2004)
\bibitem{GSG} Goldman, M., Sosin, H., Gatto, M.: Path dependent options:  `buy at the low, sell at the high.'  Journal of Finance, \textbf{34},  1111--1127 (1979)
\bibitem{HK} Harrison, J.M., Kreps, D.: Martingales and arbitrage in multiperiod securities markets. Journal of Economic Theory, \textbf{20},  381--408 (1979)
\bibitem{Haug} Haug, E.G.: The Complete Guide to Option Pricing Formulas, McGraw-Hill, New York (1998)
\bibitem{HJM}Heath, D., Jarrow, R., Morton, A.: Bond pricing and the term structure of interest rates:  a new methodology for contingent claims valuation. Econometrica, \textbf{60},  77--105 (1992)
\bibitem{Heston} Heston, S.: A closed-form solution for options with stochastic volatility with applications to bond and currency options. Review of Financial Studies, \textbf{6},  327--344 (1993)
\bibitem{HN} Heston, S., Nandi, S.: A closed-form GARCH option valuation model. Review of Financial Studies, \textbf{13},  585--625 (2000)
\bibitem{HL} Ho, T., Lee, S.: Term structure movements and pricing interest rate contingent claims. Journal of Finance, \textbf{41},  1011--1029 (1986)
\bibitem{Hull} Hull, J.: Options, Futures, and Other Derivatives,  Prentice-Hall, 5th~ed, Upper Saddle River, New Jersey (2002)
\bibitem{HW} Hull, J., White, A.: Pricing interest-rate-derivative securities. Review of Financial Studies, \textbf{3},  573--592 (1990)
\bibitem{Jackel}J\"ackel, P.: Monte Carlo Methods in Finance, Wiley, New York (2002)
\bibitem{JS}Jackson, M., Staunton, M.: Advanced Modelling in Finance Using Excel and VBA, Wiley, New York (2001)
\bibitem{JW}James J., Webber, N.: Interest Rate Modelling, Wiley, New York (2000)
\bibitem{Jamshidian89} Jamshidian, F.: An exact bond option formula., Journal of Finance, \textbf{44},  205--209 (1989)
\bibitem{Jamshidian97} Jamshidian, F.: LIBOR and swap market models and measures. Finance and Stochastics, \textbf{1},  293--330 (1997)
\bibitem{JR}Jarrow, R., Rudd, A.: Option Pricing, Dow Jones-Irwin, Homewood, Illinois (1983)
\bibitem{KS}Karatzas, I., Shreve, S.: Brownian Motion and Stochastic Calculus, Springer, New York Berlin Heidelberg (1988)
\bibitem{LR} Leisen, D.P.J., Reimer, M.: Binomial models for option valuation---examining and improving convergence., Applied Mathematical Finance, \textbf{3},  319--346 (1996)
\bibitem{LS92} Longstaff, F., Schwartz, E.: Interest rate volatility and the term structure:  a two-factor general equilibrium model., Journal of Finance, \textbf{47},  1259--1282 (1992)
\bibitem{LS01} Longstaff, F., Schwartz, E.: Valuing American options by simulation:  a simple least-squares approach. Review of Financial Studies, \textbf{14},  113--147 (2001)
\bibitem{McDonald} McDonald, R.: Derivatives Markets, Addison Wesley, Boston (2003)
\bibitem{Margrabe} Margrabe, W.: The value of an option to exchange one asset for another. Journal of Finance, \textbf{33},  177--186 (1978)
\bibitem{Merton} Merton, R.: Theory of rational option pricing. Bell Journal of Economics and Management Science, \textbf{4},  141--183 (1973)
\bibitem{MSS} Miltersen, K.R., Sandermann, K., Sondermann, D.: Closed form solutions for term structure derivatives with log-normal interest rates., Journal of Finance, \textbf{52},  409--430 (1997)
\bibitem{MX} Mina, J., Xiao, J.: Return to RiskMetrics,  The Evolution of a Standard (2001)
\bibitem{MR} Musiela, M., Rutkowski, M.: Martingale Methods in Financial Modeling, Springer, Berlin Heidelberg New York (1997)
\bibitem{Rebonato98} Rebonato, R.: Interest-Rate Option Models, 2nd ed., Wiley, New York,  (1998)
\bibitem{Rebonato02} Rebonato, R.: Modern Pricing of Interest-Rate Derivatives,  The LIBOR Market Model and Beyond, Princeton University Press, Princeton, New Jersey (2002)
\bibitem{Schonbucher} Sch{\"o}nbucher, P.: Credit Derivatives Pricing Models:  Models, Pricing and Implementation, Wiley, New York (2003)
\bibitem{Stulz} Stulz, R.: Options on the minimum or maximum of two risky assets:  analysis and applications, Journal of Financial Economics, \textbf{10},  161--185 (1982)
\bibitem{Tavakoli} Tavakoli, J.: Credit Derivatives and Synthetic Structures, Wiley, New York (2001)
\bibitem{Tavella} Tavella, D.A.: Quantitative Methods in Derivatives Pricing, Wiley, New York (2002)
\bibitem{Trigeorgis} Trigeorgis, A.: A log-transformed binomial analysis method for valuing complex multi-option investments. Journal of Financial and Quantitative Analysis, \textbf{26},  309--326 (1991)
\bibitem{Vasicek} Vasicek, O.: An equilibrium characterization of the term structure. Journal of Financial Economics, \textbf{5},  177--188 (1977)
\bibitem{Wilmott} Wilmott, P.: Paul Wilmott on Quantitative Finance, Vol.\ 2, Wiley, New York (2000)
\bibitem{WDH} Wilmott, P., Dewynne, J., Howison, S.: Option Pricing:  Mathematical Models and Computation, Oxford Financial Press, Oxford (2000)
\bibitem{Zhang} Zhang, P.G.: Exotic Options,  A Guide to Second Generation Options, 2nd ed., World Scientific Publishing, Singapore (1998)
\end{thebibliography}