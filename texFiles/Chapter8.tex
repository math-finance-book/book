\chapter{Exotic Options}\label{c_exotics}

We will only discuss a few exotic options.  The reason for studying the derivation of an option pricing formula is that it may better equip one to analyze new products.  Hopefully, this chapter will be of some assistance in that regard.  Of course, one cannot expect to derive a closed-form solution for the value of every product, and often numerical methods  will be necessary.  For a much more comprehensive presentation of exotic option pricing formulas, the book by Haug \cite{Haug} and the Excel software that accompanies it are highly recommended.  Zhang \cite{Zhang} is also a comprehensive reference for exotics. 

The valuation of an American call option on an asset paying a discrete cash dividend (rather than a continuous dividend) is considered in Sect.~\ref{s_discrete}.  Under a particular assumption on the volatility, valuing this option is very similar to valuing a compound option.  Except for the assumption of a discrete dividend in Sect.~\ref{s_discrete}, we will make the Black-Scholes assumptions throughout this chapter: there is a constant risk-free rate $r$ and the underlying asset has a  constant volatility~$\sigma$ and a constant dividend yield~$q$. 

The order in which exotics are presented in this chapter is based on the simplicity of the analysis---the chapter begins with the easiest to analyze and works towards the more difficult.  This order is roughly the inverse of importance, the most important exotics in practice being barriers, baskets, spreads and Asians.  In the cases of Asian and basket options, we will explain why there are no simple closed-form formulas (sums of lognormally distributed random variables are not lognormal).  We will use these cases and discretely-sampled barriers and lookbacks as examples in the next chapter.  


\section{Forward-Start Options}
A forward-start option \index{forward-start option} is an option for which the strike price is set equal to the stock price at some later date.  In essence, it is issued at the later date, with the strike price set at the money.  For example, an executive may know that he is to be given an option grant at some later date with the strike price set equal to the stock price at that date.

\subsection*{Forward-Start Call Payoff}

A forward-start call is defined by its maturity date $T'$ and the date $T<T'$ at which the strike price is set.  The value of a forward-start call at maturity is
$$\max(0,S(T')-S(T))\; .$$
Let
$$x= \begin{cases} 1 &\text{if $S(T')>S(T)$\;,}\\
                               0 & \text{otherwise\;.}
        \end{cases}
$$
Then, the value of the call at maturity can be written as 
$$xS(T')-xS(T)\; .$$

\subsection*{Numeraires}
\begin{enumerate}
\item Use $V(t)=\E^{qt}S(t)$ as numeraire to price the payoff $xS(T')$.  From the fundamental pricing formula \eqref{formula}, the value at date~0 is
$$\E^{-qT'}S(0)E^V[x] = \E^{-qT'}S(0)\times \text{prob}^V\!(S(T')>S(T))\; .$$
\item To price the payoff $xS(T)$, use the following portfolio as numeraire:\footnote{We are going to use equation~\eqref{probSnumeraire} at date $T'$ to define the probabilities, because it will not be known until date $T'$ whether the event $S(T')>S(T)$ is true.  Thus, we need the price of a numeraire asset at date~$T'$.  We would like this price to be a constant times $S(T)$, which is what we will obtain.  An equivalent numeraire is to make a smaller investment in the same portfolio: start with $\E^{-r(T'-T)-qT}$ shares.  This results in a final value of $S(T)$ at date~$T'$.  As will be seen, this is useful for deriving the put-call parity relation for forward-start options.\label{forwardstartnote}} purchase $\E^{-qT}$ shares of the stock at date 0 and reinvest dividends until date~$T$.  This will result in the ownership of one share at date $T$, worth $S(T)$ dollars.  At date $T$, sell the share and invest the proceeds in the risk-free asset and hold this position until date $T'$.  At date $T'$, the portfolio will be worth $\E^{r(T'-T)}S(T)$.  Let $Z(t)$ denote the value of this portfolio for each $0\leq t\leq T'$.  The fundamental pricing formula \eqref{formula} implies that the value of receiving $xS(T)$ at date $T'$ is
\begin{align*}
Z(0)E^Z\left[ \frac{xS(T)}{Z(T')}\right] &=
\E^{-qT}S(0)E^Z\left[ \frac{xS(T)}{\E^{r(T'-T)}S(T)}\right]\\&= \E^{-qT-r(T'-T)}S(0)E^Z[x] \\&= \E^{-qT-r(T'-T)}S(0) \times\text{prob}^Z(S(T')>S(T))\;.
\end{align*}
\end{enumerate}


\subsection*{Calculating Probabilities}

\begin{enumerate}
\item As in the case of a share digital, we know that
$$\log S(t) = \log S(0) + \left(r-q +\frac{1}{2}\sigma^2\right)t + \sigma B^*(t)$$
for all $t>0$,
where $B^*$ is a Brownian motion when $V$ is used as the numeraire.  Taking $t=T'$ and $t=T$ and subtracting yields
$$\log S(T')-\log S(T) = \left(r-q +\frac{1}{2}\sigma^2\right)(T'-T) + \sigma \left[B^*(T')-B^*(T)\right]\; .$$
Hence, $S(T')>S(T)$ if and only if
$$-\frac{B^*(T')-B^*(T)}{\sqrt{T'-T}} < \frac{\left(r-q +\frac{1}{2}\sigma^2\right)(T'-T)}{\sigma\sqrt{T'-T}}\; .$$
The random variable on the left hand side is a standard normal, so
$$ \text{prob}^V\!(S(T')>S(T)) = \N(d_1)\; ,$$
where
\begin{subequations}\label{forwardstartcombined}
\begin{equation}\label{forwardstart_d1}
d_1 = \frac{\left(r-q +\frac{1}{2}\sigma^2\right)(T'-T)}{\sigma\sqrt{T'-T}} =\frac{\left(r-q +\frac{1}{2}\sigma^2\right)\sqrt{T'-T}}{\sigma}\;.
\end{equation}

\item To calculate the probability $\text{prob}^Z(S(T')>S(T))$, note that between $T$ and $T'$, the portfolio with price $Z$ earns the risk-free rate $r$.  The same argument presented in Sect.~\ref{s_girsanov} shows that between $T$ and $T'$ we have
$$\frac{\D S}{S} = (r-q)\,\D t + \sigma\,\D B^*\; ,$$
where now $B^*$ denotes a Brownian motion when $Z$ is used as the numeraire.   This implies as usual that
$$ \D\log S = \left(r-q-\frac{1}{2}\sigma^2\right)\,\D t + \sigma\,\D B^*\; ,$$
which means that
$$\log S(T') - \log S(T) = \left(r-q-\frac{1}{2}\sigma^2\right)(T'-T) + \sigma(B^*(T')-B^*(T))\; .$$
Hence, $S(T')>S(T)$ if and only if
$$-\frac{B^*(T')-B^*(T)}{\sqrt{T'-T}} < \frac{\left(r-q -\frac{1}{2}\sigma^2\right)(T'-T)}{\sigma\sqrt{T'-T}}\; .$$
As before, the random variable on the left hand side is a standard normal, so
$$\text{prob}^Z(S(T')>S(T)) = \N(d_2)\; ,$$
where
\begin{equation}\label{forwardstart_d2}
d_2 = \frac{\left(r-q -\frac{1}{2}\sigma^2\right)\sqrt{T'-T}}{\sigma}=d_1-\sigma\sqrt{T'-T}\;.
\end{equation}\end{subequations}
\end{enumerate}

\subsection*{Forward-Start Call Pricing Formula}
Combining these results, we have:
\mybox{The value of a forward-start call at date 0 is
\begin{equation}\label{fstrikecall}
\E^{-qT'}S(0)\N(d_1) - \E^{-qT-r(T'-T)}S(0)\N(d_2)\;,
\end{equation}
where $d_1$ and $d_2$ are defined in \eqref{forwardstartcombined}.}

\subsection*{Put-Call Parity}\index{put-call parity}

\next Forward-strike calls and puts satisfy a somewhat unusual form of put-call parity.  The usual put-call parity is of the form:
$$\text{Call} \;+\; \text{Cash} \quad = \quad \text{Put} \;+ \;\text{Underlying}\; .$$
The amount of cash is the amount that will accumulate to the exercise price at maturity; i.e., it is $\E^{-rT'}K$.  For forward-start calls and puts, the effective exercise price is $S(T)$, which is not known at date 0.  However, the portfolio used as numeraire to value the second part of the payoff will be worth $\E^{r(T'-T)}S(T)$ at date $T'$, and by following the same strategy but starting with $\E^{-r(T'-T)-qT}$ instead of $\E^{-qT}$ shares, we will have $S(T)$ dollars at date $T'$.  The date--0 value of this portfolio should replace ``Cash'' in the above.  Thus:
\mybox{Put-call parity for forward-start calls and puts is
\begin{equation}\label{fstrikeparity}
\text{Call Price} \;+\; \E^{-r(T'-T)-qT}S(0) = \text{Put Price} \;+\; \E^{-qT'}S(0)\;.
\end{equation}}


\section{Compound Options}

A compound option \index{compound option} is an option on an option, for example a call option on a call option or a call on a put.   These options are useful for hedging when there is some uncertainty about the need for hedging which may be resolved by the exercise date of the compound option.  As speculative trades, they have the benefit of higher leverage than ordinary options.  These options were first discussed by Geske \cite{Geske}.

\subsection*{Call-on-a-Call Payoff}

Let the underlying call option have exercise price $K'$ and maturity $T'$.  Consider an option maturing at $T<T'$ to purchase the underlying call at price~$K$.

Let $C(t,S)$ denote the value at date $t$ of the underlying call when the stock price is $S$ (i.e., $C$ is the Black-Scholes formula).    It is of course rational to exercise the compound call at date $T$ if the value of the underlying call exceeds $K$; i.e., if $C(T,S(T))>K$.  Let $S^*$ denote the critical price such that $C(T,S^*)=K$.  To calculate $S^*$, we need to solve
\begin{center}
\verb!Black_Scholes_Call(Sstar,Kprime,r,sigma,q,Tprime-T) = K!.
\end{center}
for $S^*$.  We can do this by bisection or one of the other methods mentioned in Sect.~\ref{s_impliedvolatility}.
It is rational to exercise the compound option 
when
$S(T) > S^*$.  

When $S(T) > S^*$,  exercise of the compound option generates a cash flow of $-K$ at date $T$.  There is a cash flow (of $S(T')-K'$) at date $T'$ only if the compound call is exercised and the underlying call finishes in the money.  This is equivalent to:
\begin{equation}\label{compound1}
S(T) > S^* \quad\text{and}\quad S(T')>K'\;.
\end{equation}
Let
\begin{align*}
x&= \begin{cases} 1 &\text{if $S(T)>S^*$\;,}\\
                               0 & \text{otherwise\;,}
        \end{cases}\\
\intertext{and let}
y&= \begin{cases} 1 &\text{if $S(T)>S^*$ and $S(T')>K'$\;,}\\
                               0 & \text{otherwise\;.}
        \end{cases} 
\end{align*} 
The cash flows of the compound option are $-xK$ at date $T$ and $yS(T')-yK'$ at date $T'$.  We can value the compound option at date 0 by valuing these separate cash flows.

The cash flow $-xK$ is the cash flow from being short $K$ digital options on the underlying asset with strike price $S^*$ and maturity $T$.  Therefore the value at date 0 of this cash flow is $-\E^{-rT}K\N(d_2)$, where
\begin{equation}\label{calloncalld1d2}
d_1 = \frac{\log\left(\frac{S(0)}{S^*}\right)+\left(r-q+\frac{1}{2}\sigma^2\right)T}{\sigma\sqrt{T}},  \qquad d_2 = d_1-\sigma\sqrt{T}\;.
\end{equation}


\subsection*{Numeraires}
The payoffs $yS(T)$ and $yK'$ are similar to share digitals and digitals, respectively, except that the event $y=1$ is more complex than we have previously encountered.  However, we know from the analysis of share digitals and digitals that the values at date 0 of these payoffs are 
$$\E^{-q T'}S(0)\times\text{prob}^V\!(y=1) \quad \text{and}\quad \E^{-rT'}K'\times\text{prob}^R(y=1)\; ,$$
where $V(t)=\E^{qt}S(t)$ and $R(t)=\E^{rt}$.  

\subsection*{Calculating Probabilities}

We will calculate the two probabilities in terms of the bivariate normal distribution function.  
\begin{enumerate}
\item The event $y=1$ is equivalent to
$$\log S(0) + \left(r-q+\frac{1}{2}\sigma^2\right)T+\sigma B^*(T) > \log S^*$$
and
$$\log S(0) + \left(r-q+\frac{1}{2}\sigma^2\right)T'+\sigma B^*(T') > \log K'\; ,$$
where $B^*$ is a Brownian motion when the underlying asset ($V$) is used as the numeraire. These conditions can be rearranged as
\begin{equation}\label{new10}
- \frac{B^*(T)}{\sqrt{T}} < d_1 \quad \text{and} \quad - \frac{B^*(T')}{\sqrt{T'}} < d_1'\;,
\end{equation}
where $d_1$ is defined in \eqref{calloncalld1d2}, and
\begin{equation}\label{callcallds}
d_1' = \frac{\log\left(\frac{S(0)}{K'}\right)+\left(r-q+\frac{1}{2}\sigma^2\right)T'}{\sigma\sqrt{T'}}\;,
 \qquad d_2'=d_1'-\sigma\sqrt{T'}\;.
\end{equation}


The two standard normal variables on the left-hand sides in \eqref{new10} have a covariance equal to
$$\frac{1}{\sqrt{TT'}}\cov(B(T),B(T')) = \frac{1}{\sqrt{TT'}}\cov(B(T),B(T)) = \sqrt{\frac{T}{T'}}\; ,$$
the first equality following from the fact that $B(T)$ is independent of $B(T')-B(T)$ and the second from the fact that the covariance of a random variable with itself is its variance.
Hence, $\text{prob}^V\!(y=1)$ is the probability that $a\leq d_1$ and $b\leq d_1'$, where $a$ and $b$ are standard normal random variables with covariance (= correlation coefficient) of $\sqrt{T/T'}$.  We will write this probability as $\M\!\left(d_1,d_1',\sqrt{T/T'}\right)$.  A program to approximate the bivariate normal \index{bivariate normal distribution function} distribution function $\M$ is provided in Sect.~\ref{s_exotics_matlab}.

\item The calculation for $\text{prob}^R(y=1)$ is similar.  The event $y=1$
is equivalent to
$$\log S(0) + \left(r-q+\frac{1}{2}\sigma^2\right)T+\sigma B^*(T) > \log S^*\;,$$
and
$$\log S(0) + \left(r-q+\frac{1}{2}\sigma^2\right)T'+\sigma B^*(T') > \log K'\; ,$$
where $B^*$ now denotes a Brownian motion under the risk-neutral measure.  These are equivalent to
\begin{equation}\label{new11000}
- \frac{B^*(T)}{\sqrt{T}} < d_2 \quad \text{and} \quad - \frac{B^*(T')}{\sqrt{T'}} < d_2'\;.
\end{equation}
Hence, $\text{prob}^R(y=1)=\M\!\left(d_2,d_2',\sqrt{T/T'}\right)$.  
\end{enumerate}
\subsection*{Call-on-a-Call Pricing Formula}

We conclude:

\mybox{The value of a call on a call is
\begin{multline}
-\E^{-rT}K\N(d_2) + \E^{-q T'}S(0)\M\!\left(d_1,d_1',\sqrt{T/T'}\right)
 \\- \E^{-rT'}K'\M\!\left(d_2,d_2',\sqrt{T/T'}\right)\;,
 \end{multline}
 where $d_1$ and  $d_2$ are defined in \eqref{calloncalld1d2} and $d_1'$ and $d_2'$ are defined in \eqref{callcallds}.}

\subsection*{Put-Call Parity}\index{put-call parity}

European compound options with the same underlyings and strikes satisfy put-call parity in the usual way:
$$\text{Cash} + \text{Call} = \text{Underlying} + \text{Put}\; .$$
The portfolio on each side of this equation gives the owner the maximum of the strike and the value of the underlying at the option maturity.  In the case of options on calls, put-call parity is specifically
\begin{multline*}
\E^{-rT}K + \text{Value of call on call} \\= \text{Value of underlying call} + \text{Value of put on call}\; ,
\end{multline*}
where $K$ is the strike price of the compound options and $T$ is their maturity date.  Likewise, for options on puts, we have
\begin{multline*}
\E^{-rT}K + \text{Value of call on put} \\= \text{Value of underlying put} + \text{Value of put on put}\; .
\end{multline*}
Thus, the value of a put on a call can be derived from the value of a call on a call.  The value of a put on a put can be derived from the value of a call on a put, which we will now consider.

\subsection*{Call-on-a-Put Pricing Formula}

Consider a call option maturing at $T$ with strike $K$ with the underlying being a put option with strike $K'$ and maturity $T'>T$.  The underlying of the put is the asset with price $S$ and constant volatility $\sigma$.  The call on the put will never be in the money at $T$ and hence is worthless if $K> \E^{-r(T'-T)}K'$, because the maximum possible value of the put option at date $T$ is $\E^{-r(T'-T)}K'$.  So assume $K< \E^{-r(T'-T)}K'$.  

Let $S^*$ again denote the critical value of the stock price such that the call is at the money at date $T$ when $S(T)=S^*$.  This means that $S^*$ solves
\begin{center}
\verb!Black_Scholes_Put(Sstar,Kprime,r,sigma,q,Tprime-T) = K!.
\end{center} 
We leave it as an exercise to confirm the following.

\mybox{The value of a call on a put is
\begin{multline}\label{callonaput}
-\E^{-rT}K\N(-d_2) + \E^{-rT'}K'\M\!\left(-d_2,-d_2',\sqrt{T/T'}\right) \\- \E^{-q T'}S(0)\M\!\left(-d_1,-d_1',\sqrt{T/T'}\right) \;,
 \end{multline}
 where $d_1$ and  $d_2$ are defined in \eqref{calloncalld1d2} and $d_1'$ and $d_2'$ are defined in \eqref{callcallds}.}

\section{American Calls with Discrete Dividends}\label{s_discrete}

It can be optimal to exercise an American call option early if the underlying asset pays a dividend.  The optimal exercise date will be immediately prior to the asset going ``ex-dividend.''  Consider a call option maturing at $T'$ on an asset that will pay a known cash dividend $D$ at a known date $T<T'$.  We assume there is no continuous dividend payment, so $q=0$.  For simplicity, we assume that the date of the dividend payment is also the date that the asset goes ex-dividend; i.e., ownership of the asset at any date $t<T$ entitles the owner to receive the dividend at date $T$.  Under this assumption, it is reasonable also to assume that the stock price drops by $D$ when the dividend is paid.  

There is some ambiguity about how to define the asset price at the instant the dividend is paid---whether to include or exclude the dividend.  We will let $S(T)$ denote the price including the dividend and denote the price excluding the dividend by $Z(T)$, so $Z(T) = S(T)-D$.   In fact, it is convenient to let $Z(t)$ denote the price stripped of the dividend value at all dates $t \leq T$, so we will define 
$$
Z(t) = \begin{cases}S(t)-\E^{-r(T-t)}D & \text{if $t \leq T$\;,}\\
S(t) & \text{if $t > T$\;.}
\end{cases}$$
Note that $Z$ is the price of the following non-dividend-paying portfolio: buy one unit of the asset at date 0, borrow $\E^{-rT}D$ at date 0 to help finance the purchase, and use the dividend~$D$ at date $T$ to retire the debt.

If we assume as usual that the asset price $S$ has a constant volatility, then, using formula \eqref{exponential1} for a geometric Brownian motion and letting $B^*$ denote a Brownian motion under the risk-neutral measure, we have
\begin{align*}
S(T') &= [S(T)-D]\exp\left\{(r-\sigma^2/2)(T'-T)+\sigma B^*(T')-\sigma B^*(T)\right\}\\
&= \left[S(0)\exp\left\{(r-\sigma^2/2)T+\sigma B^*(T)\right\}-D\right]\\
&\qquad \times \exp\left\{(r-\sigma^2/2)(T'-T)+\sigma B^*(T')-\sigma B^*(T)\right\}\\
&=S(0)\exp\left\{(r-\sigma^2/2)T'+\sigma B^*(T')\right\} \\
&\qquad - D\exp\left\{(r-\sigma^2/2)(T'-T)+\sigma B^*(T')-\sigma B^*(T)\right\}\;.
\end{align*}
Thus, $S$ will be a sum of lognormal random variables.  A sum of lognormals is not itself lognormal, so $S$ will not be lognormal, and we are unable to calculate the option value in a simple way.

We will assume instead that $Z$ has a constant volatility $\sigma$.  Thus, $Z$ is the price of a non-dividend-paying portfolio, it satisfies the Black-Scholes assumptions, and we have $S(T')=Z(T')$.  To value a European option, we would simply use $Z(0)=S(0)-\E^{-rT}D$ as the initial asset price and $\sigma$ as the volatility.

\subsection*{American Call Payoff}
If the call is not exercised before the dividend is paid at date $T$, then its value at date $T$ will be
\begin{center}
\verb!Black_Scholes_Call(Z,K,r,sigma,0,Tprime-T)!
\end{center}
where  \verb!Z! $=Z(T)$.  Hence, exercise is optimal when
\begin{center}
\verb!Z! $+$ \verb!D! $-$ \verb!K! $>$ \verb!Black_Scholes_Call(Z,K,r,sigma,0,Tprime-T)!\;.
\end{center}
A lower bound for the Black-Scholes call value on the right-hand side is $Z(T)-\E^{-r(T'-T)}K$.  If $Z(T)+D-K$ is less than or equal to this lower bound, then exercise cannot be optimal.  Thus, if $D-K$ is less than or equal to $-\E^{-r(T'-T)}K$, then exercise will never be optimal.  In this circumstance, the dividend is simply too small to offset the time value of money on the exercise price $K$,  and the value of the American call written on the asset with price $S$ is the same as the value of the European call written on the non-dividend-paying portfolio with price $Z$.  

On the other hand, if $D-K > -\E^{-r(T'-T)}K$, then exercise will be optimal for sufficiently large $Z(T)$.  In this case, there is some
$Z^*$ such that the owner of the call will be indifferent about exercise, and exercise will be optimal for \vfil\eject
all $Z(T)>Z^*$.  This $Z^*$ is defined by
\begin{center}
\verb!Zstar! $+$ \verb!D! $-$ \verb!K! $=$ \verb!Black_Scholes_Call(Zstar,K,r,sigma,0,Tprime-T)!\;.
\end{center}
As in the previous section, we can compute $Z^*$ by bisection.  

Define
\begin{align*}
x&= \begin{cases} 1 &\text{if $Z(T)>Z^*$\;,}\\
                               0 & \text{otherwise\;,}
        \end{cases}\\
y&= \begin{cases} 1 &\text{if $Z(T)\leq Z^*$ and $Z(T')>K$\;,}\\
                               0 & \text{otherwise\;.}
        \end{cases} 
\end{align*} 
Then the American call option will pay $[Z(T)+D-K]x$ at date $T$ (due to early exercise) and $[Z(T')-K]y$ at date $T'$ (due to exercise at maturity), if $D-K > -\E^{-r(T'-T)}K$.  

\subsection*{Numeraires}
Assume for now that $D-K > -\E^{-r(T'-T)}K$.  The payoff $(D-K)x$ is the payoff of $D-K$ digital options maturing at~$T$, and the payoff $Z(T)x$ is the payoff of one share digital on the portfolio with price $Z$.  Therefore, the value of receiving  $[Z(T)+D-K]x$ at date $T$ is
$$Z(0)\N(d_1) + \E^{-rT}(D-K)\N(d_2)\; ,$$
where
\begin{subequations}\label{americancallstarcombined}
\begin{align}
d_1 &= \frac{\log\left(\frac{Z(0)}{Z^*}\right)+\left(r+\frac{1}{2}\sigma^2\right)T}{\sigma\sqrt{T}}\notag\\
&= \frac{\log\left(\frac{S(0)-\E^{-rT}D}{Z^*}\right)+\left(r+\frac{1}{2}\sigma^2\right)T}{\sigma\sqrt{T}} \;,\label{americancalld1star}\\
d_2&=d_1-\sigma\sqrt{T}\;.\label{americancalld2star}
\end{align}\end{subequations}
As in the previous section,\footnote{The only difference is that here $Z$ is the price of a non-dividend-paying portfolio, so, in the notation of the previous section, we have $V(t)=Z(t)$.} the value of receiving $[Z(T)-K]y$ at date $T'$ is 
$$Z(0)\times \text{prob}^Z(y=1) - \E^{-rT'}K\times \text{prob}^R(y=1)\; .$$

\subsection*{Calculating Probabilities}
The calculations are very similar to the calculations we did for a call option on a call.  In fact, they are exactly the same as we would do for a put option on a call.
\begin{enumerate}
\item The event $y=1$ is equivalent to
$$\log Z(0) + \left(r+\frac{1}{2}\sigma^2\right)T+\sigma B^*(T) \leq \log Z^*$$
and
$$\log Z(0) + \left(r+\frac{1}{2}\sigma^2\right)T'+\sigma B^*(T') > \log K\; ,$$
where $B^*$ is a Brownian motion when the underlying asset (with price $Z$) is used as the numeraire. 
We can write this as
\begin{equation}\label{new100}
\frac{B^*(T)}{\sqrt{T}} < -d_1 \quad \text{and} \quad - \frac{B^*(T')}{\sqrt{T'}} < d_1'\;,
\end{equation}
where $d_1$ is defined in \eqref{americancalld1star},
\begin{subequations}\label{americancallcombined}
\begin{align}
d_1' &= \frac{\log\left(\frac{Z(0)}{K}\right)+\left(r+\frac{1}{2}\sigma^2\right)T'}{\sigma\sqrt{T'}}\notag\\
&= \frac{\log\left(\frac{S(0)-\E^{-rT}D}{K}\right)+\left(r+\frac{1}{2}\sigma^2\right)T'}{\sigma\sqrt{T'}}\label{americancalld1}\\
d_2'&=d_1'-\sigma\sqrt{T'}\;.\label{americancalld2}
\end{align}\end{subequations}
The two standard normal variables on the left-hand sides in \eqref{new100} have a covariance equal to
$$-\frac{1}{\sqrt{TT'}}\cov(B(T),B(T')) = -\frac{1}{\sqrt{TT'}}\cov(B(T),B(T)) = -\sqrt{\frac{T}{T'}}\; .$$
Hence, $\text{prob}^Z(y=1)$ is the probability that $a\leq -d_1$ and $b\leq d_1'$, where $a$ and $b$ are standard normal random variables with covariance (= correlation coefficient) of $-\sqrt{T/T'}$.  We are writing this probability as $\M\!\left(-d_1,d_1',-\sqrt{T/T'}\right)$.  

\item The calculation for $\text{prob}^R(y=1)$ is similar.  The event $y=1$
is equivalent to
$$\log Z(0) + \left(r+\frac{1}{2}\sigma^2\right)T+\sigma B^*(T) \leq \log Z^*$$
and
$$\log Z(0) + \left(r+\frac{1}{2}\sigma^2\right)T'+\sigma B^*(T') > \log K\; ,$$
where $B^*$ now denotes a Brownian motion under the risk-neutral measure.  These are equivalent to
\begin{equation}\label{new11}
\frac{B^*(T)}{\sqrt{T}} \leq -d_2 \quad \text{and} \quad - \frac{B^*(T')}{\sqrt{T'}} < d_2'\;.
\end{equation}
Hence, $\text{prob}^R(y=1)=\M\!\left(-d_2,d_2',-\sqrt{T/T'}\right)$.  
\end{enumerate}

\subsection*{American Call Pricing Formula}

\mybox{Under our assumptions, the value of an American call option maturing at~$T'$ with a dividend payment of $D$ at date $T<T'$ is as follows.  
If 
$$D-K \leq -\E^{-r(T'-T)}K\;,$$
then the value of the call is given by the Black-Scholes formula
$$[S(0)-\E^{-rT}D]\N(d_1')-\E^{-rT}K\N(d_2')\; ,$$
where $d_1'$ and $d_2'$ are defined in \eqref{americancallcombined}.
On the other hand, if 
$$D-K > -\E^{-r(T'-T)}K\;,$$
then the value of the call is
\begin{multline}
[S(0)-\E^{-rT}D]\N(d_1) + \E^{-rT}(D-K)\N(d_2)\\ +[S(0)-\E^{-rT}D]\M\!\left(-d_1,d_1',-\sqrt{T/T'}\right) \\- \E^{-rT'}K\M\!\left(-d_2,d_2',-\sqrt{T/T'}\right)\;,\label{americancall}
\end{multline}
where $d_1$ and $d_2$ are defined in \eqref{americancallstarcombined} and $d_1'$ and $d_2'$ are defined in \eqref{americancallcombined}.}




\section{Choosers}\label{s_choosers}
A ``chooser option'' \index{chooser} allows the holder to choose whether the option will be a put or call at some fixed date before the option maturity.  Let $T$ denote the date at which the choice is made, $T_c$ the date at which the call expires, $T_p$ the date at which the put expires, $K_c$ the exercise price of the call, and $K_p$ the exercise price of the put, where $0<T<T_c$ and $0<T<T_p$.  A ``simple chooser'' has $T_c=T_p$ and $K_c=K_p$.  A chooser is similar in spirit to a straddle: it is a bet on volatility without making a bet on direction.  A simple chooser must be cheaper than a straddle with the same exercise price and maturity $T'=T_c=T_p$, because a straddle is always in the money at maturity, whereas a simple chooser has the same value as the straddle if it is in the money but is only in the money at $T'$ when the choice made at $T$ turns out to have been the best one.  

\subsection*{Chooser Payoff}
The value of the chooser at date $T$ will be the larger of the call and put prices.  Let $S^*$ denote the stock price at which the call and put have the same value.  We can find $S^*$ by solving
\small\begin{verbatim}
     Black_Scholes_Call(Sstar,Kc,r,sigma,q,Tc-T) 
                    = Black_Scholes_Put(Sstar,Kp,r,sigma,q,Tp-T).                           
\end{verbatim}
\normalsize

\noindent For a simple chooser with $K_c=K_p=K$ and $T_c=T_p=T'$, we can find $S^*$ from the put-call parity relation at $T$, leading to $S^*=\E^{(q-r)(T'-T)}K.$

The call will be chosen when $S(T)>S^*$ and it finishes in the money if $S(T_c)>K_c$ at date $T_c$, so the payoff of the chooser is $S(T_c)-K_c$ when 
$$S(T)>S^* \quad \text{and}\quad S(T_c)>K_c\;.
$$
The payoff is $K_p-S(T_p)$ at date $T_p$ when 
$$
S(T)<S^* \quad \text{and}\quad S(T_p)<K_p\;.
$$
Let 
\begin{align*}
x&= \begin{cases} 1 &\text{if $S(T)>S^* $ and $S(T_c)>K_c$\;,}\\
                               0 & \text{otherwise\;.}
        \end{cases}\\
\intertext{Likewise, let}
y&= \begin{cases} 1 &\text{if $S(T)<S^*$ and $S(T_p)<K_p$\;,}\\
                               0 & \text{otherwise\;.}
        \end{cases} 
\end{align*}
Then the payoff of the chooser is $xS(T_c)-xK_c$ at date $T_c$ and $yK_p-yS(T_p)$ at date $T_p$.

\subsection*{Numeraires}
As in the analysis of compound options, the value of the chooser at date 0 must be 
\begin{multline}
\E^{-q T_c}S(0)\times\text{prob}^V\!(x=1) \;- \;\E^{-rT_c}K_c\times\text{prob}^R(x=1)
\\+\; \E^{-rT_p}K_p\times\text{prob}^R(y=1) \;- \;\E^{-q T_p}S(0)\times\text{prob}^V\!(y=1)\;,\label{chooser1}
\end{multline}
where we use $V(t)=\E^{qt}S(t)$ and $R(t)=\E^{rt}$ as numeraires. 

\subsection*{Chooser Pricing Formula}
Equation \eqref{chooser1} and calculations similar to those of the previous two sections lead us to:

\mybox{The value of a chooser option is 
\begin{multline}\label{chooser2}
\E^{-q T_c}S(0)\M\!\left(d_1,d_{1c},\sqrt{T/T_c}\right) - \E^{-rT_c}K_c\M\!\left(d_2,d_{2c},\sqrt{T/T_c}\right)\\ +\E^{-rT_p}K_p\M\!\left(-d_2,-d_{2p} , \sqrt{T/T_p}\right) \\- \E^{-q T_p}S(0)\M\!\left(-d_1 ,-d_{1p} ,\sqrt{T/T_p}\right)\;,
\end{multline}
where
\begin{equation*}
\begin{array}{ll}
d_1 = \frac{\log\left(\frac{S(0)}{S^*}\right)+\left(r-q+\frac{1}{2}\sigma^2\right)T}{\sigma\sqrt{T}}\;,
& \qquad d_2=d_1-\sigma\sqrt{T}\;,\\
&\\
d_{1c} = \frac{\log\left(\frac{S(0)}{K_c}\right)+\left(r-q+\frac{1}{2}\sigma^2\right)T_c}{\sigma\sqrt{T_c}}\;,
& \qquad d_{2c}=d_{1c}-\sigma\sqrt{T_c}\;,\\
&\\d_{1p} = \frac{\log\left(\frac{S(0)}{K_p}\right)+\left(r-q+\frac{1}{2}\sigma^2\right)T_p}{\sigma\sqrt{T_p}}\;,
& \qquad d_{2p}=d_{1p}-\sigma\sqrt{T_p}\;.
\end{array}
\end{equation*}}



\section{Options on the Max or Min}

We will consider here an option written on the maximum or minimum of two asset prices; for example, a call on the maximum pays
$$\max(0,\max(S_1(T), S_2(T))-K) = \max(0,S_1(T)-K,S_2(T)-K)$$
at maturity $T$.  There are also call options on $\min(S_1(T), S_2(T))$ and put options on the maximum and minimum of two (or more) asset prices.  Pricing formulas for these options are due to Stulz \cite{Stulz}, who also discusses applications.  We will assume the two assets have constant dividend yields $q_i$, constant volatilities $\sigma_i$, and a constant correlation $\rho$.

\vfil\eject
\subsection*{Call-on-the-Max Payoff}
To value the above option, define the random variables:
\begin{align*}
x&= \begin{cases} 1 & \text{if $S_1(T)>S_2(T)$ and $S_1(T)>K$}\; ,\\
0 & \text{otherwise}\;, \end{cases}\\
y&= \begin{cases} 1 & \text{if $S_2(T)> S_1(T)$ and $S_2(T)>K$}\; ,\\
0 & \text{otherwise}\;, \end{cases}\\
z&= \begin{cases} 1 & \text{if $S_1(T) > K$ or $S_2(T)> K$}\; ,\\
0 & \text{otherwise}\;. \end{cases}
\end{align*}
Then the value of the option at maturity is
$$xS_1(T) + yS_2(T) - zK\; .$$

\subsection*{Numeraires}
Consider numeraires  $V_1(t) = \E^{q_1t}S_1(t)$,  $V_2(t)=\E^{q_2t}S_2(t)$, and $R(t)=\E^{rt}$.
By familiar arguments, the value of the option at date 0 is
\begin{multline*}
\E^{-q_1T}S_1(0)\times\text{prob}^{V_1}(x=1) + \E^{-q_2T}S_2(0)\times\text{prob}^{V_2}(y=1) \\- \E^{-rT}K\times\text{prob}^R(z=1)\; .
\end{multline*}

\subsection*{Calculating Probabilities}

\begin{enumerate}\item
We will begin by calculating $\text{prob}^{V_1}(x=1)$.  From the second and third examples in Sect.~\ref{s_girsanov}, the asset prices satisfy
\begin{align*}
\frac{\D S_1}{S_1} &= (r-q_1+\sigma^2_1)\,\D t + \sigma_1\,\D B^*_{1}\; ,\\
\frac{\D S_2}{S_2} &= (r-q_2+\rho\sigma_1\sigma_2)\,\D t + \sigma_2\,\D B^*_{2}\;,
\end{align*}
where $B^*_{1}$ and $B^*_{2}$ are Brownian motions when we use $V_1$ as the numeraire.
Thus,
\begin{align*}
\log S_1(T) &= \log S_1(0) + \left(r-q_1+\frac{1}{2}\sigma_1^2\right)T +\sigma_1B^*_{1}(T)\; ,\\
\log S_2(T) &= \log S_2(0) + \left(r-q_2+\rho\sigma_1\sigma_2-\frac{1}{2}\sigma_2^2\right)T +\sigma_2B^*_{2}(T)\;.
\end{align*}
The condition $\log S_1(T) > \log K$ is therefore equivalent to 
\begin{subequations}\label{max12combined}
 \begin{equation}
 -\frac{1}{\sqrt{T}}B^*_{1}(T) < d_{11}\;,\label{max1}
 \end{equation}
and the condition $\log S_1(T)>\log S_2(T)$ is equivalent to
\begin{equation}
\frac{\sigma_2B^*_{2}(T)-\sigma_1B^*_{1}(T)}{\sigma\sqrt{T}} < d_1\;,\label{max2}
\end{equation}
\end{subequations}
where
\begin{equation}
\sigma =\sqrt{\sigma_1^2-2\rho\sigma_1\sigma_2+\sigma_2^2}\;,\label{max2a}
\end{equation}
and
\begin{subequations}\label{max34combined}\begin{align}
d_1 &= \frac{\log\left(\frac{S_1(0)}{S_2(0)}\right)+\left(q_2-q_1+\frac{1}{2}\sigma^2\right)T}{\sigma\sqrt{T}}\;, \qquad d_2 = d_1 - \sigma\sqrt{T}\;,\label{max4}\\
d_{11}&=\frac{\log\left(\frac{S_1(0)}{K}\right)+\left(r-q_1+\frac{1}{2}\sigma_1^2\right)T}{\sigma_1\sqrt{T}}\;, \qquad d_{12} = d_{11} - \sigma_1\sqrt{T}\;.\label{max3}
\end{align}
 The random variables on the left-hand sides of \eqref{max12combined} have standard normal distributions and their correlation is
$$\rho_1 = \frac{\sigma_1-\rho\sigma_2}{\sigma}\; .$$
Therefore,
\begin{equation*}
\text{prob}^{V_1}(x=1) = \M(d_{11},d_1,\rho_1)\;,
\end{equation*}
where $\M$ again denotes the bivariate normal distribution function.

\item The probability $\text{prob}^{V_2}(y=1)$ is exactly symmetric to $\text{prob}^{V_1}(x=1)$, with the roles of $S_1$ and $S_2$ interchanged. Note that the mirror image of $d_1$ defined in \eqref{max4} is
$$\frac{\log\left(\frac{S_2(0)}{S_1(0)}\right)+\left(q_1-q_2+\frac{1}{2}\sigma^2\right)T}{\sigma\sqrt{T}}\; ,$$
which equals $-d_2$.
Therefore,
\begin{equation*}
\text{prob}^{V_2}(y=1) = \M(d_{21},-d_2,\rho_2)\;,
\end{equation*}
where
\begin{equation}
d_{21}=\frac{\log\left(\frac{S_2(0)}{K}\right)+\left(r-q_2+\frac{1}{2}\sigma_2^2\right)T}{\sigma_2\sqrt{T}},\qquad d_{22} = d_{21}-\sigma_2\sqrt{T}\;,\label{max31}
\end{equation}\end{subequations}
and
$$\rho_2 = \frac{\sigma_2-\rho\sigma_1}{\sigma}\; .$$

\item As usual, we have
\begin{align*}
\log S_1(T) &= \log S_1(0) + \left(r-q_1-\frac{1}{2}\sigma_1^2\right)T +\sigma_1B^*_{1}(T)\; ,\\
\log S_2(T) &= \log S_2(0) + \left(r-q_2-\frac{1}{2}\sigma_2^2\right)T +\sigma_2B^*_{2}(T)\;,
\end{align*}
where $B^*_{1}$ and $B^*_{2}$ now denote Brownian motions under the risk-neutral measure.  The event $z=1$ is the complement of the event
$$S_1(T)\leq K \quad \text{and} \quad S_2(T)\leq K\; ,$$
which is equivalent to 
\begin{subequations}\begin{align}
\frac{1}{\sqrt{T}}B^*_{1}(T) &< -d_{12}\;,\label{max32}
\intertext{and}
\frac{1}{\sqrt{T}}B^*_{2}(T) &< -d_{22}\;.\label{max42}
\end{align}\end{subequations}

The random variables on the left-hand sides of \eqref{max32} and \eqref{max42} are standard normals and have correlation $\rho$.
Therefore,
\begin{equation*}
\text{prob}^{R}(z=1) = 1- \M(-d_{12},-d_{22},\rho)\;.
\end{equation*}
\end{enumerate}

\subsection*{Call-on-the-Max Pricing Formula}

\mybox{
The value of a call option on the maximum of two risky asset prices with volatilities $\sigma_1$ and~$\sigma_2$ and correlation $\rho$ is
\begin{multline}\label{callonmaxformula}
\E^{-q_1T}S_1(0)\M\!\left(d_{11},d_1,\frac{\sigma_1-\rho\sigma_2}{\sigma}\right) + \E^{-q_2T}S_2(0)\M\!\left(d_{21},-d_2,\frac{\sigma_2-\rho\sigma_1}{\sigma}\right)\\+ \E^{-rT}K\M(-d_{12},-d_{22},\rho) - \E^{-rT}K\;,
\end{multline}
where $\sigma$ is defined in \eqref{max2a} and  $d_1$, $d_2$, $d_{11}$, $d_{12}$, $d_{21}$ and $d_{22}$ are defined in \eqref{max34combined}.}

\section{Barrier Options}\label{s_barriers}

\index{barrier option} \index{down-and-out option} \index{down-and-in option} \index{knock-out option} \index{knock-in option} A ``down-and-out'' call pays the usual call value at maturity if and only if the stock price does not hit a specified lower bound during the lifetime of the option.  If it does breach the lower barrier, then it is ``out.''  Conversely, a ``down-and-in'' call pays off only if the stock price \emph{does} hit the lower bound.  Up-and-out and up-and-in calls are defined similarly, and there are also put options of this sort.  The ``out'' versions are called ``knock-outs'' and the ``in'' versions are called ``knock-ins.''  

Knock-ins can be priced from knock-outs and vice-versa.  For example, the combination of a down-and-out call and a down-and-in call creates a standard European call, so the value of a down-and-in can be obtained by subtracting the value of a down-and-out from the value of a standard European call.  Likewise, up-and-in calls can be valued by subtracting the value of an up-and-out from the value of a standard European call.  Both knock-outs and knock-ins are of course less expensive than comparable standard options.


We will describe the pricing of a down-and-out call.  The pricing of up-and-out calls and knock-out puts is similar.  Often there are rebates associated with the knocking-out of a barrier option, but we will not include that feature here (see Sect.~\ref{s_finitedifferencebarriers} however).  

A down-and-out call provides a hedge against an increase in an asset price, just as does a standard call, for someone who is short the asset.  The difference is that the down-and-out is knocked out when the asset price falls sufficiently.  Presumably this is acceptable to the buyer because the need to hedge against high prices diminishes when the price falls.  In fact, in this circumstance the buyer may want  to establish a new hedge at a lower strike.  However, absent re-hedging at a lower strike, the buyer of a knock-out call obviously faces the risk that the price may reverse course after falling to the knock-out boundary, leading to regret that the option was knocked out.  The rationale for accepting this risk is  that the knock-out is cheaper than a standard call.  Thus, compared to a standard call, a down-and-out call provides cheaper but incomplete insurance.

The combination of a knock-out call and a knock-in call (or knock-out puts) with the same barrier and different strikes creates an option with a strike that is reset when the barrier is hit.  This is a hedge that adjusts automatically to the market.  An example is given in Probs.~\ref{e_standardknockout3} and~\ref{e_standardknockout4}.

\subsection*{Down-and-Out Call Payoff}
Let $L$ denote the lower barrier for the down-and-out call and assume it has not yet been breached at the valuation date, which we are calling date 0.  Denote the minimum stock price realized during the remaining life of the contract by $z = \min_{0\leq t\leq T} S(t)$.  In practice, this minimum is calculated at discrete dates (for example, based on daily closing prices), but we will assume here that the stock price is monitored continuously for the purpose of calculating the minimum. 

The down-and-out call will pay $\max(0,S(T)-K)$ if $z > L$ and 0 otherwise, 
at its maturity T.  
Let
$$x = \begin{cases} 1 & \text{if $S(T)>K$ and $z > L$\;,}\\
0 & \text{otherwise\;.} \end{cases}$$
Then the value of the down-and-out call at maturity is
$$xS(T) - xK\; .$$

\subsection*{Numeraires}
As in other cases, the value at date 0 can be written as
$$\E^{-qT}S(0)\times\text{prob}^{V}(x=1) - \E^{-rT}K\times\text{prob}^{R}(x=1)\; ,$$
where $V(t) = \E^{qt}S(t)$ and $R(t)=\E^{rt}$.

\subsection*{Calculating Probabilities}
To calculate $\text{prob}^{V}(x=1)$ and $\text{prob}^{R}(x=1)$, we consider two cases.  
\begin{enumerate}
\item Suppose $K>L$.  Define
$$y = \begin{cases} 1 & \text{if $S(T)>K$ and $z \leq L$}\\
0 & \text{otherwise\;.} \end{cases}$$   The event $S(T)>K$ is equal to the union of the disjoint events $x=1$ and $y=1$.  Therefore,
\begin{align*}
\text{prob}^{V}(x=1) &= \text{prob}^{V}(S(T)\!>\!K) - \text{prob}^{V}(y=1)\; ,\\
\text{prob}^{R}(x=1) &= \text{prob}^{R}(S(T)\!>\!K) - \text{prob}^{R}(y=1)\;.
\end{align*}

As in the derivation of the Black-Scholes formula, we have
\begin{equation}\label{casea2}
\text{prob}^{V}(S(T)\!>\!K) = \N(d_1) \quad \text{and} \quad \text{prob}^{R}(S(T)\!>\!K) = \N(d_2)\;,
\end{equation}
where
\begin{subequations}\label{casea35combined}
\begin{equation}
\label{casea3}
d_1= \frac{\log\left(\frac{S(0)}{K}\right)+\left(r-q+\frac{1}{2}\sigma^2\right)T}{\sigma\sqrt{T}}\; ,\qquad  d_2 = d_1-\sigma\sqrt{T}\;.
\end{equation}
Furthermore , defining 
\begin{equation}
\label{casea5}d_1' = \frac{\log\left(\frac{L^2}{KS(0)}\right)+\left(r-q+\frac{1}{2}\sigma^2\right)T}{\sigma\sqrt{T}}\;, \qquad d_2' = d_1'-\sigma\sqrt{T}\;,
\end{equation}\end{subequations}
it can be shown (see Appendix~\ref{a_minimum}) that
\begin{subequations}\label{casea441combined}
\begin{align}
\text{prob}^{V}(y=1) &= \left(\frac{L}{S(0)}\right)^{2\left(r-q+\frac{1}{2}\sigma^2\right)/\sigma^2}\N(d_1')\;,\label{casea4}\\
\text{prob}^{R}(y=1) &= \left(\frac{L}{S(0)}\right)^{2\left(r-q-\frac{1}{2}\sigma^2\right)/\sigma^2}\N(d_2')\;.\label{casea41}
\end{align}
\end{subequations}


\item Suppose $K \leq L$.  Then the condition $S(T)>K$ in the definition of the event $x=1$ is redundant: if $z > L \geq K$, then it is necessarily true that $S(T)>K$.  Therefore, the probability (under either numeraire) of the event $x=1$ is the probability that $z > L$.  Define
$$y = \begin{cases} 1 & \text{if $S(T)>L$ and $z \leq L$\;,}\\
0 & \text{otherwise\;.} \end{cases}$$
The event $S(T)>L$ is the union of the disjoint events $x=1$ and $y=1$.  Therefore, as in the previous case (but now with $K$ replaced by $L$),
\begin{align*}
\text{prob}^{V}(x=1) &= \text{prob}^{V}(S(T)\!>\!L) - \text{prob}^{V}(y=1)\; ,\\
\text{prob}^{R}(x=1) &= \text{prob}^{R}(S(T)\!>\!L) - \text{prob}^{R}(y=1)\;.
\end{align*}
Also as before, we know that
\begin{equation}\label{caseb2}
\text{prob}^{V}(S(T)\!>\!L) = \N(d_1) \quad \text{and} \quad \text{prob}^{R}(S(T)\!>\!L) = \N(d_2)\;,
\end{equation}
where now
\begin{subequations}\label{caseb35combined}
\begin{equation}\label{caseb3}
d_1= \frac{\log\left(\frac{S(0)}{L}\right)+\left(r-q+\frac{1}{2}\sigma^2\right)T}{\sigma\sqrt{T}}\; ,\qquad  d_2 = d_1-\sigma\sqrt{T}\;.
\end{equation}
Moreover, $\text{prob}^{V}(y=1)$ and $\text{prob}^{R}(y=1)$ are given by \eqref{casea441combined} but with $K$ replaced by $L$, which means that
\begin{equation}\label{caseb5}
d_1' = \frac{\log\left(\frac{L}{S(0)}\right)+\left(r-q+\frac{1}{2}\sigma^2\right)T}{\sigma\sqrt{T}}\;, \qquad d_2'= d_1' - \sigma\sqrt{T}\;.
\end{equation}
\end{subequations}
\end{enumerate}

\subsection*{Down-and-Out Call Pricing Formula}

\mybox{The value of a continuously-sampled down-and-out call option with barrier~$L$ is 
\begin{multline}\label{downout100}
\E^{-qT}S(0)\left[\N(d_1)-\left(\frac{L}{S(0)}\right)^{2\left(r-q+\frac{1}{2}\sigma^2\right)/\sigma^2}\N(d_1')\right]\\ - \E^{-rT}K\left[\N(d_2) - \left(\frac{L}{S(0)}\right)^{2\left(r-q-\frac{1}{2}\sigma^2\right)/\sigma^2}\N(d_2')\right]\;,
\end{multline}
where 
\begin{enumerate}
\renewcommand{\labelenumi}{(\arabic{enumi})}
\item if $K>L$, $d_1$, $d_2$ , $d_1'$ and $d_2'$ are defined in \eqref{casea35combined},
\item if $K\leq L$, $d_1$, $d_2$, $d_1'$ and $d_2'$ are defined in \eqref{caseb35combined}.
\end{enumerate}
}

\section{Lookbacks}\label{s_lookbacks}

\index{lookback option} \index{floating-strike lookback option} \index{fixed-strike lookback option} A ``floating-strike lookback call'' pays the difference between the asset price at maturity and the minimum price realized during the life of the contract.  A ``floating-strike lookback put'' pays the difference between the maximum price over the life of the contract and the price at maturity.  Thus, the floating-strike lookback call allows one to buy the asset at its minimum price, and the floating-strike lookback put allows one to sell the asset at its maximum price.   Of course, one pays upfront for this opportunity to time the market.  These options were first discussed by Goldman, Sosin and Gatto \cite{GSG}.

A ``fixed-strike lookback put'' pays the difference between a fixed strike price and the minimum price during the lifetime of the contract.   Thus, a fixed-strike lookback put and a floating-strike lookback call are similar in one respect:  both enable one to buy the asset at its minimum price.  However, the put allows one to sell the asset at a fixed price whereas the call allows one to sell it at the terminal asset price.  A ``fixed-strike lookback call'' pays the difference between the maximum price and a fixed strike price and is similar to a floating-strike lookback put in the sense that both enable one to sell the asset at its maximum price.  Fixed-strike lookback options were first discussed by Conze and Viswanathan \cite{CV}.
We will discuss the valuation of floating-strike lookback calls. As in the discussion of barrier options, we will assume that the price is continuously sampled for the purpose of computing the minimum.

\subsection*{Floating-Strike Lookback Call Payoff}

As in the previous section, let $z$ denote the minimum stock price realized over the \emph{remaining lifetime of the contract}.  This is not necessarily the minimum stock price realized during the entire lifetime of the contract.  Let $S_{\min}$ denote the minimum stock price realized during the lifetime of the contract up to and including date 0, which is the date at which we are valuing the contract.  The minimum stock price during the \emph{entire lifetime of the contract} will be the smaller of $z$ and $S_{\text{min}}$.  The payoff of the floating strike lookback call is $S(T) - \min\left(z, S_{\text{min}}\right)$.




\subsection*{Calculations}
The value at date 0 of the piece $S(T)$ is simply $\E^{-qT}S(0)$.  Using the result in Appendix~\ref{a_minimum} on the distribution of $z$, it can be shown (see, e.g., Musiela and Rutkowski \cite{MR} for the details) that the value at date 0 of receiving 
$$\min(z, S_{\text{min}})$$
at date $T$ is
\begin{multline*}
\E^{-rT}S_{\text{min}}\N(d_2) -\frac{\sigma^2}{2(r-q)}\left(\frac{S_{\text{min}}}{S(0)}\right)^{2(r-q)/\sigma^2}\E^{-rT}S(0)\N(d_2') \\
+\left(1+ \frac{\sigma^2}{2(r-q)}\right)\E^{-qT}S(0)\N(-d_1)\;.
\end{multline*}
where
\begin{subequations}\label{fslc100combined}
\begin{align}
d_1 = \frac{\log\left(\frac{S(0)}{S_{\text{min}}}\right)+\left(r-q+\frac{1}{2}\sigma^2\right)T}{\sigma\sqrt{T}}\; , &\qquad d_2 = d_1 - \sigma\sqrt{T}\;,\label{fslc100a}\\
d_1' = \frac{\log\left(\frac{S_{\text{min}}}{S(0)}\right)+\left(r-q+\frac{1}{2}\sigma^2\right)T}{\sigma\sqrt{T}}\;, &\qquad d_2'=d_1' - \sigma\sqrt{T} \;.\label{fslc100c}
\end{align}\end{subequations}
Using the fact that $[1-\N(-d_1)]\E^{-qT}S(0)=\E^{-qT}S(0)\N(d_1)$, this implies:

\subsection*{Floating-Strike Lookback Call Pricing Formula}
\mybox{The value at date 0 of a continuously-sampled floating-strike lookback call, given that the minimum price during the lifetime of the contract through date 0 is $S_{\text{min}}$ and the remaining time to maturity is $T$, is
\begin{multline}\label{fslc100}
\E^{-qT}S(0)\N(d_1)-\E^{-rT}S_{\text{min}}\N(d_2) \\+\frac{\sigma^2}{2(r-q)}\left(\frac{S_{\text{min}}}{S(0)}\right)^{2(r-q)/\sigma^2}\E^{-rT}S(0)\N(d_2') \\
-\frac{\sigma^2}{2(r-q)}\E^{-qT}S(0)\N(-d_1)\;,
\end{multline}
where $d_1$, $d_2$, and $d_2'$ are defined in \eqref{fslc100combined}.}

\section{Basket and Spread Options}\label{s_baskets}

A ``spread option'' \index{spread option} \index{basket option} is a call or a put written on the difference of two asset prices.  For example, a spread call will pay at maturity $T$ the larger of zero and $S_1(T)-S_2(T)-K$, where the $S_i$ are the asset prices and $K$ is the strike price of the call.  Spread options can be used by producers to hedge the difference between an input price and an output price.  They are also useful for hedging basis risk.  For example, someone may want to hedge an asset by selling a futures contract on a closely related but not identical asset.  This exposes the hedger to basis risk: the difference in value between the asset and the underlying asset on the futures contract.  A spread call can hedge the basis risk: take $S_1$ to be the value of the asset underlying the futures contract and $S_2$ the value of the asset being hedged.

A spread option is actually an exchange option.  Assuming constant dividend yields $q_1$ and $q_2$,  we can take the assets underlying the exchange option to be as follows
\begin{enumerate} 
\item At date 0, purchase $\E^{-q_1T}$ units of the asset with price $S_1$ and reinvest dividends, leading to a value of $S_1(T)$ at date $T$, 
\item At date 0, purchase $\E^{-q_2T}$ units of the asset with price $S_2$ and invest $\E^{-rT}K$ in the risk-free asset.  Reinvesting dividends and accumulating interest means that we will have $S_2(T)+K$ dollars at date $T$.
\end{enumerate}
However, we cannot apply Margrabe's formula to price spread options, because the second portfolio described above will have a stochastic volatility.  To see this, note that if the price $S_2(t)$ falls to a low level, then the portfolio will consist primarily of the risk-free asset, so the portfolio volatility will be near the volatility of the risk-free asset, which is zero.  On the other hand, if $S_2(t)$ becomes very high, then the portfolio will be weighted very heavily on the stock investment, and its volatility will approach the volatility of $S_2$.

A ``basket option'' is an option written on a portfolio of assets.  For example, someone may want to hedge the change in the value of the dollar relative to a basket of currencies.  A basket option is an alternative to purchasing separate options on each currency.  Generally, the basket option would have a lower premium than the separate options, because an option on a portfolio is cheaper (and pays less at maturity) than a portfolio of options.  

Letting $S_1$, \ldots, $S_n$ denote the asset prices and $w_1$, \ldots, $w_n$ the weights specified by the contract, a basket call would pay
$$\max\left(0,\;\sum_{i=1}^n w_iS_i(T) - K\right)$$
at maturity $T$.  A spread option is actually a special case of a basket option, with $n=2$, $w_1=1$, and $w_2=-1$.  The difficulty in valuing basket options is the same as that encountered in valuing spread options.  The volatility of the basket price $\sum_{i=1}^nw_iS_i(t)$ will vary over time, depending on the relative volatilities of the assets and the price changes in the assets.  For example, consider the case $n=2$ and write $S(t)$ for the basket price $w_1S_1(t)+w_2S_2(t)$.  Then
\begin{align*}
\frac{\D S}{S} &= \frac{w_1\,\D S_1}{S} + \frac{w_2\,\D S_2}{S}\\
&=\frac{w_1S_1}{S}\times \frac{\D S_1}{S_1} + \frac{w_2S_2}{S}\times \frac{\D S_2}{S_2}\;.
\end{align*}
Let $x_i(t)=w_iS_i(t)/S(t)$.  This is the fraction of the portfolio value that the $i$--th asset contributes.  It will vary randomly over time as the prices change.  Letting $\sigma_i$ denote the volatilities of the individual assets and $\rho$ their correlation, the formula just given for $\D S/S$ shows that the instantaneous volatility of the basket price at any date $t$ is
$$\sqrt{x_1^2(t)\sigma_1^2 + 2x_1(t)x_2(t)\rho\sigma_1\sigma_2 + x_2^2(t)\sigma_2^2}\; .$$
Hence, the volatility will vary randomly over time as the $x_i$ change.  As in the case of spread options, there is no simple closed-form solution for the value of a basket option.

\section{Asian Options}\label{s_asians}

An ``Asian option'' \index{Asian option} \index{average-price option} \index{average-strike option} is an option the value of which depends on the average underlying asset price during the lifetime of the option.  ``Average-price'' calls and puts are defined like standard calls and puts but with the final asset price replaced by the average price.  ``Average-strike'' calls and puts are defined like standard calls and puts but with the exercise price replaced by the average asset price.   A firm that must purchase an input at frequent intervals or will sell a product in a foreign currency at frequent intervals can use an average price option as an alternative to buying multiple options with different maturity dates.  The average-price option will generally be both less expensive and a better hedge than purchasing multiple options.

In practice, the average price is computed by averaging over the prices sampled at a finite number of discrete dates.  First, we consider the case of continuous sampling.  With continuous sampling, the average price at date $T$ for an option written at date 0 will be denoted by $A(T)$ and is defined as
$$A(T) = \frac{1}{T}\int_0^T S(t)\,\D t\; .$$
To obtain a closed-form solution for the value of an option on the average price, we face essentially the same problem as for basket and spread options:  a sum of lognormally distributed variables is not itself lognormally distributed.  In this case, the integral, which is essentially a continuous sum of the prices at different dates, is not lognormally distributed.  

An alternative contract would replace the average price with the ``geometric average.''  \index{geometric average} \index{geometric-average option} This is defined as the exponential of the average logarithm.  We will denote this by $A_\text{g}(T)$.  The average logarithm is
$$\frac{1}{T}\int_0^T \log S(t)\,\D t\; ,$$
and the geometric average is
$$A_\text{g}(T) = \exp\left(\frac{1}{T}\int_0^T \log S(t)\,\D t\right)\; .$$
The concavity of the logarithm function guarantees that
$$\log \frac{1}{T}\int_0^T S(t) > \frac{1}{T}\int_0^T \log S(t)\,\D t \; .$$
Therefore,
\begin{align*}
A(T) &= \exp\left(\log \frac{1}{T}\int_0^T S(t)\right)\\
&> \exp\left(\frac{1}{T}\int_0^T \log S(t)\,\D t\right) \\
&= A_\text{g}(T)\; .
\end{align*}
Consequently, approximating the value of an average-price or average-strike option by substituting $A_\text{g}(T)$ for $A(T)$ will produce a biased estimate of the value.  Nevertheless, the geometric average$A_\text{g}(T)$ and the arithmetic average $A(T)$ will be highly correlated, so $A_\text{g}(T)$ forms a very useful \text{control variate} for Monte-Carlo valuation of average-price and average-strike options, as will be discussed in Chap.~\ref{c_montecarlo}.  To implement the idea, we need a valuation formula for options written on $A_\text{g}(T)$.  We will derive this for an average-price call, in which $A_\text{g}(T)$ substitutes for $A(T)$.

Specifically, consider a contract that pays
$$\max(0,A_\text{g}(T)-K)$$
at its maturity $T$.
This is a ``geometric-average-price call,'', and we will analyze it in the same way that we analyzed quanto options in Chap.~\ref{c_foreignexchange}.  Let $V(t)$ denote the value at date $t$ of receiving $A_\text{g}(T)$ at date $T$.  This can be calculated, and the result will be given below.  $V(t)$ is the value of a non-dividend-paying portfolio, and, by definition, $V(T)=A_\text{g}(T)$, so the geometric-average-price call is equivalent to a standard call with~$V$ being the price of the underlying.  We will show that $V$ has a time-varying but non-random volatility.  Therefore, we can apply the Black-Scholes formula, inputting the average volatility as described in Sect.~\ref{s_timevaryingvolatility}, to value the geometric-average-price call.  We could attempt the same route to price average-price options, but we would find that the volatility of the corresponding value~$V$ would vary randomly, just as we found the basket portfolio to have a random volatility in the previous section.

The value $V(t)$ can be calculated as
$$V(t)= \E^{-r(T-t)}E^R_t\big[A_\text{g}(T)\big]\; .$$
Define
$$A_g(t) = \exp\left(\frac{1}{t}\int_0^t \log S(u)\,du\right)\; .$$
We will verify at the end of this section that
\begin{equation}\label{geometricaveragev1}
V(t) = \E^{-r(T-t)}A_g(t)^{\frac{t}{T}}S(t)^{\frac{T-t}{T}}\exp\left(\frac{(r-q-\sigma^2/2)(T-t)^2}{2T} + \frac{\sigma^2(T-t)^3}{6T^2}\right)\;.
\end{equation}
Two points are noteworthy.  First, the value at date 0 is
\begin{align}
V(0) &= \E^{-rT}S(0)\exp\left(\frac{(r-q-\sigma^2/2)T}{2} + \frac{\sigma^2T}{6}\right)\notag\\
&=\exp\left(-\frac{6r+6q + \sigma^2}{12}T\right)S(0)\;.\label{asianV0}
\end{align}
Second, the volatility comes from the factor
$$S(t)^{\frac{T-t}{T}}\; ,$$
and, by It\^o's formula,
$$\frac{ \D S^{\frac{T-t}{T}}}{S^{\frac{T-t}{T}}} = \text{something}\,\,\D t + \left(\frac{T-t}{T}\right)\sigma\,\D B\; .$$
This implies that the average volatility, in the sense of Sect.~\ref{s_timevaryingvolatility}, is
$$\sigma_{\text{avg}} = \sqrt{\frac{1}{T}\int_0^T \left(\frac{T-t}{T}\right)^2\sigma^2\,\D t}
=\frac{\sigma}{\sqrt{3}}\; .$$
Applying the Black-Scholes formula yields:
\mybox{
The value at date 0 of a continuously-sampled geometric-average-price call written at date~0 and having $T$ years to maturity is 
$$V(0)\N(d_1)-\E^{-rT}K\N(d_2)\; ,$$
where
$$d_1 = \frac{\log\left(\frac{V(0)}{K}\right)+\left(r+\frac{1}{2}\sigma_{\text{avg}}^2\right)T}{\sigma_{\text{avg}}\sqrt{T}}, \qquad d_2 = d_1 - \sigma_{\text{avg}}\sqrt{T}\; ,$$
$V(0)$ is defined in \eqref{asianV0}, and $\sigma_{\text{avg}}=\sigma/\sqrt{3}$.
}

We can also value a discretely-sampled geometric-average-price call by the same arguments.  Consider dates $0<t_0< t_1 < \cdots t_N=T$, where $t_i-t_{i-1}=\varDelta t$ for each $i$ and suppose the price is to be sampled  at the dates $t_1,\ldots,t_N$.  Now let $V(t)$ denote the value at date $t$ of the contract that pays
\begin{equation}\label{geometricaveragev0}
\exp\left(\frac{1}{N} \sum_{i=1}^N \log S(t_i)\right) = \left(\prod_{i=1}^N S(t_i)\right)^{1/N}
\end{equation}
at date $T$.  The call option will pay $\max(0,V(T)-K)$ at date $T$.  Let $k$ denote the integer such that $t_{N-k-1} \leq t < t_{N-k}$.  This means that we have already observed the prices $S(t_1), \ldots, S(t_{N-k-1})$ and we have yet to observe the $k+1$ prices $S(t_{N-k}), \ldots ,S(t_N)$.  Define $\varepsilon = (t_{N\!-k}-t)/\varDelta t$, which is fraction of the interval $\varDelta t$ that must pass before we reach the next sampling date $t_{N\!-k}$.  We will show at the end of this section that 
\begin{multline}\label{geometricaveragev2}
V(t) = \E^{-r(T-t)}S(t)^{\frac{k+1}{N}}\prod_{i=1}^{N-k-1}S(t_i)^{\frac{1}{N}}\\
\times\,\exp\left(\left[ \frac{(k+1)\varepsilon\nu}{N}\!+\! \frac{k(k+1)\nu}{2N}\! +\! \frac{(k+1)^2\sigma^2\varepsilon}{2N^2}\!+\!\frac{k(k+1)(2k+1)\sigma^2}{12N^2} \right]\varDelta t\right)\;,
\end{multline}
where $\nu = r-q-\sigma^2/2$

Again, two points are noteworthy.  Assume the call was written at date 0 and the first observation date $t_1$ is $\varDelta t$ years away.  Then, we have $k+1=N$ and $\varepsilon=1$ so
\begin{equation}\label{asianV02}
V(0) = \E^{-rT}S(0)\exp\left( \frac{(N+1)\nu\varDelta t}{2} + \frac{(N+1)(2N+1)\sigma^2\varDelta t}{12N} \right)\;.
\end{equation}
Second, the volatility of $V(t)$ comes from the factor $S(t)^{(k+1)/N}$, and
$$\frac{ \D S^{\frac{k+1}{N}}}{S^{\frac{k+1}{N}}} = \text{something}\,\,\D t + \left(\frac{k+1}{N}\right)\sigma\,\D B\; .$$
This implies that the average volatility, in the sense of Sect.~\ref{s_timevaryingvolatility}, is
\begin{align}
\sigma_{\text{avg}} &= \sqrt{\frac{1}{N}\sum_{k=0}^{N-1} \left(\frac{k+1}{N}\right)^2\sigma^2\,\D t}\notag\\
&=\frac{\sigma}{N^{3/2}}\sqrt{\frac{N(N+1)(2N+1)}{6}}\;,\label{asiansigmaavg}
\end{align}
where we have used the fact that $\sum_{i=1}^N i^2 = N(N+1)(2N+1)/6$ to obtain the second equality.
Thus, the Black-Scholes formula implies:
\mybox{
The value at date 0 of a discretely-sampled geometric-average-price call written at date~0 and having $T$ years to maturity is 
\begin{equation}\label{disc_geom_avg_call}
V(0)\N(d_1)-\E^{-rT}K\N(d_2)\;,
\end{equation}
where
$$d_1 = \frac{\log\left(\frac{V(0)}{K}\right)+\left(r+\frac{1}{2}\sigma_{\text{avg}}^2\right)T}{\sigma_{\text{avg}}\sqrt{T}}, \qquad d_2 = d_1 - \sigma_{\text{avg}}\sqrt{T}\; ,$$
$V(0)$ is defined in \eqref{asianV02},
and $\sigma_{\text{avg}}$ is defined in \eqref{asiansigmaavg}.
}
This formula will be used in Sect.~\ref{s_controlvariates} as a control variate for pricing discretely-sampled average-price calls (even average-price calls that were written before the date of valuation).

\begin{petit}
We will now derive equations \eqref{geometricaveragev1} and \eqref{geometricaveragev2}.  We will begin with \eqref{geometricaveragev1}.
The random variable $A_g(T)$  is normally distributed under the risk-neutral measure given information at time $t$.  To establish this, and to calculate the mean and variance of $A_g(T)$, the key is to change the order of integration in the integral in the second line below to obtain the third line:
\begin{align*}
\int_t^T \log S(u)\,du &= \int_t^T \left\{\log S(t) + \left(r-q-\frac{1}{2}\sigma^2\right)(u-t) + \sigma [B(u)-B(t)]\right\}\,du\\
&= (T-t)\log S(t) + \left(r-q-\frac{1}{2}\sigma^2\right)\frac{(T-t)^2}{2} + \sigma\int_t^T \int_t^u \D B(s)\,du\\
&= (T-t)\log S(t) + \left(r-q-\frac{1}{2}\sigma^2\right)\frac{(T-t)^2}{2} + \sigma\int_t^T \int_s^T du\,\D B(s)\\
&= (T-t)\log S(t) + \left(r-q-\frac{1}{2}\sigma^2\right)\frac{(T-t)^2}{2} + \sigma\int_t^T (T-s)\,\D B(s)
\end{align*}
and then to note that $\int_t^T (T-s)\,\D B(s)$ is normally distributed with mean zero and variance equal to 
$$\int_t^T (T-s)^2\,ds =\frac{(T-t)^3}{3}\; .$$
Therefore $E^R_t\left[A_g(T)\right]$ is the expectation of the exponential of a normally distributed random variable.  Equation~\eqref{geometricaveragev1} now follows from the fact that if $x$ is normally distributed with mean $\mu$ and variance $\sigma^2$ then $E\left[\E^x\right] = \E^{\mu+\sigma^2/2}$.  


To establish \eqref{geometricaveragev2}, note that the discounted risk-neutral expectation of \eqref{geometricaveragev0}, conditional on having observed $S(t_1), \ldots, S(t_{N-k-1})$, is
\begin{align}
V(t) &= \E^{-r(T-t)}E^R_t \left[\exp\left(\frac{1}{N} \sum_{i=1}^N \log S(t_i)\right)\right]\notag\\
&= \E^{-r(T-t)}\exp\left(\frac{1}{N}\sum_{i=1}^{N-k-1} \log S(t_i)\right)\times E^R_t \left[\exp\left(\frac{1}{N} \sum_{i=N-k}^N \log S(t_i)\right)\right]\notag\\
&=\left(\prod_{i=1}^{N-k-1}S(t_i)^{\frac{1}{N}}\right)\times \E^{-r(T-t)}E^R_t \left[\exp\left(\frac{1}{N} \sum_{i=N-k}^N \log S(t_i)\right)\right]\;.\label{geometricaveragev3}
\end{align}
Let  $\varDelta_0B = B(t_{N-k})-B(t)$ and $\varDelta_iB = B(t_{N\!-k+i})-B(t_{N\!-k+i-1})$ for $i \geq 1$.  We can write the sum of logarithms inside the expectation above as
\begin{multline*}
\sum_{i=0}^{k}\big\{[\log S(t) + (t_{N-k+i}-t)\nu + \sigma [B(t_{N-k+i})-B(t)]\big\}\\
=(k+1)\log S(t) + \sum_{i=0}^{k} (\varepsilon + i)\nu\varDelta t + \sigma\sum_{i=0}^{k} [\varDelta_0B + \varDelta_1B + \cdots + \varDelta_iB] \\
=(k+1)\log S(t) + (k+1)\varepsilon\nu\varDelta t + \frac{k(k+1)}{2}\nu\varDelta t + \sigma\sum_{i=0}^{k} (k+1-i)\varDelta_iB\;,
\end{multline*}
where to obtain the last equality we used the fact that $\sum_{i=0}^k i = k(k+1)/2$.
The random variables $\varDelta_iB$ are normally distributed with mean zero and variance $\varDelta t$ (the variance is $\varepsilon \varDelta t$ for $i=0$).  Thus, the sum of logarithms is
 a normally distributed random variable with mean
$$(k+1)\log S(t) + (k+1)\varepsilon\nu\varDelta t + \frac{k(k+1)}{2}\nu\varDelta t$$
and variance
$$(k+1)^2\sigma^2\varepsilon\varDelta t + \sigma^2\sum_{i=1}^{k} (k+1-i)^2\varDelta t = (k+1)^2\sigma^2\varepsilon\varDelta t +\frac{k(k+1)(2k+1)\sigma^2}{6}\; ,$$
using the fact that $\sum_{i=1}^k i^2 = k(k+1)(2k+1)/6$.
The expectation of the exponential of a normally distributed random variable equals the exponential of its mean plus one-half of its variance, and the exponential of $(k+1)\log S(t)/N$ is $S(t)^{(k+1)/N}$.  Therefore the conditional expectation in \eqref{geometricaveragev3} is
$$S(t)^{\frac{k+1}{N}}\exp\left(\left[ \frac{(k+1)\varepsilon\nu}{N}+ \frac{k(k+1)\nu}{2N} + \frac{(k+1)^2\sigma^2\varepsilon}{2N^2}+\frac{k(k+1)(2k+1)\sigma^2}{12N^2} \right]\varDelta t\right)\; ,$$
which implies \eqref{geometricaveragev2}.
\end{petit}
\section{Calculations in VBA}\label{s_exotics_matlab}

The new features in the option pricing formulas in this chapter are the use of the bivariate normal distribution function and sometimes the need to compute a critical (at-the-money) value of the underlying asset price.  We will compute the critical values by bisection, in the same way that we computed implied volatilities for the Black-Scholes formula in Chap.~\ref{c_blackscholes}.

\subsection*{Bivariate Normal Distribution Function}

The following is a fast approximation of the bivariate \index{bivariate normal distribution function} cumulative normal distribution function, accurate to six decimal places, due to Drezner \cite{Drezner}).  For given numbers $a$ and $b$, this function gives the probability that $\xi_1<a$ and $\xi_2<b$ where $\xi_1$ and $\xi_2$ are standard normal random variables with a given correlation $\rho$, which we must input.  

\addcontentsline{lof}{figure}{BiNormalProb}
\small\begin{verbatim}
Function BiNormalProb(a, b, rho)
Dim a1, b1, sum, z1, Z2, z3, rho1, rho2, Delta, x, y, i, j
x = Array(0.24840615, 0.39233107, 0.21141819, _
        0.03324666, 0.00082485334)
y = Array(0.10024215, 0.48281397, 1.0609498, _
        1.7797294, 2.6697604)
a1 = a / Sqr(2 * (1 - rho ^ 2))
b1 = b / Sqr(2 * (1 - rho ^ 2))
If a <= 0 & b <= 0 & rho <= 0 Then
    sum = 0
    For i = 0 To 4
        For j = 0 To 4
            z1 = a1 * (2 * y(i) - a1)
            Z2 = b1 * (2 * y(j) - b1)
            z3 = 2 * rho * (y(i) - a1) * (y(j) - b1)
            sum = sum + x(i) * x(j) * Exp(z1 + Z2 + z3)
        Next j
    Next i
    BiNormalProb = sum * Sqr(1 - rho ^ 2) / Application.Pi
ElseIf a <= 0 & b >= 0 & rho >= 0 Then
    BiNormalProb = Application.NormSDist(a)-BiNormalProb(a,-b,-rho)
ElseIf a >= 0 & b <= 0 & rho >= 0 Then
    BiNormalProb = Application.NormSDist(b)-BiNormalProb(-a,b,-rho)
ElseIf a >= 0 & b >= 0 & rho <= 0 Then
    sum = Application.NormSDist(a) + Application.NormSDist(b)
    BiNormalProb = sum - 1 + BiNormalProb(-a, -b, rho)
ElseIf a * b * rho > 0 Then
    rho1 = (rho*a-b) * Sgn(a) / Sqr(a^ 2 - 2*rho*a*b + b^ 2)
    rho2 = (rho*b-a) * Sgn(b) / Sqr(a^2 - 2*rho*a*b + b^ 2)
    Delta = (1 - Sgn(a) * Sgn(b)) / 4
    BiNormalProb = BiNormalProb(a,0,rho1) _
                   +BiNormalProb(b,0,rho2) - Delta
End If
End Function
\end{verbatim}\normalsize

\noindent Notice that this function calls itself.  This is an example of ``recursion.''


\subsection*{Forward-Start Call}

The forward-start call pricing formula is of the same form as the Black-Scholes, Margrabe, Black, and Merton formulas, as discussed in Sect.~\ref{s_matlabimplementations}.  We can compute it with our \verb!Generic_Option! pricing function.

\addcontentsline{lof}{figure}{Forward Start Call}
\small\begin{verbatim}
Function Forward_Start_Call(S, r, sigma, q, Tset, TCall)
'
' Inputs are S = initial stock price
'            r = risk-free rate
'            sigma = volatility
'            q = dividend yield
'            Tset = time until the strike is set
'            TCall = time until call matures >= Tset
'
Dim P1, P2
P1 = Exp(-q * TCall) * S
P2 = Exp(-q * Tset - r * (TCall - Tset)) * S
Forward_Start_Call = Generic_Option(P1, P2, sigma, TCall - Tset)
End Function
\end{verbatim}\normalsize


\subsection*{Call on a Call}

We will use bisection to find the critical price $S^*$.  We can use $e^{q(T'-T)}(K+K')$ as an upper bound for $S^*$ and 0 as a lower bound.\footnote{We set the value of the call to be zero when the stock price is zero.  The upper bound works because (by put-call parity and the fact that the put value is nonnegative) $C(T,S) \geq e^{-q(T'-T)}S-e^{-r(T'-T)}K'$.  Therefore, when $S = e^{q(T'-T)}(K+K')$, we have 
$C(T,S) \geq K + K' - e^{-r(T'-T)}K' > K$.
}  
The following uses $10^{-6}$ as the error tolerance in the bisection.

\addcontentsline{lof}{figure}{Call on Call}
\small\begin{verbatim}
Function Call_On_Call(S, Kc, Ku, r, sigma, q, Tc, Tu)
Dim tol, lower, upper, guess, flower, fupper, fguess, Sstar
Dim d1, d2, d1prime, d2prime, rho, N2, M1, M2
'
' Inputs are S = initial stock price
'            Kc = strike price of compound call
'            Ku = strike price of underlying call option
'            r = risk-free rate
'            sigma = volatility
'            q = dividend yield
'            Tc = time to maturity of compound call
'            Tu = time to maturity of underlying call >= Tc
'
' The first step is to find Sstar.
'
tol = 10 ^ -6
lower = 0
upper = exp(q * T) * (Kc + Ku)
guess = 0.5 * lower + 0.5 * upper
flower = -Kc
fupper = Black_Scholes_Call(upper, Ku, r, sigma, q, Tu - Tc) - Kc
fguess = Black_Scholes_Call(guess, Ku, r, sigma, q, Tu - Tc) - Kc
Do While upper - lower > tol
    If fupper * fguess < 0 Then
        lower = guess
        flower = fguess
        guess = 0.5 * lower + 0.5 * upper
        guess = Black_Scholes_Call(guess,Ku,r,sigma,q,Tu-Tc)-Kc
    Else
        upper = guess
        fupper = fguess
        guess = 0.5 * lower + 0.5 * upper
        fguess = Black_Scholes_Call(guess,Ku,r,sigma,q,Tu-Tc)-Kc
    End If
Loop
Sstar = guess
'
' Now we calculate the probabilities.
'
d1 = (Log(S/Sstar) + (r-q+sigma^2/2)*Tc) / (sigma*Sqr(Tc))
d2 = d1 - sigma * Sqr(Tc)
d1prime = (Log(S/Ku) + (r-q+sigma^2/2) * Tu) / (sigma*Sqr(Tu))
d2prime = d1prime - sigma * Sqr(Tu)
rho = Sqr(Tc / Tu)
N2 = Application.NormSDist(d2)
M1 = BiNormalProb(d1, d1prime, rho)
M2 = BiNormalProb(d2, d2prime, rho)
'
' Now we calculate the option price.
'
Call_On_Call = -Exp(-r * Tc) * Kc * N2 + Exp(-q * Tu) * S * M1 _
               -Exp(-r * Tu) * Ku * M2
End Function
\end{verbatim}\normalsize

\subsection*{Call on a Put}

The implementation of the call-on-a-put formula is of course very similar to that of a call-on-a-call.  One difference is that there is no obvious upper bound for $S^*$, so we start with $2K'$ (= \verb!2*K2!) and double this until the value of the put is below $K$.  We can take 0 again to be the lower bound.  Recall that we assume $K<\E^{-r(T'-T)}K'$ and the right-hand side of this is the value of the put at date $T$ when $S(T)=0$.

\addcontentsline{lof}{figure}{Call on Put}
\small\begin{verbatim}
Function Call_On_Put(S, Kc, Ku, r, sigma, q, Tc, Tu)
Dim tol, lower, flower, upper, fupper, guess, fguess, Sstar
Dim d1, d2, d1prime, d2prime, rho, N2, M1, M2
'
' Inputs are S = initial stock price
'            Kc = strike price of compound call
'            Ku = strike price of underlying put option
'            r = risk-free rate
'            sigma = volatility
'            q = dividend yield
'            Tc = time to maturity of compound call
'            Tu = time to maturity of underlying put >= Tc
'
tol = 10 ^ -6
lower = 0
flower = Exp(-r * (Tu - Tc)) * Ku - Kc
upper = 2 * Ku
'
'  We double upper until the put value is below Kc
'
fupper = Black_Scholes_Put(upper,Ku,r,sigma,q,Tu-Tc)-Kc
Do While fupper > 0
    upper = 2 * upper
    fupper = Black_Scholes_Put(upper,Ku,r,sigma,q,Tu-Tc)-Kc
Loop
'
' Now we do the bisection to find Sstar
'
guess = 0.5 * lower + 0.5 * upper
fguess = Black_Scholes_Put(guess,Ku,r,sigma,q,Tu-Tc)-Kc
Do While upper - lower > tol
    If fupper * fguess < 0 Then
        lower = guess
        flower = fguess
        guess = 0.5 * lower + 0.5 * upper
        fguess = Black_Scholes_Put(guess,Ku,r,sigma,q,Tu-Tc)-Kc
    Else
        upper = guess
        fupper = fguess
        guess = 0.5 * lower + 0.5 * upper
        fguess = Black_Scholes_Put(guess,Ku,r,sigma,q,Tu-Tc)-Kc
    End If
Loop
Sstar = guess
'
' Now we calculate the probabilities.
'
d1 = (Log(S/Sstar) + (r-q+sigma^2/2)*Tc) / (sigma*Sqr(Tc))
d2 = d1 - sigma * Sqr(Tc)
d1prime = (Log(S/Ku) + (r-q+sigma^2/2)*Tu) / (sigma*Sqr(Tu))
d2prime = d1prime - sigma * Sqr(Tu)
rho = Sqr(Tc / Tu)
N2 = Application.NormSDist(-d2)
M1 = BiNormalProb(-d1, -d1prime, rho)
M2 = BiNormalProb(-d2, -d2prime, rho)
'
' Now we calculate the option price.
'
Call_On_Put = -Exp(-r * Tc) * Kc * N2 + Exp(-r * Tu) * Ku * M2 _
            - Exp(-q * Tu) * S * M1
End Function
\end{verbatim}\normalsize

\subsection*{American Calls with Discrete Dividends}

To value an American call when there is one dividend payment before the option matures, we input the initial asset price $S(0)$ and then compute $Z(0)=X(0)-\E^{-rT}D$.  If $D-K \leq -\E^{-r(T'-T)}K$, we return the Black-Scholes value of a European call written on $Z$.  Otherwise, we need to compute $Z^*$ and our bisection algorithm requires an upper bound for $Z^*$, which would be any value of $Z(T)$ such that exercise at $T$ is optimal.  It is not obvious what this should be, so we start with $K$ and keep doubling this until we obtain a value of $Z(T)$ at which exercise would be optimal.  Then, we use the bisection algorithm to compute $Z^*$ and finally compute the option value \eqref{americancall}.
\addcontentsline{lof}{figure}{American Call Dividend}
\small\begin{verbatim}
Function American_Call_Dividend(S, K, r, sigma, Div, TDiv, TCall)
'
' Inputs are S = initial stock price
'            K = strike price
'            r = risk-free rate
'            sigma = volatility
'            Div = cash dividend
'            TDiv = time until dividend payment
'            TCall = time until option matures >= TDiv
'
Dim LessDiv, upper, tol, lower, flower, fupper, guess, fguess
Dim LessDivStar, d1, d2, d1prime, d2prime, rho, N1, N2, M1, M2
LessDiv = S - Exp(-r * TDiv) * Div          ' called Z in text
If Div / K <= 1 - Exp(-r * (TCall - TDiv)) Then  
    American_Call_Dividend = _
       Black_Scholes_Call(LessDiv, K, r, sigma, 0, TCall)
    Exit Function
End If
'
' Now we find an upper bound for the bisection.
'
upper = K
Do While upper + Div - K < _
               Black_Scholes_Call(upper,K,r,sigma,0,TCall-TDiv)
   upper = 2 * upper
Loop
'
' Now we use bisection to compute Zstar = LessDivStar.
'
tol = 10 ^ -6
lower = 0
flower = Div - K
fupper = upper + Div - K _
         - Black_Scholes_Call(upper,K,r,sigma,0,TCall-TDiv)
guess = 0.5 * lower + 0.5 * upper
fguess = guess + Div - K _
         - Black_Scholes_Call(guess,K,r,sigma,0,TCall-TDiv)
Do While upper - lower > tol
    If fupper * fguess < 0 Then
        lower = guess
        flower = fguess
        guess = 0.5 * lower + 0.5 * upper
        fguess = guess + Div - K _
               - Black_Scholes_Call(guess,K,r,sigma,0,TCall- Div)
    Else
        upper = guess
        fupper = fguess
        guess = 0.5 * lower + 0.5 * upper
        fguess = guess + Div - K _
               - Black_Scholes_Call(guess,K,r,sigma,0,TCall-TDiv)
    End If
Loop
LessDivStar = guess
'
' Now we calculate the probabilities and the option value.
'
d1 = (Log(LessDiv / LessDivStar) _
   + (r + sigma ^ 2 / 2) * TDiv) / (sigma * Sqr(TDiv))
d2 = d1 - sigma * Sqr(TDiv)
d1prime = (Log(LessDiv / K) _
        + (r + sigma ^ 2 / 2) * TCall) / (sigma * Sqr(TCall))
d2prime = d1prime - sigma * Sqr(TCall)
rho = -Sqr(TDiv / TCall)
N1 = Application.NormSDist(d1)
N2 = Application.NormSDist(d2)
M1 = BiNormalProb(-d1, d1prime, rho)
M2 = BiNormalProb(-d2, d2prime, rho)
American_Call_Dividend = LessDiv*N1 + Exp(-r*TDiv)*(Div-K)*N2 _
                       + LessDiv*M1 - Exp(-r*TCall)*K*M2
End Function
\end{verbatim}\normalsize

\subsection*{Choosers}

To implement the bisection to compute $S^*$, we can take zero as a lower bound and $K_c+K_p$ as an upper bound.\footnote{We take the call value to be zero and the put value to be $\E^{-r(T_p-T)}K_p$ at date $T$ when the stock price is zero.  To see why the upper bound works, note that when the stock price is $S$ at date $T$,  the call is worth at least $S^*-K_c$ and the put is worth no more than $K_p$; i.e, $C \geq S-K_c$ and $P \leq K_p$.  Therefore, $C-P \geq S-K_c-K_p$.  Hence when $S=K_c+K_p$, we have $C-P\geq 0$. }
\addcontentsline{lof}{figure}{Chooser}
\small\begin{verbatim}
Function Chooser(S, Kc, Kp, r, sigma, q, T, Tc, Tp)
'
' Inputs are S = initial stock price
'            Kc = strike price of call option
'            Kp = strike price of put option
'            r = risk-free rate
'            sigma = volatility
'            Div = cash dividend
'            T = time until choice must be made
'            Tc = time until call matures >= T
'            Tp = time until put matures >= T
'
Dim tol, lower, upper, guess, flower, CallUpper, PutUpper
Dim fupper, CallGuess, Putguess, fguess, Sstar, d1, d2
Dim d1c, d2c, d1p, d2p, rhoc, rhop, M1c, M2c, M1p, M2p
'
' First we find Sstar by bisection.
'
tol = 10 ^ -6
lower = 0
upper = Kc + Kp
guess = 0.5 * Kc + 0.5 * Kp
flower = -Exp(-r * (Tp - T)) * Kp
fupper = Black_Scholes_Call(upper,Kc,r,sigma,q,Tc-T) _
       - Black_Scholes_Put(upper,Kp,r,sigma,q,Tp-T)
fguess = Black_Scholes_Call(guess,Kc,r,sigma,q,Tc-T) _
       - Black_Scholes_Put(guess,Kp,r,sigma,q,Tp-T)
Do While upper - lower > tol
    If fupper * fguess < 0 Then
        lower = guess
        flower = fguess
        guess = 0.5 * lower + 0.5 * upper
        fguess = Black_Scholes_Call(guess,Kc,r,sigma,q,Tc-T) _
               - Black_Scholes_Put(guess,Kp,r,sigma,q,Tp-T)
    Else
        upper = guess
        fupper = fguess
        guess = 0.5 * lower + 0.5 * upper
        fguess = Black_Scholes_Call(guess,Kc,r,sigma,q,Tc-T) _
               - Black_Scholes_Put(guess,Kp,r,sigma,q,Tp-T)
    End If
Loop
Sstar = guess
'
' Now we compute the probabilities and option value.
'
d1 = (Log(S/Sstar) + (r-q+sigma^2/2)*T) / (sigma*Sqr(T))
d2 = d1 - sigma * Sqr(T)
d1c = (Log(S/Kc) + (r-q+sigma^2/2)*Tc) / (sigma*Sqr(Tc))
d2c = d1c - sigma * Sqr(Tc)
d1p = (Log(S/Kp) + (r-q+sigma^2/2)*Tp) / (sigma*Sqr(Tp))
d2p = d1p - sigma * Sqr(Tp)
rhoc = Sqr(T / Tc)
rhop = Sqr(T / Tp)
M1c = BiNormalProb(d1, d1c, rhoc)
M2c = BiNormalProb(d2, d2c, rhoc)
M1p = BiNormalProb(-d1, -d1p, rhop)
M2p = BiNormalProb(-d2, -d2p, rhop)
Chooser = Exp(-q*Tc)*S*M1c - Exp(-r*Tc)*Kc*M2c _
        + Exp(-r*Tp)*Kp*M2p - Exp(-q*Tp)*S*M1p
End Function
\end{verbatim}\normalsize

\subsection*{Call on the Max}

\addcontentsline{lof}{figure}{Call on Max}
\small\begin{verbatim}
Function Call_On_Max(S1, S2, K, r, sig1, sig2, rho, q1, q2, T)
'
' Inputs are S1 = price of stock 1
'            S2 = price of stock 2
'            K = strike price
'            r = risk-free rate
'            sig1 = volatility of stock 1
'            sig2 = volatility of stock 2
'            rho = correlation
'            q1 = dividend yield of stock 1
'            q2 = dividend yield of stock 2
'            T = time to maturity
'
Dim sigma, d1, d2, d11, d12, d21, d22, rho1, rho2, M1, M2, M3
sigma = Sqr(sig2 ^ 2 - 2 * rho * sig1 * sig2 + sig1 ^ 2)
d1 = (Log(S1/S2) + (q2-q1+sigma^2/2)*T) / (sigma*Sqr(T))
d2 = d1 - sigma * Sqr(T)
d11 = (Log(S1/K) + (r-q1+sig1^2/2)*T) / (sig1*Sqr(T))
d12 = d11 - sig1 * Sqr(T)
d21 = (Log(S2/K) + (r-q2+sig2^2/2)*T) / (sig2*Sqr(T))
d22 = d21 - sig2 * Sqr(T)
rho1 = (sig1 - rho * sig2) / sigma
rho2 = (sig2 - rho * sig1) / sigma
M1 = BiNormalProb(d11, d1, rho1)
M2 = BiNormalProb(d21, -d2, rho2)
M3 = BiNormalProb(-d12, -d22, rho)
Call_On_Max = Exp(-q1 * T) * S1 * M1 + Exp(-q2 * T) * S2 * M2 _
            + Exp(-r * T) * K * M3 - Exp(-r * T) * K
End Function
\end{verbatim}\normalsize

\vfil\eject
\subsection*{Down-and-Out Calls}

\addcontentsline{lof}{figure}{Down And Out Call}
\small\begin{verbatim}
Function Down_And_Out_Call(S, K, r, sigma, q, T, Barrier)
'
' Inputs are S = initial stock price
'            K = strike price
'            r = risk-free rate
'            sigma = volatility
'            q = dividend yield
'            T = time to maturity
'            Barrier = knock-out barrier < S
'
Dim a, b, d1, d2, d1prime, d2prime, N1, N2
Dim N1prime, N2prime, x, y, q1, q2
If K > Barrier Then
    a = S / K
    b = Barrier * Barrier / (K * S)
Else
    a = S / Barrier
    b = Barrier / S
End If
d1 = (Log(a) + (r-q+0.5*sigma^2)*T) / (sigma*Sqr(T))
d2 = d1 - sigma * Sqr(T)
d1prime = (Log(b) + (r-q+0.5*sigma^2)*T) / (sigma*Sqr(T))
d2prime = d1prime - sigma * Sqr(T)
N1 = Application.NormSDist(d1)
N2 = Application.NormSDist(d2)
N1prime = Application.NormSDist(d1prime)
N2prime = Application.NormSDist(d2prime)
x = 1 + 2 * (r - q) / (sigma ^ 2)
y = x - 2
q1 = N1 - (Barrier / S) ^ x * N1prime
q2 = N2 - (Barrier / S) ^ y * N2prime
Down_And_Out_Call = Exp(-q * T) * S * q1 - Exp(-r * T) * K * q2
End Function
\end{verbatim}\normalsize

\subsection*{Floating-Strike Lookbacks}

\addcontentsline{lof}{figure}{Floating Strike Call}
\small\begin{verbatim}
Function Floating_Strike_Call(S, r, sigma, q, T, SMin)
'
' Inputs are S = initial stock price
'            r = risk-free rate
'            sigma = volatility
'            q = dividend yield
'            T = time to maturity
'            Smin = minimum during past life of contract
'
Dim d1, d2, d1prime, d2prime, N1, N2, N2prime, x, y
d1 = (Log(S/SMin) + (r-q+0.5*sigma^2)*T) / (sigma*Sqr(T))
d2 = d1 - sigma * Sqr(T)
d2prime = (Log(SMin/S) + (r-q-0.5*sigma^2)*T) / (sigma*Sqr(T))
N1 = Application.NormSDist(d1)
N2 = Application.NormSDist(d2)
N2prime = Application.NormSDist(d2prime)
x = 2 * (r - q) / (sigma ^ 2)
Floating_Strike_Call = Exp(-q*T)*S*N1 - Exp(-r*T)*SMin*N2 _
                     + (1/x)*(SMin/S)^x * Exp(-r*T)*SMin*N2prime _
                     - (1/x)*Exp(-q*T)*S*(1-N1)
End Function
\end{verbatim}\normalsize


\subsection*{Discretely-Sampled Geometric-Average-Price Calls}

\addcontentsline{lof}{figure}{Discrete Geom Average Price Call}
\small\begin{verbatim}
Function Discrete_Geom_Average_Price_Call(S,K,r,sigma,q,T,N)
'
' Inputs are S = initial stock price
'            K = stock price
'            r = risk-free rate
'            sigma = volatility
'            q = dividend yield
'            T = time to maturity
'            N = number of time periods
'
Dim dt, nu, a, V, sigavg
dt = T / N
nu = r - q - 0.5 * sigma ^ 2
a = N * (N + 1) * (2 * N + 1) / 6
V = Exp(-r*T)*S*Exp(((N+1)*nu/2 + sigma^2*a/(2*N^2))*dt)
sigavg = sigma * Sqr(a) / (N ^ 1.5)
Discrete_Geom_Average_Price_Call = _
               Black_Scholes_Call(V, K, r, sigavg, 0, T)
End Function
\end{verbatim}\normalsize

\section*{Problems}\addcontentsline{toc}{section}{Problems}
\begin{prob} Intuitively, the value of a forward-start call option should be lower the closer is the date~$T$ at which the strike is set to the date $T'$ at which the option matures, because then the option has less time to maturity after being ``created'' at $T$.  Create an Excel worksheet to confirm this.  Allow the user to input $S$, $r$, $\sigma$, $q$, and $T'$.  Compute and plot the value of the option for $T=0.1T'$, $T=0.2T'$, \ldots, $T=0.9T'$.
\end{prob}\begin{prob} Create an Excel worksheet to demonstrate the additional leverage of a call-on-a-call relative to a standard call.  Allow the user to input $S$, $r$, $\sigma$, $q$, and $T'$.  Use the \verb!Black-Scholes_Call! function to compute and output the value $C$ of a European call with strike $K'=S$ (i.e., the call is at the money) and maturity $T'$.  Use the \verb!Call_on_Call! function to compute and output the value of a call option on the call with strike $K=C$ (i.e., the call-on-a-call is at the money) and maturity $T=0.5T'$.  Compute the percentage returns the standard European call and the call-on-a-call would experience if the stock price $S$ instantaneously increased by 10\%.
\end{prob}\begin{prob} Create an Excel worksheet to illustrate the early exercise premium for an American call on a stock paying a discrete dividend.  Allow the user to input $S$, $r$, $\sigma$, and~$T'$.  Take the date of the dividend payment to be $T=0.5T'$ and take the strike price to be $K=S$.  As discussed in Sect.~\ref{s_discrete}, the value of a European call is given by the Black-Scholes formula with $S-\E^{-rT}D$ being the initial asset price and $q=0$ being the constant dividend yield.  Use the function \verb!American_Call_Dividend! to compute the value of an American call for dividends $D=.1S$, \ldots $D=.9S$.  Subtract the value of the European call with the same dividend to obtain the early exercise premium.  Plot the early exercise premium against the dividend $D$.
\end{prob}\begin{prob} Create a VBA function to value a simple chooser (a chooser option in which $K_c=K_p$ and $T_c=T_p$) using put-call parity to compute $S^*$ as mentioned in Sect.~\ref{s_choosers}.  Verify that the function gives the same result as the function \verb!Chooser!.
\end{prob}\begin{prob} Create an Excel worksheet to compare the cost of a simple chooser to that of a straddle (straddle = call + put with same strike and maturity).  Allow the user to input $S$, $r$, $\sigma$, $q$, and $T'$.  Take the time to maturity of the underlying call and put to be $T'$ for both the chooser and the straddle.  Take the strike prices to be $K=S$.  Take the time the choice must be made for the chooser to be $T=0.5T'$.  Compute the cost of the chooser and the cost of the straddle.
\end{prob}\begin{prob} A stock has fallen in price and you are attempting to persuade a client that it is now a good buy.  The client believes it may fall further before bouncing back and hence is inclined to postpone a decision.  To convince the client to buy now, you offer to deliver the stock to him at the end of two months at which time he will pay you the lowest price it trades during the two months plus a fee for your costs.  The stock is not expected to pay a dividend during the next two months.  Assuming the stock actually satisfies the Black-Scholes assumptions, find a formula for the minimum fee that you would require.  (Hint:  It is almost in Sect.~\ref{s_lookbacks}.)  Create an Excel worksheet allowing the user to input $S$, $r$, and $\sigma$  and computing the minimum fee.
\end{prob}\begin{prob} \label{e_standardknockout} Suppose you must purchase 100 units of an asset at the end of a year.  Create an Excel worksheet simulating the asset price and comparing the quality of the following hedges (assuming 100 contracts of each):
\begin{enumerate}
\renewcommand{\labelenumi}{(\alph{enumi})}
\item a standard European call,
\item a down-and-out call in which the knock-out barrier is 10\% below the current price of the asset.
\end{enumerate}
Take both options to be at the money at the beginning of the year.  Allow the user to input $S$, $r$, $\sigma$ and $q$.   Generate 500 simulated end-of-year costs (net of the option values at maturity) for each hedging strategy and create histogram charts to visually compare the hedges.  Note: to create histograms, you will need the Data Analysis add-in, which may be need to be loaded (click Tools/Add Ins).
\end{prob}\begin{prob}\label{e_standardknockout2}
Compute the prices of the options in the previous exercise.  Modify the simulations to compare the end-of-year costs including the costs of the options, adding interest on the option prices to put everything on an end-of-year basis.
\end{prob}\begin{prob} 
Modify Prob.~\ref{e_standardknockout} by including a third hedge: a combination of a down-and-out call as in part (b) of Prob.~\ref{e_standardknockout} and a down-and-in call with knockout barrier and strike 10\% below the current price of the asset.  Note that this combination forms a call option with strike that is reset when the underlying asset price hits a barrier.\label{e_standardknockout3}
\end{prob}\begin{prob}
Modify Prob.~\ref{e_standardknockout2} by including the hedge in Prob.~\ref{e_standardknockout3}.  Value the down-and-in call using the function \verb!Down_And_Out_Call! and the fact that a down-and-out and down-and-in with the same strikes and barriers form a standard option. \label{e_standardknockout4}
\end{prob}\begin{prob} \label{e_averagehedge} Each week you purchase 100 units of an asset, and you want to hedge your total quarterly (13-week) cost.  Create an Excel worksheet simulating the asset price and comparing the quality of the following hedges: 
\begin{enumerate}
\renewcommand{\labelenumi}{(\alph{enumi})}
\item a standard European call maturing at the end of the quarter ($T=0.25$) on 1300 units of the asset, 
\item 13 call options maturing at the end of each week of the quarter, each written on 100 units of the asset, and 
\item a discretely sampled average-price call with maturity $T=0.25$ written on 1300 units of the asset, where the sampling is at the end of each week.  
\item a discretely sampled geometric-average-price call with maturity $T=0.25$ written on 1300 units of the asset, where the sampling is at the end of each week.
\end{enumerate}
Allow the user to input $S$, $r$, $\sigma$ and $q$.  Assume all of the options are at the money at the beginning of the quarter ($K=S$).  Compare the hedges as in Prob.~\ref{e_standardknockout}.
\end{prob}\begin{prob} In the setting of the previous problem, compute the prices of the options in parts (a), (b) and (d).  Modify the simulations in the previous problem to compare the end-of-quarter costs including the costs of the options (adding interest on the option prices to put everything on an end-of-quarter basis).\label{e_averagehedge2}
\end{prob}\begin{prob} Using the put-call parity relation, derive a formula for the value of a forward-start put.
\end{prob}\begin{prob} Derive formula \eqref{callonaput} for the value of a call on a put.
\end{prob}\begin{prob} Complete the derivation of formula \eqref{chooser2} for the value of a chooser option.
\end{prob}\begin{prob} Derive a formula for the value of a put option on the maximum of two risky asset prices.
\end{prob}\begin{prob} Using the result of the preceding exercise and Margrabe's formula, verify that calls and puts (having the same strike $K$ and maturity $T$) on the maximum of two risky asset prices satisfy the following put-call parity relation:
\begin{multline*}
\E^{-rT}K + \text{Value of call on max} \\
= \E^{-q_2T}S_2(0) + \text{Value of option to exchange asset 2 for asset 1} \\+ \text{Value of put on max}\;.
\end{multline*}
\end{prob}
